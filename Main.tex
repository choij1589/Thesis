\RequirePackage{fix-cm} % documentclass 이전에 넣는다.
% oneside : 단면 인쇄용
% twoside : 양면 인쇄용
% ko : 국문 논문 작성
% master : 석사
% phd : 박사
% openright : 챕터가 홀수쪽에서 시작
\documentclass[oneside,phd,openright]{Style/snuthesis}

\include{Style/snutocstyle} % SNU toc style

%%%%%%%%%%%%%%%%%%%%%%%%%%%%%%%%%%%%%%%%
%% 다른 패키지 로드
%% http://faq.ktug.or.kr/faq/pdflatex%B0%FAlatex%B5%BF%BD%C3%BB%E7%BF%EB
%% 필요에 따라 직접 수정 필요
\ifpdf
	\input glyphtounicode\pdfgentounicode=1 %type 1 font사용시
	\usepackage[pdftex]{graphicx}
  \usepackage{subfig}
  \usepackage{xspace}
  \usepackage{amsmath,amssymb,cancel} %% -- To write formula
  \usepackage{multirow} %% -- To use multirow on tables
  \usepackage{Style/pennames} %% -- From CMS note util
  \usepackage{ifthen} %% -- For ptdr-definitions
  \newcommand{\cmsSymbolFace}[1]{\text{#1}} %% CMS symbol formatting
  \usepackage{Style/ptdr-definitions} %% From CMS note util
\else
	\usepackage[dvipdfmx]{graphicx}
\fi
%%%%%%%%%%%%%%%%%%%%%%%%%%%%%%%%%%%%%%%%

\usepackage[backend=biber,sorting=none]{biblatex}
\addbibresource{Main.bib}
\usepackage[]{hyperref}
\hypersetup{}
\usepackage{lipsum} % lorem ipsum

%% -- Definitions for Charged Higgs Analysis
\newcommand{\mt}{\ensuremath{m_{\text{t}}}\xspace}
\newcommand{\PGtp}{\ensuremath{\tau^+}\xspace}
\newcommand{\PGtm}{\ensuremath{\tau^-}\xspace}
\newcommand{\PWpm}{\ensuremath{\text{W}^\pm}\xspace}
\newcommand{\PHp}{\ensuremath{\text{H}^+}\xspace}
\newcommand{\PHm}{\ensuremath{\text{H}^-}\xspace}
\newcommand{\pT}{\ensuremath{p_{\mathrm{T}}}\xspace}
\newcommand{\mZ}{\ensuremath{m_{\text{Z}}}\xspace}
\newcommand{\mW}{\ensuremath{m_{\text{W}}}\xspace}
\newcommand{\mHc}{\ensuremath{m_{\PHc}}\xspace}
\newcommand{\mA}{\ensuremath{m_{\PA}}\xspace}
\newcommand{\mH}{\ensuremath{m_{\PH}}\xspace}
\newcommand{\mh}{\ensuremath{m_{\Ph}}\xspace}
\newcommand{\PHc}{\ensuremath{\PH^{\pm}}\xspace}
\renewcommand{\PHpm}{\ensuremath{\PH^{\pm}}\xspace}
\newcommand{\PA}{\ensuremath{\text{A}}\xspace}
\renewcommand{\Ph}{\ensuremath{\text{h}}\xspace}
\renewcommand{\PH}{\ensuremath{\text{H}}\xspace}
\renewcommand{\fbinv}{\ensuremath{\text{fb}^{-1}}\xspace}
\newcommand{\SU}{\ensuremath{\mathrm{SU}}\xspace}
\newcommand{\U}{\ensuremath{\mathrm{U}}\xspace}

%% \title : 22pt로 나오는 큰 제목
%% \title* : 16pt로 나오는 작은 제목
\title{Search for light charged Higgs boson decaying into a W boson and a CP-odd Higgs boson in rare top quark decays in proton-proton collisions at $\sqrt{s} = 13$ and $13.6~\TeV$ at the CMS experiment}
\title*{CMS 실험에서 $\sqrt{s} = 13$ 및 $13.6~\TeV$ 양성자-양성자 충돌의 희귀 탑 쿼크 붕괴를 통해 W 보존과 CP-홀수 힉스 보존으로 붕괴하는 가벼운 하전 힉스 보존 탐색}

\academicko{이학}
\schoolen{COLLEGE OF NATURAL SCIENCE}
\departmenten{DEPARTMENT OF PHYSICS AND ASTRONOMY}
\departmentko{물리천문학부}

%% 저자 이름 Author's(Your) name
\author{최진}
\author*{최~진} % Same as \author.

%% 학번 Student number
\studentnumber{2019-20508}

%% 지도교수님 성함 Advisor's name
\advisor{양운기}
\advisor*{양~운~기}

%% 학위 수여일 Graduation date
%% 표지에 적히는 날짜.
%% 학위 수여일이 아니라 논문 발간년도를 적어야 할 수도 있음.
\graddate{June 2026}

%% 논문 제출일 Submission date
%% (?) Use Korean date format.
\submissiondate{2026~년~6~월}

%% 논문 인준일 Approval date
%% (?) Use Korean date format.
\approvaldate{2026~년~6~월}

%% Note: 인준지의 교수님 성함은
%% 컴퓨터로 출력하지 않고, 교수님께서
%% 자필로 쓰시기도 합니다.
%% Committee members' names
\committeemembers%
{최선호}%
{양운기}%
{유종희}%
{김형도}%
{홍길동}
%% Length of underline
%\setlength{\committeenameunderlinelength}{7cm}

\begin{document}
\pagenumbering{Roman}
\makefrontcover
\makefrontcover
\makeapproval

\cleardoublepage
\pagenumbering{roman}
% 초록 Abstract

\keyword{Charged Higgs boson, Two-Higgs-Doublet Model, CMS, LHC, Beyond Standard Model}

\begin{abstract}
This thesis presents a search for light charged Higgs bosons in rare top quark decays using proton-proton collision data collected by the CMS detector at the LHC. The data correspond to an integrated luminosity of 138~\fbinv at 13~\TeV and 62~\fbinv at 13.6~\TeV. The search targets the decay chain $\PQt \to \PHc\PQb$, followed by $\PHc \to \PWp\PA$ and $\PA \to \Pgmp\Pgmm$, where \PHc is the charged Higgs boson and \PA is a CP-odd Higgs boson predicted by Two-Higgs-Doublet Models (2HDM). The final state consists of three charged leptons ($\Pe\Pgm\Pgm$ or $\Pgm\Pgm\Pgm$), missing transverse momentum, and at least two jets including one b-tagged jet. The search covers charged Higgs boson masses($\mHc$) from 70 to 160\GeV and CP-odd Higgs boson masses($\mA$) from 15\GeV to $\mHc - 5\GeV$, including off-shell $\PHc \to \PWp\PA$ decays. A ParticleNet-based classifier is used to discriminate signal from \PZ boson backgrounds, and signal is extracted using a binned maximum likelihood fit to the dimuon invariant mass distribution. No statistically significant excess over the Standard Model prediction is observed, and upper limits at 95\% confidence level are set on the signal branching ratio. This result represents the first search for the decay $\PHc \to \PW\PA$ including off-shell effects.
\end{abstract}


\tableofcontents
\listoffigures
\listoftables

\cleardoublepage
\pagenumbering{arabic}

%% Main chapters
\chapter{Introduction}
\label{ch:introduction}

The Standard Model (SM) of particle physics represents one of the most successful scientific theories ever developed, providing an elegant mathematical framework that describes the fundamental constituents of matter and their interactions. Since its formulation in the 1960s and 1970s, the SM has withstood rigorous experimental scrutiny, culminating in the historic discovery of the Higgs boson by the ATLAS and CMS collaborations at the Large Hadron Collider (LHC) in 2012~\cite{HiggsDiscoveryATLAS,HiggsDiscoveryCMS}. This discovery confirmed the Brout-Englert-Higgs mechanism as the source of electroweak symmetry breaking and provided a compelling explanation for the origin of mass for fundamental particles.

Despite its remarkable success, the SM is widely regarded as an incomplete description of nature. Several observed phenomena remain unexplained within its framework: the existence and nature of dark matter, the matter-antimatter asymmetry in the universe, the hierarchy problem associated with the Higgs boson mass, and the origin of neutrino masses. These unresolved questions strongly suggest the existence of physics beyond the Standard Model (BSM), motivating extensive theoretical and experimental efforts to uncover new fundamental principles.

Among the various BSM extensions, models with extended Higgs sectors have attracted considerable attention. The Two-Higgs-Doublet Model (2HDM) represents one of the simplest and most well-motivated extensions, introducing a second Higgs doublet to the SM scalar sector~\cite{Branco:2011iw}. This extension is particularly compelling because it arises naturally in several theoretical frameworks, including supersymmetric theories~\cite{Gunion:1989we}, models addressing the hierarchy problem~\cite{Martin:1997ns}, and the Peccei-Quinn mechanism proposed to solve the strong CP problem~\cite{Peccei:1977hh,Peccei:1977ur,Peccei:2006as}.

The 2HDM predicts a rich phenomenology with five physical Higgs bosons: two CP-even neutral scalars (\Ph and \PH), one CP-odd neutral pseudoscalar (\PA), and a pair of charged Higgs bosons (\PHpm). The discovery of any of these additional Higgs bosons would constitute unambiguous evidence for new physics and provide crucial insights into the structure of electroweak symmetry breaking beyond the SM.

\section{Motivation for Charged Higgs Searches}

The charged Higgs boson (\PHpm) is a particularly distinctive signature of extended Higgs sectors, as its existence would definitively establish physics beyond the SM. Unlike neutral scalars, which could potentially be confused with the SM Higgs boson, charged Higgs bosons have no SM counterpart, making their detection a clear discovery channel for new physics.

Searches for charged Higgs bosons at hadron colliders are typically categorized based on the relationship between the charged Higgs mass (\mHc) and the top quark mass (\mt). For $\mHc < \mt$, the dominant production mechanism is through top quark decays: $\PQt \to \PHc\PQb$. In this scenario, charged Higgs bosons are produced in abundance from $\ttbar$ pair production, one of the most copious processes at the LHC. For $\mHc > \mt$, production occurs primarily through gluon fusion or in association with a top quark.

Previous searches have predominantly focused on fermionic decay modes of the charged Higgs boson, such as $\PHpm \to \PQt\PAQb$, $\PHpm \to \PGtp\PGn$, and $\PHpm \to \PQc\PAQs$. However, bosonic decay modes, particularly $\PHpm \to \PWpm\PA$, can become dominant in specific regions of the 2HDM parameter space. In a Type-I 2HDM with large values of \tanb (the ratio of vacuum expectation values of the two Higgs doublets), the decay $\PHpm \to \PWpm\PA$ can achieve branching fractions approaching unity, as the fermionic decay modes become suppressed.

\section{The Search Channel: $\PHpm \to \PWpm\PA \to \PWpm\PGmp\PGmm$}

This thesis presents a search for light charged Higgs bosons produced in top quark decays, with subsequent decay through the chain $\PHpm \to \PWpm\PA$ followed by $\PA \to \PGmp\PGmm$. The complete signal process can be written as:
\begin{equation}
\Pp\Pp \to \ttbar \to (\PHpm\PQb)(\PWmp\PQb) \to (\PWpm\PA\PQb)(\PWmp\PQb) \to (\PWpm\PGmp\PGmm\PQb)(\PWmp\PQb)
\end{equation}
where one top quark decays via the charged Higgs while the other undergoes the SM decay $\PQt \to \PWp\PQb$.

The choice of the dimuon decay channel for the pseudoscalar \PA offers several experimental advantages. The Compact Muon Solenoid (CMS) detector, as its name suggests, provides exceptional muon reconstruction capabilities with excellent momentum resolution. This enables precise reconstruction of the dimuon invariant mass, which is crucial for identifying narrow resonance structures over the continuum background. The analysis can therefore scan fine mass bins across a wide range of \mA values, maximizing sensitivity to potential signal contributions.

The final state topology consists of three charged leptons (from the two \PW bosons and the dimuon pair from \PA decay), missing transverse momentum from neutrinos, and at least two jets from bottom quark hadronization. Events are categorized into two channels based on the lepton flavors: the $\Pe\Pgm\Pgm$ channel (one electron, two muons) and the $\Pgm\Pgm\Pgm$ channel (three muons).

\section{Extension to Off-Shell Decays}

A significant advancement of this analysis compared to previous searches is the inclusion of off-shell $\PHpm \to \PWpm\PA$ decays. Previous CMS and ATLAS searches~\cite{CMSRun2ChargedHiggs,ATLASRun2ChargedHiggs} restricted their scope to on-shell decays, requiring $\mA < \mHc - m_{\PW}$. This constraint excluded a substantial portion of the 2HDM parameter space where the decay proceeds through off-shell \PW bosons.

Calculations using the 2HDMC program~\cite{2HDMC} demonstrate that the targeted decay chain maintains significant branching ratios even in the off-shell region, extending the physics reach of the search. The inclusion of off-shell decays, however, introduces additional challenges, particularly from backgrounds involving \PZ boson decays to muon pairs. To address this, the analysis employs advanced machine learning techniques based on graph neural networks for effective background discrimination.

\section{Analysis Strategy Overview}

The search employs a combination of data-driven methods and Monte Carlo (MC) simulation to estimate the various background contributions. The dominant backgrounds include:
\begin{itemize}
\item Nonprompt leptons from jets misidentified as leptons or from heavy-flavor hadron decays
\item Photon conversions, both external (in detector material) and internal (Dalitz decays)
\item Irreducible prompt backgrounds from diboson (\PW\PZ, \PZ\PZ) and $\ttbar$+V production
\end{itemize}

The nonprompt lepton background is estimated using the matrix method (fake rate method), which extrapolates from control regions enriched in misidentified leptons to the signal region. Conversion backgrounds are constrained using scale factors derived from $\PZ\Pgg$ control regions. Prompt backgrounds are estimated from MC simulation with appropriate theoretical and experimental systematic uncertainties.

Signal extraction is performed using a binned maximum likelihood fit to the dimuon invariant mass distribution. The analysis scans over a two-dimensional grid of charged Higgs masses (70--160\GeV) and pseudoscalar masses (15\GeV to $\mHc-5\GeV$), providing comprehensive coverage of the accessible parameter space.

Machine learning techniques play a central role in enhancing the sensitivity of the search. A ParticleNet-based~\cite{ParticleNet} classifier is trained to discriminate signal events from the dominant \PZ boson backgrounds, exploiting the distinct kinematic and topological features of the signal process. This approach yields significant improvements in sensitivity, particularly in the challenging off-shell decay regions.

\section{Dataset and Experimental Context}

The analysis utilizes proton-proton collision data collected by the CMS experiment at the LHC during Run~2 (2016--2018) at a center-of-mass energy of $\sqrt{s} = 13\TeV$, corresponding to an integrated luminosity of 138\fbinv. The unprecedented size of this dataset, combined with the high $\ttbar$ production cross section at the LHC, provides substantial statistical power for the search.

The CMS detector, with its precise tracking system, high-resolution electromagnetic calorimeter, and powerful muon detection capabilities, is ideally suited for this analysis. The combination of excellent lepton identification, efficient b-jet tagging, and accurate missing transverse momentum reconstruction enables effective selection of the signal topology while suppressing backgrounds.

\section{Thesis Structure}

This thesis is organized as follows. Chapter~\ref{ch:theory} provides the theoretical foundation, reviewing the Standard Model, the Two-Higgs-Doublet Model, and the phenomenology of charged Higgs bosons relevant to this search. Chapter~\ref{ch:detector} describes the LHC accelerator complex and the CMS detector, with emphasis on the subsystems most relevant for this analysis.

Chapter~\ref{ch:datasets} details the datasets and Monte Carlo samples used, including the signal modeling approach. Chapter~\ref{ch:objects} discusses the reconstruction and identification of physics objects: electrons, muons, jets, b-tagged jets, and missing transverse momentum. The event selection criteria and signal extraction strategy are presented in Chapter~\ref{ch:selection}.

Chapter~\ref{ch:background} describes the methods used for background estimation, including the matrix method for nonprompt leptons and control region techniques for conversion backgrounds. Chapter~\ref{ch:systematics} summarizes the systematic uncertainties considered in the analysis.

The results of the search, including observed and expected limits on the signal cross section, are presented in Chapter~\ref{ch:results}. Finally, Chapter~\ref{ch:conclusion} provides conclusions and discusses prospects for future searches.

\chapter{Theoretical Background}
\label{ch:theory}

This chapter provides the theoretical foundation for the search presented in this thesis. We begin with an overview of the Standard Model of particle physics, followed by a discussion of its limitations and the motivation for extended Higgs sectors. The Two-Higgs-Doublet Model is then introduced, with particular emphasis on the phenomenology of charged Higgs bosons relevant to this analysis.

\section{The Standard Model of Particle Physics}
\label{sec:theory_sm}

The Standard Model (SM) is a quantum field theory that describes the fundamental particles and their interactions through three of the four known fundamental forces: the electromagnetic, weak, and strong interactions. Gravity, while well-described by general relativity at macroscopic scales, is not incorporated into the SM framework.

\subsection{Spacetime Symmetries and Particle Classification}
\label{sec:spacetime_symmetries}

The construction of the Standard Model begins not with a list of particles, but with a principle: the laws of physics must take the same form regardless of where or when an experiment is performed, or which inertial frame the observer occupies. As Weinberg emphasized, the result of an experiment on Earth must be predicted by the same laws that would apply on the Moon---not because of any dynamical mechanism, but because the laws themselves respect the symmetries of spacetime~\cite{Weinberg:1995mt}.

This requirement of invariance under spacetime transformations is encoded in the Poincar\'{e} group---the group of all isometries of Minkowski spacetime, comprising translations, rotations, and Lorentz boosts. By Noether's theorem, each continuous symmetry implies a conserved quantity: translational invariance in space and time yields conservation of momentum and energy, while rotational invariance yields conservation of angular momentum.

The most profound consequence of these symmetries for particle physics is Wigner's classification~\cite{Wigner:1939cj}: elementary particles correspond to irreducible unitary representations of the Poincar\'{e} group, labeled by two invariants---mass $m \geq 0$ and spin $s = 0, \frac{1}{2}, 1, \frac{3}{2}, \ldots$. This is not merely a convenient labeling scheme; it is the \emph{definition} of what constitutes an elementary particle.

The structure of the Lorentz group further constrains what types of fields can describe particles. The Lie algebra of the proper orthochronous Lorentz group $\mathrm{SO}^+(3,1)$ admits, upon complexification, a decomposition into two copies of the $\mathfrak{su}(2)$ algebra:
\begin{equation}
\mathfrak{so}(3,1)_{\mathbb{C}} \cong \mathfrak{su}(2) \oplus \mathfrak{su}(2)
\end{equation}
Finite-dimensional representations are therefore labeled by a pair $(j_L, j_R)$, where each index independently takes values $0, \frac{1}{2}, 1, \ldots$. The familiar fields of particle physics correspond to specific representations in this classification:
\begin{itemize}
\item $(0, 0)$: scalar field (e.g., the Higgs boson)
\item $(\frac{1}{2}, 0)$: left-handed Weyl spinor
\item $(0, \frac{1}{2})$: right-handed Weyl spinor
\item $(\frac{1}{2}, 0) \oplus (0, \frac{1}{2})$: Dirac spinor (massive fermions)
\item $(\frac{1}{2}, \frac{1}{2})$: four-vector field (gauge bosons)
\end{itemize}
The existence of two distinct spinor representations---left-handed and right-handed---is not imposed by hand but follows from the structure of spacetime itself. Their physical distinction through the weak interaction is discussed in Section~\ref{sec:gauge_structure}.

The union of quantum mechanics with special relativity carries a further consequence. Consistent quantization of fields on Minkowski spacetime---requiring Lorentz invariance, unitarity, and locality---leads to the CPT theorem: every particle must have a corresponding antiparticle with identical mass and spin but opposite internal quantum numbers. The spin-statistics theorem, arising from the same requirements, dictates that the half-integer-spin fermions described by spinor representations obey Fermi--Dirac statistics, while integer-spin bosons obey Bose--Einstein statistics. Matter and antimatter are thus not independent postulates but inevitable consequences of relativistic quantum field theory~\cite{Streater:1989vi}.

The experimentally observed matter content of the SM consists of twelve spin-$\frac{1}{2}$ fermions organized into three generations, each containing two quarks and two leptons, summarized in Table~\ref{tab:sm_fermions}.

\begin{table}[htbp]
\centering
\caption{The three generations of fermions in the Standard Model with their electric charges and approximate masses~\cite{PDG2024}.}
\label{tab:sm_fermions}
\begin{tabular}{lllcl}
\hline\hline
Generation & Particle & Type & Charge & Mass \\
\hline
First & up (\PQu) & quark & $+2/3$ & $\sim$2.2 MeV \\
      & down (\PQd) & quark & $-1/3$ & $\sim$4.7 MeV \\
      & electron (\Pe) & lepton & $-1$ & 0.511 MeV \\
      & $\nu_e$ & neutrino & $0$ & $<$1 eV \\
\hline
Second & charm (\PQc) & quark & $+2/3$ & $\sim$1.27 GeV \\
       & strange (\PQs) & quark & $-1/3$ & $\sim$93 MeV \\
       & muon (\Pgm) & lepton & $-1$ & 105.7 MeV \\
       & $\nu_\mu$ & neutrino & $0$ & $<$1 eV \\
\hline
Third & top (\PQt) & quark & $+2/3$ & $\sim$172.8 GeV \\
      & bottom (\PQb) & quark & $-1/3$ & $\sim$4.18 GeV \\
      & tau (\Pgt) & lepton & $-1$ & 1.777 GeV \\
      & $\nu_\tau$ & neutrino & $0$ & $<$1 eV \\
\hline\hline
\end{tabular}
\end{table}

Quarks carry color charge and participate in all three SM interactions, while leptons do not carry color charge and are thus absent from the strong interaction. This generational structure was not predicted from first principles but emerged from successive experimental discoveries---from the muon (1936) and strange quark physics, through the electroweak unification by Glashow, Weinberg, and Salam~\cite{Glashow:1961tr,Weinberg:1967tq,Salam:1968rm}, to the third-generation particles discovered between 1975 and 1995.

\subsection{Gauge Structure and Interactions}
\label{sec:gauge_structure}

The SM is based on the gauge symmetry group:
\begin{equation}
G_{\text{SM}} = \SU(3)_C \times \SU(2)_L \times \U(1)_Y
\end{equation}
where $\SU(3)_C$ describes the strong interaction (quantum chromodynamics, QCD), and $\SU(2)_L \times \U(1)_Y$ describes the electroweak interaction before symmetry breaking.

\subsubsection{The Gauge Principle}

The SM interactions follow from a single organizing principle: the Lagrangian must be invariant under \emph{local} gauge transformations---independent phase rotations at each spacetime point. The ordinary derivative $\partial_\mu$ is not covariant under such transformations, necessitating a connection---the covariant derivative $D_\mu = \partial_\mu - igA_\mu^a T^a$---that parallel-transports symmetry charges between neighboring points. The connection fields $A_\mu^a$ are precisely the gauge bosons, one for each generator of the symmetry group: eight gluons for $\SU(3)_C$, three weak bosons ($W^1$, $W^2$, $W^3$) for $\SU(2)_L$, and one hypercharge boson ($B$) for $\U(1)_Y$. Force-carrying particles thus arise not as ad hoc additions but as geometric consequences of local symmetry.

\subsubsection{Parity Violation and the Chiral Structure of Weak Interactions}

As discussed in Section~\ref{sec:spacetime_symmetries}, the Lorentz group admits two distinct spinor representations---left-handed $(\frac{1}{2}, 0)$ and right-handed $(0, \frac{1}{2})$. The electromagnetic and strong interactions treat these identically, preserving parity. It was therefore a profound surprise when Wu's 1957 experiment on cobalt-60 beta decay~\cite{Wu:1957my} demonstrated that the weak interaction maximally violates parity, coupling exclusively to left-handed fermions and right-handed antifermions.

This observation was codified in the $V{-}A$ (vector minus axial-vector) theory of Feynman and Gell-Mann~\cite{Feynman:1958ty} and Sudarshan and Marshak~\cite{Sudarshan:1958vf}, which describes the charged weak current as:
\begin{equation}
J^\mu_{\text{CC}} \propto \bar{\psi}\gamma^\mu(1 - \gamma^5)\psi
\end{equation}
The projection operator $(1 - \gamma^5)/2$ selects only the left-handed component of the fermion field, implementing maximal parity violation. This empirical structure finds its natural explanation in the gauge framework: left-handed fermions transform as $\SU(2)_L$ doublets, while right-handed fermions are $\SU(2)_L$ singlets. The subscript ``$L$'' in $\SU(2)_L$ directly encodes this experimental fact.

Left-handed fermions are thus organized into $\SU(2)_L$ doublets:
\begin{equation}
\ell_L = \begin{pmatrix} \nu_e \\ e^- \end{pmatrix}_L, \quad
q_L = \begin{pmatrix} u \\ d \end{pmatrix}_L
\end{equation}
while right-handed fermions are $\SU(2)_L$ singlets: $e_R$, $u_R$, $d_R$ (and similarly for the other generations). In the minimal SM, right-handed neutrinos $\nu_R$ are not included. The $\SU(2)_L$ generators act non-trivially only on left-handed doublets, so that weak charged currents couple exclusively to left-handed fermions (and right-handed antifermions).

\subsubsection{Quantum Number Assignments}

Each fermion chirality state carries conserved charges associated with the gauge symmetries. The weak isospin $I_3$ is the eigenvalue of the $\SU(2)_L$ generator $T^3$, and the weak hypercharge $Y$ is the $\U(1)_Y$ charge. After electroweak symmetry breaking, the unbroken $\U(1)_{\text{em}}$ subgroup defines the electric charge through the Gell-Mann--Nishijima relation:
\begin{equation}
Q = I_3 + Y
\label{eq:gellmann_nishijima}
\end{equation}

Table~\ref{tab:fermion_quantum_numbers} summarizes the quantum number assignments for the first generation of fermions. The pattern repeats for the second and third generations.

\begin{table}[htbp]
\centering
\caption{Quantum number assignments for the first generation of fermions under $\SU(2)_L \times \U(1)_Y$. The electric charge $Q$ is related to weak isospin $I_3$ and hypercharge $Y$ through $Q = I_3 + Y$. The pattern repeats for the second and third generations.}
\label{tab:fermion_quantum_numbers}
\begin{tabular}{lccccc}
\hline\hline
Particle & Chirality & $I_3$ & $Y$ & $Q$ & Representation \\
\hline
$\nu_e$ & Left & $+1/2$ & $-1/2$ & $0$ & \multirow{2}{*}{$\SU(2)_L$ doublet} \\
$e^-$ & Left & $-1/2$ & $-1/2$ & $-1$ & \\
$u$ & Left & $+1/2$ & $+1/6$ & $+2/3$ & \multirow{2}{*}{$\SU(2)_L$ doublet} \\
$d$ & Left & $-1/2$ & $+1/6$ & $-1/3$ & \\
\hline
$\nu_e$ & Right & $0$ & $0$ & $0$ & $\SU(2)_L$ singlet \\
$e^-$ & Right & $0$ & $-1$ & $-1$ & $\SU(2)_L$ singlet \\
$u$ & Right & $0$ & $+2/3$ & $+2/3$ & $\SU(2)_L$ singlet \\
$d$ & Right & $0$ & $-1/3$ & $-1/3$ & $\SU(2)_L$ singlet \\
\hline\hline
\end{tabular}
\end{table}

Note that left-handed fermions have $I_3 = \pm 1/2$ (doublet members), while right-handed fermions have $I_3 = 0$ (singlets). The hypercharge $Y$ differs from the electric charge $Q$; for example, the left-handed electron has $Y = -1/2$ but $Q = -1$.

\subsubsection{Gauge Bosons and Couplings}

The gauge principle predicts massless gauge bosons in the gauge eigenstate basis: three fields $A^1_\mu$, $A^2_\mu$, $A^3_\mu$ for $\SU(2)_L$ and one field $B_\mu$ for $\U(1)_Y$. However, the observed physical spectrum---the massive $\PW^\pm$ and $\PZ$ bosons alongside the massless photon $\Pgg$---requires a mechanism to generate masses while preserving gauge invariance. This mechanism, electroweak symmetry breaking, is the subject of Section~\ref{sec:ewsb}.

The physical mass eigenstates arise from mixing between the gauge eigenstates. The charged $\PWpm$ bosons are combinations of $A^1_\mu$ and $A^2_\mu$, while the neutral $\PZ$ boson and photon $\Pgg$ are rotations of $A^3_\mu$ and $B_\mu$ through the weak mixing angle $\theta_W$:
\begin{equation}
\tan\theta_W = \frac{g'}{g}
\end{equation}
where $g$ and $g'$ are the $\SU(2)_L$ and $\U(1)_Y$ coupling constants, with the experimentally measured value $\swsq \approx 0.23$~\cite{PDG2024}. The elementary electric charge is:
\begin{equation}
e = g \sin\theta_W = g' \cos\theta_W
\label{eq:electric_charge}
\end{equation}
The photon couples universally to electric charge $Q$, while the $\PZ$ boson couples to the combination:
\begin{equation}
Q_Z = I_3 - \swsq Q
\label{eq:z_coupling}
\end{equation}
which differs between fermion species depending on their quantum numbers as shown in Table~\ref{tab:fermion_quantum_numbers}.

\subsection{Electroweak Symmetry Breaking and the Higgs Mechanism}
\label{sec:ewsb}

Direct mass terms such as $\frac{1}{2}m^2 W_\mu W^\mu$ explicitly break gauge invariance, destroying the cancellations that ensure renormalizability. The Brout--Englert--Higgs mechanism~\cite{Higgs:1964pj,Englert:1964et,Guralnik:1964eu} resolves this through spontaneous symmetry breaking: the gauge symmetry is preserved in the Lagrangian but broken by the vacuum state, generating gauge boson masses dynamically while maintaining renormalizability.

\subsubsection{The Higgs Doublet and Spontaneous Symmetry Breaking}

The SM introduces a complex scalar $\SU(2)_L$ doublet with hypercharge $Y = +1/2$:
\begin{equation}
\Phi = \begin{pmatrix} \phi^+ \\ \phi^0 \end{pmatrix}
\end{equation}
with four real degrees of freedom. The most general renormalizable scalar potential is:
\begin{equation}
V(\Phi) = -\mu^2 \Phi^\dagger\Phi + \lambda(\Phi^\dagger\Phi)^2
\label{eq:higgs_potential}
\end{equation}
For $\mu^2 > 0$ and $\lambda > 0$, the minimum is not at $\Phi = 0$ but on a circle of degenerate vacua with $|\Phi|^2 = \mu^2/(2\lambda)$. The system selects one vacuum, spontaneously breaking $\SU(2)_L \times \U(1)_Y$ to $\U(1)_{\text{em}}$:
\begin{equation}
\langle\Phi\rangle = \frac{1}{\sqrt{2}}\begin{pmatrix} 0 \\ v \end{pmatrix}, \quad v = \frac{\mu}{\sqrt{\lambda}} \approx 246\GeV
\label{eq:higgs_vev}
\end{equation}

Of the four generators of $\SU(2)_L \times \U(1)_Y$, three are broken by this vacuum. By Goldstone's theorem, each broken generator produces a massless scalar---three Goldstone bosons in total. In a gauge theory, these are not physical particles: they are absorbed (``eaten'') by the $\PW^\pm$ and $\PZ$ bosons, providing the longitudinal polarization states required by massive vector bosons. In the unitary gauge, the Higgs field reduces to:
\begin{equation}
\Phi(x) = \frac{1}{\sqrt{2}}\begin{pmatrix} 0 \\ v + h(x) \end{pmatrix}
\label{eq:unitary_gauge}
\end{equation}
where $h(x)$ is the physical Higgs boson with mass $m_h = \sqrt{2\lambda}\,v \approx 125\GeV$~\cite{ATLAS:2012yda,CMS:2012qbp}. The four degrees of freedom of the original doublet thus rearrange into three longitudinal gauge boson polarizations plus one physical scalar.

\subsubsection{Gauge Boson Mass Generation}

Gauge boson masses emerge from the kinetic term $|D_\mu\Phi|^2$ evaluated at the vacuum. The covariant derivative acting on the Higgs doublet is:
\begin{equation}
D_\mu\Phi = \left(\partial_\mu - ig\frac{\sigma^a}{2}A^a_\mu - ig'Y B_\mu\right)\Phi
\label{eq:covariant_derivative}
\end{equation}
Substituting $\langle\Phi\rangle = (0, v/\sqrt{2})^T$, the terms quadratic in gauge fields yield masses. The charged combinations $W^\pm_\mu = (A^1_\mu \mp i A^2_\mu)/\sqrt{2}$ acquire mass:
\begin{equation}
m_W = \frac{gv}{2}
\label{eq:w_mass}
\end{equation}
The neutral fields $A^3_\mu$ and $B_\mu$ mix through the weak mixing angle $\theta_W$ to form the photon $A_\mu$ and $Z$ boson:
\begin{equation}
\begin{pmatrix} A_\mu \\ Z_\mu \end{pmatrix} =
\begin{pmatrix} \cos\theta_W & \sin\theta_W \\ -\sin\theta_W & \cos\theta_W \end{pmatrix}
\begin{pmatrix} B_\mu \\ A^3_\mu \end{pmatrix}
\label{eq:neutral_mixing}
\end{equation}
The photon remains massless, coupling to the unbroken $\U(1)_{\text{em}}$ generator, while the $\PZ$ boson acquires mass:
\begin{equation}
m_Z = \frac{v\sqrt{g^2 + g'^2}}{2} = \frac{m_W}{\cos\theta_W}
\label{eq:z_mass}
\end{equation}
The relation $m_Z = m_W/\cos\theta_W$ is a prediction of the symmetry breaking mechanism, confirmed experimentally to sub-percent accuracy~\cite{PDG2024}.

\subsubsection{Physical Interactions and Fermion Masses}

In the mass eigenstate basis, the electroweak covariant derivative takes the form:
\begin{equation}
D_\mu = \partial_\mu - ieA_\mu Q - i\frac{g}{\cos\theta_W}Z_\mu Q_Z - ig\left(W^+_\mu T^+ + W^-_\mu T^-\right)
\label{eq:covariant_mass_basis}
\end{equation}
where $T^\pm = (T^1 \pm iT^2)/\sqrt{2}$ are the weak isospin raising and lowering operators. The photon couples to electric charge $Q$, the $\PZ$ boson to $Q_Z = I_3 - \swsq Q$ (see Table~\ref{tab:fermion_quantum_numbers}), and the $\PWpm$ bosons mediate transitions between doublet partners ($u \leftrightarrow d$, $\nu_e \leftrightarrow e^-$).

Fermion masses cannot arise from direct mass terms $m\bar{\psi}\psi$, which would connect left-handed doublets to right-handed singlets and thus violate gauge invariance. Instead, they are generated through Yukawa couplings to the Higgs doublet:
\begin{equation}
\mathcal{L}_{\text{Yukawa}} = -y_f \bar{\psi}_L \Phi \psi_R + \text{h.c.}
\end{equation}
After symmetry breaking, these yield fermion masses $m_f = y_f v / \sqrt{2}$. Unlike gauge boson masses, which are predicted by $v$, $g$, and $g'$, the Yukawa couplings $y_f$ are free parameters spanning six orders of magnitude---from $y_e \sim 3 \times 10^{-6}$ to $y_t \sim 1$. This unexplained ``flavor hierarchy'' motivates many extensions of the SM.

\section{Motivations for an Extended Scalar Sector}
\label{sec:theory_bsm}

The Higgs sector of the Standard Model is minimal: a single $\SU(2)_L$ doublet suffices to break electroweak symmetry and generate all observed masses. However, nothing in the gauge structure forbids additional scalar fields, and extending the scalar sector is among the most conservative modifications to the SM. Remarkably, such extensions can address several of the SM's outstanding problems simultaneously.

Cosmological observations establish that approximately 27\% of the universe's energy density consists of dark matter~\cite{Planck:2018vyg}, for which the SM provides no candidate. An additional scalar doublet stabilized by a discrete $\mathbb{Z}_2$ symmetry can serve as a weakly interacting dark matter candidate. The observed matter--antimatter asymmetry of the universe requires CP violation beyond what the CKM matrix provides~\cite{Sakharov:1967dj}; additional Higgs doublets introduce new complex couplings in the scalar potential that can drive electroweak baryogenesis. Neutrino oscillation experiments have established non-zero neutrino masses~\cite{PhysRevLett.81.1158,SNO:2002tuh}, which the minimal SM cannot accommodate; extended scalar sectors offer mass generation mechanisms such as the type-II seesaw with an $\SU(2)_L$ triplet or radiative generation through an inert doublet.

Beyond these experimental anomalies, theoretical considerations also point toward extended scalar sectors. The Higgs boson mass receives quadratically divergent quantum corrections that would naturally drive it to the Planck scale---the hierarchy problem. Supersymmetry, which addresses this through partner-particle loop cancellations, requires at least two Higgs doublets to give masses to both up-type and down-type fermions. Similarly, the Peccei--Quinn solution to the strong CP problem---explaining why the QCD vacuum angle satisfies $|\bar{\theta}| < 10^{-10}$~\cite{Abel:2020pzs}---introduces an additional scalar field and predicts the axion~\cite{Peccei:1977hh}.

The simplest such extension that preserves the SM gauge structure is the Two-Higgs-Doublet Model, discussed in the following section.

\section{The Two-Higgs-Doublet Model}
\label{sec:theory_2hdm}

\subsection{General Review}

\subsubsection{Scalar Potential and Symmetry Breaking}

The 2HDM introduces two complex $\SU(2)_L$ doublets $\Phi_1$ and $\Phi_2$, each with hypercharge $Y = +1/2$:
\begin{equation}
\Phi_i = \begin{pmatrix} \phi_i^+ \\ \phi_i^0 \end{pmatrix}, \quad i = 1,2
\end{equation}

The most general renormalizable scalar potential respecting the gauge symmetry is:
\begin{align}
V(\Phi_1, \Phi_2) &= m_{11}^2 \Phi_1^\dagger\Phi_1 + m_{22}^2 \Phi_2^\dagger\Phi_2 - m_{12}^2(\Phi_1^\dagger\Phi_2 + \text{h.c.}) \nonumber\\
&+ \frac{\lambda_1}{2}(\Phi_1^\dagger\Phi_1)^2 + \frac{\lambda_2}{2}(\Phi_2^\dagger\Phi_2)^2 + \lambda_3(\Phi_1^\dagger\Phi_1)(\Phi_2^\dagger\Phi_2) \nonumber\\
&+ \lambda_4(\Phi_1^\dagger\Phi_2)(\Phi_2^\dagger\Phi_1) + \frac{\lambda_5}{2}\left[(\Phi_1^\dagger\Phi_2)^2 + \text{h.c.}\right]
\end{align}
where we have imposed a softly broken $\mathbb{Z}_2$ symmetry ($\Phi_1 \to \Phi_1$, $\Phi_2 \to -\Phi_2$) to avoid tree-level flavor-changing neutral currents (FCNCs).

After electroweak symmetry breaking, both doublets can acquire vevs:
\begin{equation}
\langle\Phi_i\rangle = \frac{1}{\sqrt{2}}\begin{pmatrix} 0 \\ v_i \end{pmatrix}
\end{equation}
with $v^2 = v_1^2 + v_2^2 = (246\GeV)^2$. The ratio of vevs defines an important parameter:
\begin{equation}
\tan\beta \equiv \frac{v_2}{v_1}
\end{equation}

\subsubsection{Physical Higgs Bosons}

The eight degrees of freedom in the two doublets rearrange as follows after symmetry breaking:
\begin{itemize}
\item Three Goldstone bosons ($G^\pm$, $G^0$) become the longitudinal modes of \PWpm and \PZ
\item Five physical Higgs bosons remain: \Ph, \PH (CP-even), \PA (CP-odd), and \PHpm (charged)
\end{itemize}

In the CP-conserving case, the neutral CP-even states mix through an angle $\alpha$:
\begin{equation}
\begin{pmatrix} \Ph \\ \PH \end{pmatrix} = \begin{pmatrix} \cos\alpha & \sin\alpha \\ -\sin\alpha & \cos\alpha \end{pmatrix} \begin{pmatrix} \rho_1 \\ \rho_2 \end{pmatrix}
\end{equation}
where $\rho_{1,2}$ are the neutral CP-even components of $\Phi_{1,2}$.

The masses of the physical Higgs bosons are related to the potential parameters:
\begin{align}
m_{\Ph,\PH}^2 &= \frac{1}{2}\left[(A_{11} + A_{22}) \mp \sqrt{(A_{11} - A_{22})^2 + 4A_{12}^2}\right] \\
m_\PA^2 &= m_{12}^2\left(\frac{1}{s_\beta c_\beta}\right) - \lambda_5 v^2 \\
m_{\PHpm}^2 &= m_\PA^2 + \frac{1}{2}(\lambda_5 - \lambda_4)v^2
\end{align}
where $s_\beta = \sin\beta$, $c_\beta = \cos\beta$, and $A_{ij}$ are elements of the CP-even mass matrix.

\subsubsection{Types of 2HDM}

To avoid FCNCs at tree level, discrete symmetries are imposed such that each type of fermion couples to only one Higgs doublet. This leads to four distinct types of 2HDM:

\begin{table}[htbp]
\centering
\caption{Yukawa coupling structure in the four types of 2HDM. The table shows which Higgs doublet couples to each type of fermion.}
\label{tab:2hdm_types}
\begin{tabular}{ccccc}
\hline\hline
Type & Up-type quarks & Down-type quarks & Charged leptons \\
\hline
Type-I & $\Phi_2$ & $\Phi_2$ & $\Phi_2$ \\
Type-II & $\Phi_2$ & $\Phi_1$ & $\Phi_1$ \\
Type-X (Lepton-specific) & $\Phi_2$ & $\Phi_2$ & $\Phi_1$ \\
Type-Y (Flipped) & $\Phi_2$ & $\Phi_1$ & $\Phi_2$ \\
\hline\hline
\end{tabular}
\end{table}

This analysis is primarily motivated by the Type-I 2HDM, where all fermions couple to $\Phi_2$. In this scenario, the couplings of the charged Higgs to fermions are proportional to $\cot\beta$, and for large \tanb, these couplings become suppressed.

\subsubsection{The Alignment Limit}

The discovered 125~GeV Higgs boson has properties consistent with the SM predictions within experimental uncertainties~\cite{ATLAS:2022vkf,CMS:2022dwd}. This constrains the 2HDM parameter space to be near the ``alignment limit,'' where one of the CP-even Higgs bosons has SM-like couplings.

The alignment limit is characterized by $\cos(\beta - \alpha) \to 0$ or equivalently $\sin(\beta - \alpha) \to 1$. In this limit, \Ph has SM-like couplings, while \PH couples to vector bosons proportionally to $\cos(\beta - \alpha)$.

\subsection{Charged Higgs Boson Phenomenology}
\label{sec:theory_charged}

\subsubsection{Production Mechanisms}

For $\mHc < \mt$, charged Higgs bosons are predominantly produced through top quark decays:
\begin{equation}
\PQt \to \PHc\PQb
\end{equation}
competing with the SM decay $\PQt \to \PWp\PQb$. The branching fraction depends on \mHc and \tanb:
\begin{equation}
\mathcal{B}(\PQt \to \PHc\PQb) \propto \left(\frac{m_t^2}{v^2}\cot^2\beta + \frac{m_b^2}{v^2}\tan^2\beta\right)\left(1 - \frac{m_{\PHc}^2}{m_t^2}\right)^2
\end{equation}

At low \tanb, the $m_t$ term dominates, while at high \tanb, the $m_b$ term can become important (particularly in Type-II models). In Type-I models, both terms are suppressed at high \tanb.

For $\mHc > \mt$, production occurs through:
\begin{itemize}
\item Associated production with top quark: $\Pg\Pg/\Pg\PQb \to \PHc\PQt\PQb$
\item Pair production: $\Pq\Paq \to \PHc\PHm$
\end{itemize}

\subsubsection{Decay Modes}

The charged Higgs can decay through both fermionic and bosonic channels:

\textbf{Fermionic decays:}
\begin{itemize}
\item $\PHpm \to \PQt\PAQb$ (kinematically forbidden for $\mHc < \mt$)
\item $\PHpm \to \PGtp\PGn$
\item $\PHpm \to \PQc\PAQs$, $\PHpm \to \PQc\PAQb$
\end{itemize}

\textbf{Bosonic decays:}
\begin{itemize}
\item $\PHpm \to \PWpm\Ph$
\item $\PHpm \to \PWpm\PA$
\item $\PHpm \to \PWpm\PH$
\end{itemize}

The partial widths for fermionic decays in a Type-I 2HDM are:
\begin{align}
\Gamma(\PHpm \to \PQt\PAQb) &\propto \frac{m_t^2 + m_b^2}{v^2}\cot^2\beta \\
\Gamma(\PHpm \to \PGtp\PGn) &\propto \frac{m_\tau^2}{v^2}\cot^2\beta
\end{align}

The bosonic decay $\PHpm \to \PWpm\PA$ has partial width:
\begin{equation}
\Gamma(\PHpm \to \PWpm\PA) = \frac{g^2}{64\pi}\frac{\cos^2(\beta-\alpha)}{m_\PW^2}\lambda^{3/2}(m_{\PHc}^2, m_\PW^2, m_\PA^2) \cdot m_{\PHc}
\end{equation}
where $\lambda(a,b,c) = (1 - b/a - c/a)^2 - 4bc/a^2$ is the Källén function.

In the alignment limit ($\cos(\beta-\alpha) \to 0$), this decay is suppressed. However, away from exact alignment, particularly for large \tanb in Type-I models where fermionic decays are suppressed, the bosonic decay $\PHpm \to \PWpm\PA$ can become dominant.

\subsubsection{The Target Decay Chain}

This analysis targets the decay chain:
\begin{equation}
\PHpm \to \PWpm\PA \to \PWpm\PGmp\PGmm
\end{equation}

The pseudoscalar \PA decays to fermion pairs with branching fractions proportional to the squared fermion masses. For $\mA > 2m_\Pgm$ and below the $\PQb\PAQb$ threshold, the dimuon channel provides a clean experimental signature:
\begin{equation}
\mathcal{B}(\PA \to \PGmp\PGmm) \approx \frac{m_\mu^2}{m_\tau^2 + m_c^2 + 3m_\mu^2} \approx 3\%
\end{equation}
for $2m_\mu < \mA < 2m_\tau$. For heavier \PA masses, the $\tau\tau$ and $\PQb\PAQb$ channels dominate.

\subsubsection{Off-Shell Decays}

An important feature of this analysis is the inclusion of off-shell $\PHpm \to \PWpm\PA$ decays. When $\mA > \mHc - m_\PW$, the \PW boson in the decay is virtual. The three-body decay width can be written as:
\begin{equation}
\Gamma(\PHpm \to \PA\ell\nu) = \int \frac{d\Gamma(\PHpm \to \text{W}^{\pm *}\PA)}{dm_{W^*}^2} \cdot \mathcal{B}(\text{W}^{\pm *} \to \ell\nu) \, dm_{W^*}^2
\end{equation}

While the off-shell decay rate is suppressed compared to on-shell decays, it remains significant in regions of parameter space where fermionic decays are suppressed. Calculations using 2HDMC~\cite{2HDMC} show that the product of branching fractions $\mathcal{B}(\PQt \to \PHc\PQb) \times \mathcal{B}(\PHc \to \PWpm\PA) \times \mathcal{B}(\PA \to \PGmp\PGmm)$ can reach values accessible at the LHC even in the off-shell region.

\subsection{Previous Searches and Current Constraints}
\label{sec:theory_constraints}

\subsubsection{LEP and Tevatron Results}

At LEP, the DELPHI and OPAL experiments searched for charged Higgs pair production $\Pe^+\Pe^- \to \PHc\PHm$ and set lower bounds $\mHc > 72$--80\GeV depending on the assumed decay modes~\cite{DELPHIChargedHiggs,OPALChargedHiggs}.

The CDF experiment at the Tevatron searched for $\PHpm \to \PWpm\PA \to \PWpm\PGtp\PGtm$ and set limits on the branching fraction~\cite{CDFChargedHiggs}.

\subsubsection{LHC Run 1 and Run 2 Results}

During LHC Run 1 and Run 2, both ATLAS and CMS performed extensive searches for charged Higgs bosons:
\begin{itemize}
\item $\PHpm \to \PGtp\PGn$: Strong constraints for $\mHc < \mt$~\cite{ATLAS:2018gfm,CMS:2019bfg}
\item $\PHpm \to \PQt\PAQb$: Constraints for $\mHc > \mt$~\cite{ATLAS:2021upq,CMS:2020imj}
\item $\PHpm \to \PQc\PAQs/\PQc\PAQb$: Constraints at low \tanb~\cite{ATLAS:2024oqu,CMS:2024iqz}
\item $\PHpm \to \PWpm\PA$: Searches by CMS (35.9\fbinv)~\cite{CMSRun2ChargedHiggs} and ATLAS (138\fbinv)~\cite{ATLASRun2ChargedHiggs}
\end{itemize}

The previous CMS search for $\PHpm \to \PWpm\PA \to \PWpm\PGmp\PGmm$ using 2016 data (35.9\fbinv) found no significant excess and set upper limits on the signal cross section. The ATLAS search using full Run 2 data similarly found no evidence for charged Higgs production in this channel.

\subsubsection{Theoretical Constraints}

The 2HDM parameter space is constrained by:
\begin{itemize}
\item \textbf{Perturbativity}: Requiring quartic couplings remain perturbative
\item \textbf{Unitarity}: S-matrix unitarity bounds on scalar scattering amplitudes
\item \textbf{Vacuum stability}: The potential must be bounded from below
\item \textbf{Electroweak precision observables}: The $\rho$ parameter constrains mass splittings
\item \textbf{Flavor physics}: $\PQb \to \PQs\Pgg$, $B$ meson mixing, and other processes
\end{itemize}

These constraints define the viable parameter space for the search, guiding the choice of mass ranges and benchmark scenarios considered in this analysis.

\chapter{The LHC and CMS Detector}
\label{ch:detector}

This chapter provides an overview of the experimental apparatus used to collect the data analyzed in this thesis. We begin with a description of the Large Hadron Collider (LHC), the world's largest and most powerful particle accelerator, followed by a detailed description of the Compact Muon Solenoid (CMS) detector.

\section{The Large Hadron Collider}
\label{sec:lhc}

The Large Hadron Collider (LHC)~\cite{Evans:2008zzb} is a circular hadron collider located at CERN (European Organization for Nuclear Research) near Geneva, Switzerland. It occupies a 26.7~km circumference tunnel, originally constructed for the Large Electron-Positron Collider (LEP), at depths ranging from 45 to 170 meters below the surface.

\subsection{Accelerator Complex}

The LHC is the final stage of a chain of accelerators that progressively increase the energy of proton beams:

\begin{enumerate}
\item \textbf{Linac4}: A linear accelerator that produces protons by ionizing hydrogen gas and accelerates them to 160~MeV.
\item \textbf{Proton Synchrotron Booster (PSB)}: Four superimposed synchrotron rings accelerate protons to 2~GeV.
\item \textbf{Proton Synchrotron (PS)}: Accelerates protons to 26~GeV, also defining the bunch structure.
\item \textbf{Super Proton Synchrotron (SPS)}: Accelerates protons to 450~GeV for injection into the LHC.
\item \textbf{LHC}: Accelerates two counter-rotating proton beams to energies up to 6.5~TeV (Run~2) or 6.8~TeV (Run~3).
\end{enumerate}

The LHC employs 1232 superconducting dipole magnets operating at 1.9~K to bend the proton beams around the circular path. The dipole magnets use NbTi superconducting cables and can produce magnetic fields up to 8.3~T. Additional superconducting quadrupole and higher-order magnets provide beam focusing and orbit correction.

\subsection{Beam Parameters}

During LHC Run~2 (2015--2018), proton-proton collisions were produced at a center-of-mass energy of $\sqrt{s} = 13\TeV$. Key beam parameters include:

\begin{table}[htbp]
\centering
\caption{LHC beam parameters during Run~2 operation.}
\label{tab:lhc_parameters}
\begin{tabular}{lc}
\hline\hline
Parameter & Value \\
\hline
Center-of-mass energy & 13\TeV \\
Beam energy & 6.5\TeV \\
Number of bunches per beam & up to 2556 \\
Protons per bunch & $\sim 1.15 \times 10^{11}$ \\
Bunch spacing & 25~ns \\
Peak instantaneous luminosity & $2.1 \times 10^{34}$~cm$^{-2}$s$^{-1}$ \\
\hline\hline
\end{tabular}
\end{table}

\subsection{Luminosity}

The instantaneous luminosity $\mathcal{L}$ characterizes the collision rate capability of a collider:
\begin{equation}
\mathcal{L} = \frac{N_1 N_2 n_b f_{\text{rev}}}{4\pi\sigma_x\sigma_y} F
\end{equation}
where $N_{1,2}$ are the number of particles per bunch in each beam, $n_b$ is the number of colliding bunches, $f_{\text{rev}}$ is the revolution frequency, $\sigma_{x,y}$ are the transverse beam sizes at the interaction point, and $F$ is a geometric factor accounting for the crossing angle.

The integrated luminosity, obtained by integrating the instantaneous luminosity over time, determines the total amount of collision data collected:
\begin{equation}
L = \int \mathcal{L}(t) \, dt
\end{equation}

During Run~2, the CMS experiment recorded a total integrated luminosity of 138\fbinv, distributed across three data-taking periods: 36.3\fbinv in 2016, 41.5\fbinv in 2017, and 59.7\fbinv in 2018.

\subsection{Pileup}

The high luminosity at the LHC results in multiple proton-proton interactions per bunch crossing, known as ``pileup.'' The average number of pileup interactions $\langle\mu\rangle$ is given by:
\begin{equation}
\langle\mu\rangle = \frac{\mathcal{L} \sigma_{\text{inel}}}{n_b f_{\text{rev}}}
\end{equation}
where $\sigma_{\text{inel}} \approx 80$~mb is the inelastic proton-proton cross section.

During Run~2, the average pileup ranged from approximately 20 to 50 interactions per bunch crossing. Pileup events contribute additional particles and energy deposits that must be accounted for in physics object reconstruction and calibration.

\section{The Compact Muon Solenoid Detector}
\label{sec:cms}

The Compact Muon Solenoid (CMS)~\cite{CMS:2008xjf} is a multipurpose particle detector designed to study a wide range of physics processes at the LHC. The detector is characterized by a powerful superconducting solenoid magnet, excellent tracking capabilities, and precise muon detection systems.

\subsection{Coordinate System}

The CMS coordinate system is defined with the origin at the nominal interaction point. The $x$-axis points toward the center of the LHC ring, the $y$-axis points vertically upward, and the $z$-axis points along the beam direction toward the Jura mountains. The azimuthal angle $\phi$ is measured from the $x$-axis in the $x$-$y$ plane, and the polar angle $\theta$ is measured from the $z$-axis.

A commonly used variable is the pseudorapidity:
\begin{equation}
\eta = -\ln\left[\tan\left(\frac{\theta}{2}\right)\right]
\end{equation}
For massless particles, pseudorapidity equals rapidity, and differences in pseudorapidity are invariant under Lorentz boosts along the beam axis.

The transverse momentum $\pT = p\sin\theta$ and transverse energy $\ET = E\sin\theta$ are defined in the plane perpendicular to the beam axis.

\subsection{Overall Structure}

The CMS detector has a cylindrical geometry with overall dimensions of 21.6~m in length, 14.6~m in diameter, and a total weight of approximately 14,000 tonnes. From inside to outside, the main subsystems are:

\begin{enumerate}
\item Silicon pixel and strip tracker
\item Electromagnetic calorimeter (ECAL)
\item Hadron calorimeter (HCAL)
\item Superconducting solenoid magnet
\item Muon detection systems
\end{enumerate}

The central ``barrel'' region covers $|\eta| < 1.5$, while ``endcap'' detectors extend coverage to $|\eta| < 2.5$ for tracking and $|\eta| < 5.0$ for calorimetry.

\subsection{Silicon Tracker}

The CMS silicon tracker~\cite{CMS:2014pgm} is the innermost detector subsystem, designed to precisely measure the trajectories of charged particles emerging from the collision point.

\subsubsection{Pixel Detector}

The pixel detector consists of silicon sensors with pixel sizes of $100 \times 150~\mu\text{m}^2$. During Run~2, the original 3-layer pixel detector was replaced by an upgraded 4-layer system (Phase-1 upgrade) in early 2017~\cite{CMS:2012sda}, providing improved tracking performance and radiation tolerance.

The upgraded pixel detector contains:
\begin{itemize}
\item 4 barrel layers (BPIX) at radii of 3.0, 6.8, 10.9, and 16.0~cm
\item 3 forward disks (FPIX) on each side at $|z| = 29.1$, 39.6, and 51.6~cm
\end{itemize}

\subsubsection{Strip Detector}

The silicon strip tracker surrounds the pixel detector and provides tracking coverage up to $|\eta| < 2.5$. It consists of:
\begin{itemize}
\item Tracker Inner Barrel (TIB): 4 layers covering $|z| < 65$~cm
\item Tracker Inner Disks (TID): 3 disks on each side
\item Tracker Outer Barrel (TOB): 6 layers covering $|z| < 110$~cm
\item Tracker Endcaps (TEC): 9 disks on each side
\end{itemize}

The total active silicon area is approximately 200~m$^2$, with about 75 million readout channels. The tracker achieves a transverse momentum resolution of $\sigma(\pT)/\pT \approx 1$--2\% for tracks with $\pT \approx 100\GeV$ in the central region.

\subsection{Electromagnetic Calorimeter}

The electromagnetic calorimeter (ECAL)~\cite{CMS:2013lxn} measures the energies of electrons and photons. It consists of lead tungstate (PbWO$_4$) crystals chosen for their short radiation length (0.89~cm), small Molière radius (2.2~cm), and fast scintillation response (80\% of light within 25~ns).

The ECAL is divided into:
\begin{itemize}
\item \textbf{Barrel (EB)}: 61,200 crystals covering $|\eta| < 1.479$, with front face dimensions of $22 \times 22$~mm$^2$
\item \textbf{Endcaps (EE)}: 7,324 crystals per endcap covering $1.479 < |\eta| < 3.0$, with front face dimensions of $28.6 \times 28.6$~mm$^2$
\item \textbf{Preshower (ES)}: Silicon strip detectors in front of the endcaps for improved $\pi^0$ rejection
\end{itemize}

Light from the crystals is detected by avalanche photodiodes (APDs) in the barrel and vacuum phototriodes (VPTs) in the endcaps. The energy resolution is:
\begin{equation}
\frac{\sigma(E)}{E} = \frac{2.8\%}{\sqrt{E}} \oplus \frac{12\%}{E} \oplus 0.3\%
\end{equation}
where $E$ is in GeV and $\oplus$ denotes addition in quadrature.

\subsection{Hadron Calorimeter}

The hadron calorimeter (HCAL)~\cite{CMS:2012sof} measures the energies of hadrons and contributes to the measurement of jets and missing transverse energy. The HCAL uses brass as the absorber material and plastic scintillator tiles as the active medium.

The HCAL consists of:
\begin{itemize}
\item \textbf{Barrel (HB)}: $|\eta| < 1.3$
\item \textbf{Endcap (HE)}: $1.3 < |\eta| < 3.0$
\item \textbf{Outer (HO)}: Additional layers outside the solenoid for tail catcher
\item \textbf{Forward (HF)}: Steel/quartz-fiber calorimeter covering $3.0 < |\eta| < 5.0$
\end{itemize}

The energy resolution for single hadrons is approximately $\sigma(E)/E \approx 100\%/\sqrt{E} \oplus 5\%$.

\subsection{Superconducting Solenoid}

The CMS solenoid~\cite{CMS:2009nxa} is a superconducting magnet that generates a uniform 3.8~T magnetic field inside its bore. Key parameters include:
\begin{itemize}
\item Inner bore diameter: 6.3~m
\item Length: 12.5~m
\item Stored energy: 2.6~GJ
\item Operating current: 18.2~kA
\end{itemize}

The magnetic flux is returned through an iron yoke that also serves as the structural support for the muon system. The strong magnetic field bends charged particle trajectories, enabling momentum measurements from track curvature.

\subsection{Muon System}

The muon system~\cite{CMS:2018rym} is located outside the solenoid and interleaved with the iron return yoke. It provides muon identification, triggering, and standalone momentum measurement. Three types of gaseous detectors are employed:

\subsubsection{Drift Tubes (DT)}

Drift tube chambers cover the barrel region ($|\eta| < 1.2$), where the muon rate is relatively low and the magnetic field is uniform. Four stations of chambers are embedded in the return yoke, each containing several layers of rectangular drift cells. The single-hit position resolution is approximately 200~$\mu$m.

\subsubsection{Cathode Strip Chambers (CSC)}

CSCs are used in the endcap regions ($0.9 < |\eta| < 2.4$), where the muon rate is higher and the magnetic field is non-uniform. Each CSC consists of six layers of anode wires interleaved with cathode strips. The position resolution is approximately 50--140~$\mu$m depending on the chamber type.

\subsubsection{Resistive Plate Chambers (RPC)}

RPCs provide complementary information in both barrel ($|\eta| < 1.6$) and endcap ($|\eta| < 1.9$) regions. While their spatial resolution is coarser ($\sim$1~cm), they provide excellent time resolution ($\sim$1~ns), making them particularly useful for triggering.

The muon system, combined with the inner tracker, achieves a muon momentum resolution of $\sigma(\pT)/\pT \approx 1$--2\% for $\pT < 100\GeV$ and better than 10\% up to $\pT \approx 1\TeV$.

\subsection{Trigger System}

The LHC bunch crossing rate of 40~MHz produces far more collision data than can be recorded. The CMS trigger system~\cite{CMS:2016ngn,CMS:2020cmk} reduces this rate to approximately 1~kHz of events written to permanent storage.

\subsubsection{Level-1 Trigger (L1)}

The Level-1 trigger is implemented in custom hardware and makes decisions within approximately 4~$\mu$s. It uses coarse-grained information from the calorimeters and muon systems to identify high-$\pT$ electrons, photons, muons, jets, and $\tau$-leptons, as well as global quantities such as total energy and missing transverse energy.

The L1 trigger reduces the event rate from 40~MHz to approximately 100~kHz.

\subsubsection{High-Level Trigger (HLT)}

The High-Level Trigger is a software-based system running on a processor farm. It has access to the full detector information and applies reconstruction algorithms similar to those used in offline analysis. The HLT reduces the rate to approximately 1~kHz.

For this analysis, events are selected using dilepton triggers requiring combinations of electrons and/or muons, as described in detail in Chapter~\ref{ch:selection}.

\section{Particle Flow Reconstruction}
\label{sec:pf}

The CMS detector employs the particle-flow (PF) algorithm~\cite{CMS:2017yfk} for event reconstruction. This algorithm aims to reconstruct each individual particle in an event by combining information from all detector subsystems.

The PF algorithm proceeds through several steps:
\begin{enumerate}
\item Reconstructing tracks in the silicon tracker and standalone segments in the muon system
\item Clustering energy deposits in the calorimeters
\item Linking these elements based on geometric compatibility
\item Identifying individual particles: muons, electrons, photons, charged hadrons, and neutral hadrons
\end{enumerate}

The output is a comprehensive list of PF candidates that serves as input for jet clustering, missing transverse momentum calculation, and isolation variable computation. The PF approach provides improved energy resolution for jets, better missing transverse momentum measurement, and more accurate lepton isolation compared to using individual detector subsystems alone.

\chapter{Datasets and Monte Carlo Simulation}
\label{ch:datasets}

This chapter describes the collision data and Monte Carlo (MC) simulated samples used in this analysis. The data were collected by the CMS experiment during the LHC Run~2 data-taking period (2016--2018) at a center-of-mass energy of $\sqrt{s} = 13\TeV$. MC simulations are used to model signal processes and various Standard Model background contributions.

\section{Data Samples}
\label{sec:data_samples}

\subsection{Integrated Luminosity}

The analysis uses proton-proton collision data collected during LHC Run~2, corresponding to a total integrated luminosity of 138\fbinv. The data are divided into several data-taking periods with different detector configurations and operating conditions:

\begin{table}[htbp]
\centering
\caption{Integrated luminosity for each data-taking period in Run~2.}
\label{tab:luminosity}
\begin{tabular}{lc}
\hline\hline
Data-taking period & Integrated luminosity (\fbinv) \\
\hline
2016 preVFP & 19.5 \\
2016 postVFP & 16.8 \\
2017 & 41.5 \\
2018 & 59.8 \\
\hline
Total & 138 \\
\hline\hline
\end{tabular}
\end{table}

The 2016 data are divided into ``preVFP'' and ``postVFP'' periods, corresponding to data taken before and after changes to the silicon strip tracker front-end electronics (VFP = ``Voltage Frontend Processor''). This division accounts for different tracking performance characteristics between the two periods.

\subsection{Data Quality Requirements}

Good quality events are selected using certified ``Golden JSON'' files provided by the CMS Luminosity Physics Object Group (LumiPOG). These files identify run and luminosity section ranges where all detector subsystems were functioning properly. Events outside these certified ranges are excluded from the analysis.

The luminosity measurement is performed using the CMS luminometer systems, primarily the Pixel Luminosity Telescope (PLT) and the Fast Beam Condition Monitor (BCM1F). The luminosity uncertainty is approximately 1--2\% depending on the data-taking year, with correlations between years carefully tracked.

\subsection{Primary Datasets and Trigger Strategy}

Events are collected using unprescaled dilepton triggers that provide high efficiency for the trilepton final states targeted by this analysis. The trigger strategy varies slightly between the two analysis channels:

\textbf{$\Pe\Pgm\Pgm$ channel}: Events are selected from the MuonEG primary dataset using electron-muon cross-triggers. These triggers require the simultaneous presence of an electron and a muon satisfying loose identification and isolation requirements.

\textbf{$\Pgm\Pgm\Pgm$ channel}: Events are selected from the DoubleMuon primary dataset using dimuon triggers. These triggers require two muons with tracking-based isolation requirements.

The specific trigger paths used vary by data-taking year to account for changes in the trigger menu. Representative triggers include:
\begin{itemize}
\item \texttt{HLT\_Mu17\_TrkIsoVVL\_Mu8\_TrkIsoVVL\_DZ\_Mass3p8}: Double muon trigger with isolation and invariant mass requirements
\item \texttt{HLT\_Mu8\_TrkIsoVVL\_Ele23\_CaloIdL\_TrackIdL\_IsoVL\_DZ}: Electron-muon cross trigger
\item \texttt{HLT\_Mu23\_TrkIsoVVL\_Ele12\_CaloIdL\_TrackIdL\_IsoVL\_DZ}: Muon-electron cross trigger with reversed $\pT$ thresholds
\end{itemize}

Additional prescaled single-lepton triggers are used for auxiliary measurements, such as the determination of fake rates for the nonprompt lepton background estimation. Unprescaled single-lepton triggers are used for measuring lepton identification and trigger efficiencies.

\section{Monte Carlo Simulation}
\label{sec:mc_simulation}

Monte Carlo simulations play a crucial role in this analysis, providing modeling of signal processes and irreducible backgrounds, training data for machine learning algorithms, and templates for systematic uncertainty evaluation.

\subsection{Event Generation}

MC samples are generated using various event generators depending on the physics process:

\textbf{\MGvATNLO}~\cite{MG5_aMCatNLO}: Used for processes requiring next-to-leading order (NLO) accuracy with matched parton showers. Version 2.6.5 is used for Run~2 samples. The FxFx~\cite{FXFXMerging} matching scheme is employed for processes with additional jets at the matrix element level.

\textbf{\POWHEG}~\cite{Powheg1,Powheg2,Powheg3}: Used for several processes at NLO accuracy, particularly diboson production, $\ttbar$ production, and Higgs boson production. \POWHEG provides consistent matching of NLO calculations with parton shower algorithms.

\textbf{\PYTHIA}~\cite{Pythia8}: Used for parton shower simulation, hadronization, and underlying event modeling for all samples. The CP5 tune~\cite{CPTunes} is used to describe the underlying event characteristics.

All samples use the NNPDF3.1 NNLO parton distribution function (PDF) set~\cite{NNPDF31}.

\subsection{Detector Simulation}

Generated events are processed through a full simulation of the CMS detector response using \GEANTfour~\cite{Geant4}. This simulation models the passage of particles through detector material, including electromagnetic and hadronic interactions, multiple scattering, and energy deposition in sensitive detector elements.

The simulated detector response is then processed through the same reconstruction algorithms as collision data. Corrections are applied to account for known differences between simulation and data, including lepton identification efficiencies, trigger efficiencies, and energy scale calibrations.

\subsection{Pileup Modeling}

Pileup interactions are included in the simulation using minimum-bias events generated with \PYTHIA. The distribution of pileup multiplicity in simulation is reweighted to match the distribution observed in data, using the recommended minimum-bias cross section value of 69.2~mb.

The pileup reweighting procedure is validated by comparing the distributions of the number of reconstructed primary vertices between data and simulation.

\section{Background Monte Carlo Samples}
\label{sec:bkg_mc}

\subsection{Prompt Trilepton Backgrounds}

Several SM processes produce three or more prompt leptons in the final state and constitute irreducible backgrounds:

\textbf{Diboson production ($\PW\PZ$, $\PZ\PZ$)}: The dominant irreducible background arises from $\PW\PZ$ production where both bosons decay leptonically ($\PW\PZ \to 3\ell\nu$). The $\PZ\PZ \to 4\ell$ process contributes when one lepton is not reconstructed or fails identification requirements. These processes are simulated at NLO accuracy, with the $\PZ\PZ$ cross section scaled by a k-factor of 1.16 to account for NNLO corrections~\cite{ZZNNLOCalc}.

\textbf{$\ttbar$ + V/H production}: Processes where a top quark pair is produced in association with a \PW boson, \PZ boson, or Higgs boson can produce multilepton final states. The $\ttbar\PZ \to \ell\ell\nu\nu\PQb\PAQb\ell\ell$ process is particularly relevant as it can mimic the signal topology.

\textbf{Single top + Z ($\cPqt\PZ\cPq$)}: Single top production in association with a \PZ boson contributes to the trilepton final state.

\textbf{Triboson production}: Processes such as $\PW\PW\PW$, $\PW\PW\PZ$, $\PW\PZ\PZ$, and $\PZ\PZ\PZ$ have small cross sections but produce multilepton signatures.

\textbf{Higgs boson production}: SM Higgs production through gluon fusion, vector boson fusion, and associated production with vector bosons contributes when the Higgs decays to $\PZ\PZ^* \to 4\ell$.

\subsection{Conversion Backgrounds}

Photon conversions, both external (in detector material) and internal (Dalitz decays), can produce additional leptons that mimic the signal topology:

\textbf{Drell-Yan + $\gamma$ ($\PZ\gamma$)}: Events with a \PZ boson and an associated photon, where the photon converts to an electron-positron pair, can enter the selection if one conversion electron fails identification.

\textbf{$\ttbar\gamma$}: Top pair production with an associated photon can contribute through similar conversion mechanisms.

\subsection{Control Region Samples}

Additional MC samples are used for control region studies and validation:

\textbf{Drell-Yan ($\PZ/\gamma^* \to \ell\ell$)}: Large samples of $\PZ$ boson production are used for lepton efficiency measurements, fake rate validation, and conversion background studies.

\textbf{$\ttbar$}: Top pair production samples are used for validation of the nonprompt lepton estimation method.

\textbf{Single top}: Single top production in various channels (t-channel, s-channel, $\cPqt\PW$) contributes to control region studies.

\section{Signal Monte Carlo Samples}
\label{sec:signal_mc}

\subsection{Signal Process Generation}

Signal MC samples for the process $\Pp\Pp \to \ttbar \to (\PHc\PQb)(\PW\PQb)$ are generated at leading order using the \MGvATNLO generator in the 5-flavor scheme. The decay chain $\PHc \to \PW\PA \to \PW\mu^+\mu^-$ is also performed within \MGvATNLO.

Key features of the signal generation include:

\begin{itemize}
\item Top quark mass fixed at $m_t = 172.5\GeV$, consistent with SM MC samples
\item Narrow width approximation for \PHc and \PA, with widths set to 1~MeV
\item All \PW boson decay modes included except fully hadronic decays
\item Both on-shell and off-shell $\PHc \to \PW^{(*)}\PA$ decays considered
\end{itemize}

\subsection{Signal Mass Points}

Signal samples are generated for a grid of $(\mHc, \mA)$ mass points covering the accessible parameter space. The charged Higgs mass ranges from 70 to 160\GeV, while the pseudoscalar mass ranges from 15\GeV up to $\mHc - 5\GeV$.

\begin{table}[htbp]
\centering
\caption{Signal mass points $(\mHc, \mA)$ in GeV used in the analysis.}
\label{tab:signal_masses}
\begin{tabular}{cl}
\hline\hline
$\mHc$ (\GeVns) & $\mA$ (\GeVns) \\
\hline
70 & 15, 18, 40, 55, 65 \\
100 & 15, 21, 60, 70, 80, 95 \\
115 & 15, 27, 87, 110 \\
130 & 15, 30, 55, 83, 90, 100, 125 \\
145 & 15, 35, 92, 140 \\
160 & 15, 18, 21, 27, 30, 35, 40, 50, 55, 60, 65, 70, \\
    & 80, 85, 87, 90, 92, 95, 98, 100, 110, 120, 125, 135, 140, 155 \\
\hline\hline
\end{tabular}
\end{table}

The densest sampling is at $\mHc = 160\GeV$, which is the boundary between the light ($\mHc < m_t$) and heavy ($\mHc > m_t$) charged Higgs regimes. Additional mass points are included in regions where the sensitivity is expected to vary rapidly with mass.

\subsection{Signal Cross Section Definition}

The signal cross section is defined as:
\begin{equation}
\sigma_{\text{sig}} = \sigma(\Pp\Pp \to \ttbar) \times \left[\mathcal{B}(\PQt \to \PW\PQb)\mathcal{B}(\PAQt \to \PHm\PAQb)\mathcal{B}(\PHm \to \PWm\PA)\mathcal{B}(\PA \to \Pgmp\Pgmm) + \text{c.c.}\right]
\end{equation}

The $\ttbar$ production cross section at $\sqrt{s} = 13\TeV$ is taken to be $\sigma_{\ttbar} = 833.9$~pb for $m_t = 172.5\GeV$ and $\alpha_s = 0.118$, following the ATLAS-CMS recommended prediction~\cite{TTbarXsecRecommedation}.

The branching fraction for the \PW boson decays in signal events (excluding fully hadronic decays) is:
\begin{equation}
\mathcal{B}(\PW\PW \to \ell\nu\Pq\Pq'/\ell\nu\ell'\nu') = 1 - \mathcal{B}(\PW \to \Pq\Pq')^2 = 1 - 0.6741^2 = 0.5456
\end{equation}

\subsection{Signal Acceptance and Efficiency}

The signal acceptance and reconstruction efficiency depend on both $\mHc$ and $\mA$. Key features of the signal topology that affect acceptance include:

\begin{itemize}
\item Harder muon $\pT$ spectrum from \PA decay for larger $\mA$
\item Softer muon $\pT$ spectrum in the off-shell region ($\mA > \mHc - m_\PW$)
\item Varying dimuon invariant mass resolution depending on $\mA$
\item Different jet and $\ptmiss$ characteristics across the mass plane
\end{itemize}

These variations motivate the mass-dependent optimization of the analysis selection and the scanning approach used for limit setting.

\section{Sample Processing}
\label{sec:sample_processing}

\subsection{Data Format}

The analysis uses the NanoAOD data format~\cite{NanoAOD}, which provides a compact representation of reconstructed physics objects suitable for analysis-level studies. NanoAODv9 is used for Run~2 data processing.

The NanoAOD format includes high-level physics objects (electrons, muons, jets, $\ptmiss$) with associated identification and isolation variables, generator-level information for MC samples, and trigger decisions and prescale information.

Custom additions to the standard NanoAOD content include variables needed for the ParticleNet classifier and additional generator-level information for signal modeling studies.

\subsection{Event Weights}

MC events are weighted to account for various effects:

\textbf{Cross section normalization}: Events are weighted by $\sigma \times \mathcal{L} / N_{\text{gen}}$, where $\sigma$ is the process cross section, $\mathcal{L}$ is the integrated luminosity, and $N_{\text{gen}}$ is the number of generated events.

\textbf{Pileup reweighting}: Events are reweighted to match the observed pileup distribution in data.

\textbf{Lepton efficiency corrections}: Scale factors are applied to correct for differences in lepton identification, isolation, and trigger efficiencies between data and simulation.

\textbf{b-tagging corrections}: Scale factors account for differences in b-tagging efficiency and mistag rates between data and simulation.

\textbf{L1 prefiring corrections}: For 2016 and 2017 data, corrections are applied to account for the L1 trigger prefiring effect, where spurious triggers from previous bunch crossings can veto real events.

The detailed treatment of these corrections and their associated systematic uncertainties is described in Chapter~\ref{ch:systematics}.

\chapter{Object Reconstruction and Identification}
\label{ch:objects}

This chapter describes the reconstruction and identification of physics objects used in this analysis: electrons, muons, jets, b-tagged jets, and missing transverse momentum. The identification criteria are designed to efficiently select prompt leptons from signal processes while suppressing backgrounds from misidentified jets and nonprompt leptons from heavy-flavor decays.

\section{Electrons}
\label{sec:electron_id}

\subsection{Electron Reconstruction}

Electrons are reconstructed by matching energy clusters in the electromagnetic calorimeter (ECAL) to tracks in the silicon tracker. The reconstruction algorithm begins with ECAL superclusters, which combine energy deposits accounting for bremsstrahlung photon emission. These superclusters are then matched to tracks reconstructed using a Gaussian Sum Filter (GSF) algorithm that accounts for energy loss through bremsstrahlung.

Signal electrons in this analysis typically have transverse momenta between a few \GeV and 100\GeV and are produced in the central detector region. Based on trigger acceptance, electrons with $\pT > 15\GeV$ are considered. The analysis uses electrons reconstructed within the tracker coverage ($|\eta| < 2.5$), excluding the ECAL barrel-endcap transition region ($1.442 < |\eta| < 1.566$) where reconstruction quality is degraded.

\subsection{Electron Identification}

The electron identification employs a boosted decision tree (BDT) discriminant provided by the EGM Physics Object Group~\cite{EGMPOGMVAID}. This multivariate classifier discriminates prompt electrons from misreconstructed objects and nonprompt electrons from photon conversions or heavy-flavor decays. Training is performed separately for three pseudorapidity regions: inner barrel ($|\eta| < 0.8$), outer barrel ($0.8 < |\eta| < 1.47$), and endcap ($|\eta| > 1.47$).

The classifier uses track and shower-shape variables as inputs while excluding isolation variables. This approach provides higher identification efficiency than cut-based selections, particularly for electrons below 40\GeV. The analysis uses the ``wp90'' working point, designed to provide approximately 90\% identification efficiency for electrons from \PW and \PZ boson decays.

\subsection{Isolation and Impact Parameter Requirements}

Additional requirements ensure that electrons originate from the primary vertex and are well-isolated from hadronic activity. The relative mini-isolation variable ($I_{\text{mini}}$) is used to quantify the hadronic activity around the electron:
\begin{equation}
I_{\text{mini}} = \frac{1}{\pT^\ell}\left(\sum p_T^{\text{charged}} + \max\left(0, \sum p_T^{\text{neutral}} - \rho \cdot A_{\text{eff}}\right)\right)
\end{equation}
where the sums are over charged hadrons, neutral hadrons, and photons within a cone of size $\Delta R$, and the pileup contribution is subtracted using the average transverse momentum density $\rho$ multiplied by an effective area $A_{\text{eff}}$.

The cone size varies with electron $\pT$ to account for the collimation of decay products from boosted particles:
\begin{equation}
\Delta R = \begin{cases}
0.2 & \text{for } \pT < 50\GeV \\
10\GeV/\pT & \text{for } 50 \leq \pT < 200\GeV \\
0.05 & \text{for } \pT \geq 200\GeV
\end{cases}
\end{equation}

This approach, originally developed for analyses with boosted top quarks~\cite{Rehermann_MiniIso}, proves effective in searches with high jet multiplicity or nonprompt lepton backgrounds from heavy-flavor decays.

The analysis requires $I_{\text{mini}} < 0.1$ for signal electrons, achieving approximately 90\% efficiency for prompt electrons and 15\% efficiency for nonprompt electrons from b jets. Additionally, a tight impact parameter requirement of $|\text{SIP3D}| < 4$ is applied, where SIP3D is the three-dimensional impact parameter significance. This provides approximately 95\% efficiency for prompt electrons and 40\% efficiency for nonprompt electrons.

\subsection{Trigger Emulation Cuts}

To reduce bias in the nonprompt background estimation arising from electron quality requirements in triggers, additional cuts emulating trigger-level selections are applied. These include requirements on shower shape variables ($\sigma_{i\eta i\eta}$), track-cluster matching variables ($\Delta\eta_{\text{in}}$, $\Delta\phi_{\text{in}}$), hadronic energy fraction (H/E), and relative isolation variables.

\subsection{Working Points}

Two working points are defined for electron identification:

\textbf{Tight working point}: Used for signal electron selection, requiring MVANoIso wp90, $I_{\text{mini}} < 0.1$, $|\text{SIP3D}| < 4$, at most one expected missing inner hit, and passing the conversion veto.

\textbf{Loose working point}: Used for background estimation in sideband regions and for vetoing additional low-quality electrons. Requirements are relaxed: MVANoIso $> (0.985, 0.96, 0.85)$ in the three $\eta$ regions, $I_{\text{mini}} < 0.4$, and $|\text{SIP3D}| < 8$.

\subsection{Efficiency Measurement}

Electron identification efficiency is measured using the tag-and-probe method with dielectron Drell-Yan events. Oppositely charged electron pairs with $60 < M(\Pe\Pe) < 120\GeV$ passing single-electron triggers are selected. Tag electrons must match the trigger and have $\pT$ above threshold values that vary by year.

The numbers of passing and failing probes are extracted by fitting the invariant mass distribution. Signal modeling uses a Drell-Yan MC template convolved with a Gaussian, while background is modeled with an empirical shape function. Scale factors are derived as the ratio of efficiencies measured in data and simulation.

Systematic uncertainties are evaluated through alternative signal and background shape models and different MC generators. The typical efficiency uncertainty is 1--2\%, with larger uncertainties in low-$\pT$ and high-$\pT$ regions due to limited statistics and generator dependence.

The electron identification efficiency measurements have been validated and approved by the EGM Physics Object Group~\cite{EGM-IDApproval}.

\section{Muons}
\label{sec:muon_id}

\subsection{Muon Reconstruction}

Muons are reconstructed by combining tracks in the silicon tracker with track segments in the muon chambers. The CMS muon reconstruction produces several types of muon objects:
\begin{itemize}
\item \textbf{Standalone muons}: Reconstructed using only muon chamber information
\item \textbf{Tracker muons}: Tracker tracks extrapolated to match muon chamber segments
\item \textbf{Global muons}: Combined fit of tracker and muon chamber hits
\end{itemize}

Signal muons typically have transverse momenta between a few \GeV and 100\GeV. Based on trigger acceptance, muons with $\pT > 10\GeV$ are considered. The analysis uses muons reconstructed within the full muon system acceptance ($|\eta| < 2.4$).

\subsection{Muon Identification}

The baseline identification uses the medium ID criteria defined by the Muon Physics Object Group~\cite{MuonPOGCBID}. This requires high-quality tracks with few kinks, a large fraction of compatible constituent segments, consistent track parameters between tracker and muon systems, and good global track fit quality. These criteria efficiently discriminate real muons from misreconstructed objects while maintaining high efficiency.

\subsection{Isolation and Impact Parameter Requirements}

To distinguish prompt muons from secondary muons originating from heavy-flavor decays in jets, additional requirements on isolation and impact parameter are applied.

The relative mini-isolation is defined analogously to electrons:
\begin{equation}
I_{\text{mini}} = \frac{1}{\pT^\mu}\left(\sum p_T^{\text{charged}} + \max\left(0, \sum p_T^{\text{neutral}} - \rho \cdot A_{\text{eff}}\right)\right)
\end{equation}
with the same $\pT$-dependent cone size. The analysis requires $I_{\text{mini}} < 0.1$ for signal muons, achieving approximately 90\% efficiency for prompt muons and only 4\% efficiency for nonprompt muons from b jets. Compared to standard PF isolation with $\Delta R = 0.4$ at the same background efficiency, the mini-isolation approach recovers approximately 10\% identification efficiency per muon.

A tight impact parameter requirement of $|\text{SIP3D}| < 3$ is applied, providing approximately 95\% efficiency for prompt muons and 20\% efficiency for nonprompt muons from b jets.

\subsection{Working Points}

Two working points are defined:

\textbf{Tight working point}: POG medium ID, $I_{\text{mini}} < 0.1$, $|\text{SIP3D}| < 3$, $|d_z| < 0.1$~cm, and relative tracker isolation (R03) $< 0.4$.

\textbf{Loose working point}: POG medium ID, $I_{\text{mini}} < 0.6$, $|\text{SIP3D}| < 5$, with the same $d_z$ and tracker isolation requirements.

\subsection{Momentum Corrections}

Rochester corrections~\cite{RochCorr} are applied to muon momenta to account for possible biases from detector alignment and magnetic field differences between data and simulation. These corrections improve the dimuon mass resolution and reduce systematic effects from momentum scale uncertainties.

\subsection{Efficiency Measurement}

Muon reconstruction and identification efficiency is measured using the tag-and-probe method with dimuon Drell-Yan events. Oppositely charged muon pairs with $70 < M(\mu\mu) < 110\GeV$ are selected. Tag muons must pass POG tight ID and medium isolation criteria and match single-muon triggers.

The efficiency is measured as a function of muon $\pT$ and $|\eta|$. Scale factors are derived by comparing efficiencies in data and simulation. Systematic uncertainties are evaluated by varying tag muon criteria, the number of mass bins, signal and background shapes, alternative generators, and the fitting mass range.

The muon identification efficiency measurements have been validated and approved by the Muon Physics Object Group~\cite{MUO-IDApproval-Run2, MUO-IDApproval-Run3}.

\section{Jets}
\label{sec:jet_id}

\subsection{Jet Reconstruction}

Jets are reconstructed using the anti-$k_t$ algorithm~\cite{Cacciari:2008gp} with distance parameter $R = 0.4$, using particle-flow candidates as inputs. For Run~2, jets are reconstructed using the charged hadron subtraction (CHS) approach to mitigate pileup effects (AK4PFchs). For Run~3, the pileup per particle identification (PUPPI) weighting scheme is used instead (AK4PFPuppi).

The analysis considers jets with $\pT > 20\GeV$ and $|\eta| < 2.4$ (2.5 for 2017--2023 data). Signal events contain jets from b quarks produced in top decays (typically peaking around 70\GeV) and jets from \PW boson decays (peaking around 40\GeV).

\subsection{Jet Energy Corrections}

Jet energy scale (JES) corrections are applied following the recommendations of the JME Physics Object Group. The corrections include:
\begin{itemize}
\item \textbf{L1 pileup correction}: Removes energy from pileup interactions
\item \textbf{L2L3 MC truth correction}: Corrects jet response based on simulation
\item \textbf{L2L3 residual correction}: Applied to data to correct for remaining data-simulation differences
\end{itemize}

Jet energy resolution (JER) corrections are also applied to simulation to match the resolution observed in data. These corrections are provided by the JME POG and are applied using the correctionlib framework~\cite{JERCRecommendation}.

\subsection{Jet Identification and Cleaning}

Jets must pass tight jet ID criteria to reject jets arising from detector noise. The jet identification efficiency exceeds 98\% with background rejection better than 98\%.

For Run~2 data with $\pT < 50\GeV$, additional loose pileup jet ID criteria are applied to reject jets originating from pileup interactions~\cite{PileupJetID}.

Since leptons are included in particle-flow jet reconstruction, jets must be separated from identified leptons by $\Delta R > 0.4$ to avoid double counting. This cleaning procedure also ensures consistent treatment in the nonprompt lepton background estimation.

Additionally, jet veto maps provided by the JME POG~\cite{JetVetoMap} are applied to reject jets or events affected by known detector issues during Run~2 and Run~3.

\section{B-Tagged Jets}
\label{sec:btag}

\subsection{B-Tagging Algorithm}

Jets originating from b quarks are identified using the DeepJet algorithm~\cite{Bols_DeepJet}, a deep neural network that uses information from all jet constituents. Unlike previous algorithms that relied on a subset of high-level variables related to heavy-flavor decay signatures, DeepJet exploits the full information content of the jet, including track impact parameters, secondary vertex properties, and charged particle kinematics.

\subsection{Working Point}

The analysis uses the medium working point defined by the BTV Physics Object Group~\cite{BTVRecommendation}, providing approximately 80\% efficiency for b jets and 1\% misidentification rate for light-flavor jets. This working point offers a good balance between signal efficiency and background rejection for the $\ttbar$-dominated final state.

\subsection{Efficiency Corrections}

Differences between data and simulation in b-tagging efficiency and mistag rates are corrected using flavor-dependent scale factors measured in QCD multijet events. The corrections are applied on a jet-by-jet basis using the ``method 1a'' approach, accounting for both the tagging efficiency for true b jets and the misidentification probability for other jet flavors.

\section{Missing Transverse Momentum}
\label{sec:met}

\subsection{Reconstruction}

Missing transverse momentum ($\ptmiss$) is reconstructed as the negative vector sum of the transverse momenta of all particle-flow candidates, weighted using the PUPPI algorithm (PUPPI-MET). The PUPPI weighting suppresses contributions from pileup interactions.

Type-1 corrections are applied to propagate jet energy scale corrections and lepton energy corrections to the $\ptmiss$ calculation. This improves the $\ptmiss$ resolution and reduces systematic effects from object energy scale uncertainties.

\subsection{Event Filters}

Events with anomalous $\ptmiss$ due to detector noise, poorly reconstructed objects, or beam backgrounds are rejected using event noise filters recommended by the JME POG~\cite{METFilters}. These filters include requirements on primary vertex quality, beam halo rejection, HCAL noise rejection, ECAL dead cell handling, and identification of anomalous particle-flow muons.

The complete set of noise filters applied varies between Run~2 and Run~3 data, following the official recommendations for each data-taking period.

\chapter{Event Selection and Signal Extraction Strategy}
\label{ch:selection}

This chapter describes the event selection criteria used to identify signal candidate events and the strategy employed to extract a potential signal from the dominant backgrounds. The analysis uses two complementary approaches: a baseline selection exploiting the distinctive kinematic features of the signal, and a machine learning-based classifier to enhance discrimination in regions where traditional methods are less effective.

\section{Signal Region Definition}
\label{sec:signal_region}

The signal region is designed to select events consistent with the charged Higgs boson decay topology while suppressing backgrounds. Events are categorized into two channels based on the lepton flavors: the $\Pe\Pgm\Pgm$ channel (one electron, two oppositely-charged muons) and the $\Pgm\Pgm\Pgm$ channel (three muons with total charge $\pm 1$).

\subsection{Baseline Selection}

The baseline selection optimizes signal sensitivity while maintaining efficiency across the explored mass range. The key selection requirements are:

\textbf{Lepton multiplicity}: Exactly three leptons passing tight identification criteria are required, with no additional leptons passing loose identification. In the $\Pe\Pgm\Pgm$ channel, this requires exactly one electron and two oppositely-charged muons. In the $\Pgm\Pgm\Pgm$ channel, exactly three muons with a charge sum of $\pm 1$ are required.

\textbf{Lepton transverse momentum}: Electrons must have $\pT > 15\GeV$ (10\GeV for loose identification), and muons must have $\pT > 10\GeV$. To ensure high trigger efficiency, additional constraints are applied based on the trigger thresholds.

For the $\Pe\Pgm\Pgm$ channel, events must satisfy either $\pT(\Pgm_1) > 25\GeV$ and $\pT(\Pe) > 15\GeV$, or $\pT(\Pgm_1) > 10\GeV$ and $\pT(\Pe) > 25\GeV$. For the $\Pgm\Pgm\Pgm$ channel, the leading muon must have $\pT > 20\GeV$ and the subleading muon must have $\pT > 10\GeV$.

\textbf{Dimuon invariant mass}: The invariant mass of each opposite-sign muon pair must exceed 12\GeV to suppress backgrounds from low-mass resonances such as $J/\psi$ and $\Upsilon$ mesons, as well as virtual photon contributions.

\textbf{Jet requirements}: At least two jets with $\pT > 20\GeV$ and $|\eta| < 2.4$ are required, consistent with the signal topology containing two b quarks from top decays. Additionally, at least one jet must be b-tagged using the DeepJet medium working point. These requirements strongly suppress diboson and Drell-Yan backgrounds while maintaining good signal efficiency.

\subsection{Trigger Strategy}

Events are collected using unprescaled dilepton triggers designed for trilepton analyses. The $\Pe\Pgm\Pgm$ channel uses electron-muon cross-triggers from the MuonEG primary dataset, while the $\Pgm\Pgm\Pgm$ channel uses dimuon triggers from the DoubleMuon (Run~2) or Muon (Run~3) primary datasets.

Representative trigger paths include:
\begin{itemize}
\item \texttt{HLT\_Mu17\_TrkIsoVVL\_Mu8\_TrkIsoVVL\_DZ\_Mass3p8}: Dimuon trigger with isolation and mass requirements
\item \texttt{HLT\_Mu8\_TrkIsoVVL\_Ele23\_CaloIdL\_TrackIdL\_IsoVL\_DZ}: Electron-muon cross trigger
\item \texttt{HLT\_Mu23\_TrkIsoVVL\_Ele12\_CaloIdL\_TrackIdL\_IsoVL\_DZ}: Muon-electron cross trigger with reversed thresholds
\end{itemize}

The trigger efficiency for signal events passing the offline selection exceeds 95\% in both channels.

\section{Control Regions}
\label{sec:control_regions}

Several control regions are defined orthogonal to the signal region to validate background estimation methods and measure data-driven correction factors.

\subsection{Z+Fake Control Region}

This region is enriched in events containing a genuine \PZ boson decay to muons plus a nonprompt or fake lepton. The selection is identical to the signal region except:
\begin{itemize}
\item The opposite-sign dimuon mass must be within 10\GeV of the \PZ mass: $|M(\Pgm^+\Pgm^-) - 91.2\GeV| < 10\GeV$
\item No b-tagged jets are required ($N_b = 0$)
\end{itemize}

This control region is used to validate the nonprompt background estimation method and assess the modeling of \PZ boson kinematics.

\subsection{Conversion Control Region}

This region targets events where a photon converts to an electron-positron pair, mimicking the trilepton signature. The selection requires:
\begin{itemize}
\item The opposite-sign dimuon mass must be away from the \PZ peak: $|M(\Pgm^+\Pgm^-) - 91.2\GeV| > 10\GeV$
\item The trilepton mass must be consistent with a \PZ boson: $|M(3\ell) - 91.2\GeV| < 10\GeV$
\item Low missing transverse momentum: $\ptmiss < 40\GeV$
\item No b-tagged jets ($N_b = 0$)
\end{itemize}

This region is dominated by $\PZ\gamma$ events where the photon undergoes external or internal conversion, and is used to derive conversion background scale factors.

\subsection{WZ Control Region}

For Run~3 data, an additional \PW\PZ control region is defined to validate the diboson background modeling following changes in the MC sample production. The selection requires:
\begin{itemize}
\item The opposite-sign dimuon mass must be within 10\GeV of the \PZ mass
\item High missing transverse momentum: $\ptmiss > 40\GeV$ (consistent with a leptonic \PW decay)
\item No b-tagged jets ($N_b = 0$)
\end{itemize}

\section{Dimuon Pair Selection in the $\Pgm\Pgm\Pgm$ Channel}
\label{sec:pair_selection}

In the $\Pgm\Pgm\Pgm$ channel, there are two possible opposite-sign muon pairs that could originate from the pseudoscalar \PA decay. The correct assignment is determined based on studies using signal simulation.

For signal mass hypotheses with $\mHc > 100\GeV$ and $\mA > 60\GeV$, selecting the pair with the larger invariant mass provides the correct assignment in 50--70\% of events. For other mass hypotheses, selecting the pair with the smaller invariant mass yields correct assignments in 60--95\% of events. The analysis uses the optimal assignment strategy determined from simulation for each mass point.

\section{ParticleNet-Based Event Classification}
\label{sec:particlenet}

The previous CMS search using 2016 data~\cite{CMSRun2ChargedHiggs} relied primarily on the dimuon invariant mass distribution, where the narrow \PA resonance produces a distinct peak. However, when $\mA$ approaches $\mZ$, the signal peak overlaps with the abundant \PZ boson background, significantly reducing the discrimination power.

To address this challenge, this analysis employs a ParticleNet-based~\cite{ParticleNet} event classifier to distinguish signal from \PZ-associated backgrounds. ParticleNet is a graph neural network originally designed for jet tagging but applied here for event-level classification, treating each reconstructed object as a node in a particle cloud.

\subsection{Classifier Architecture}

The ParticleNet model transforms events into particle cloud representations, with each particle type converted into graph nodes with specific features:

\begin{itemize}
\item \textbf{Muons and electrons}: Four-momentum components ($E$, $p_x$, $p_y$, $p_z$), charge, and one-hot encoded particle type identification
\item \textbf{Jets}: Four-momentum components and b-tagging information
\item \textbf{Missing transverse momentum}: Four-momentum representation with $\eta$ and mass set to zero
\end{itemize}

The network architecture consists of:
\begin{enumerate}
\item A graph normalization layer for stable training
\item Three Dynamic Edge Convolution (EdgeConv) layers that dynamically construct k-nearest neighbor graphs in feature space ($k=4$), with residual connections and dropout regularization
\item Global mean pooling to aggregate node-level features into graph-level representations
\item Two fully connected layers with batch normalization and dropout
\item An output layer producing four class logits: signal, nonprompt background, diboson background, and $\ttbar$+X background
\end{enumerate}

\subsection{Training Procedure}

The classifier is trained using relaxed event selection criteria to enrich the training dataset and enable validation in regions outside the signal region. The relaxations include loosening lepton identification from tight to loose, removing the trigger requirement, and omitting the b-jet requirement.

Separate classifiers are trained for signal mass points with $60 < \mA < 120\GeV$, where discrimination from \PZ backgrounds is most challenging. For each mass point, a classifier is trained using:
\begin{itemize}
\item Signal MC samples at the target mass point
\item $\ttbar$ MC for the nonprompt background class
\item $\PW\PZ$ and $\PZ\PZ$ MC for the diboson class
\item $\ttbar\PZ$ and $\cPqt\PZ\cPq$ MC for the $\ttbar$+X class
\end{itemize}

Training uses 5-fold cross-validation with a 3:1:1 split for training, validation, and test sets. A weighted cross-entropy loss function accounts for normalization differences among background sources. Hyperparameter optimization is performed using a genetic algorithm, searching over the number of hidden nodes, optimizer choice, learning rate scheduler, initial learning rate, and weight decay.

\subsection{Classifier Output}

The trained classifier produces four softmax-normalized probability scores for each event. For signal extraction, likelihood ratios are constructed:
\begin{equation}
\text{LR}(\text{signal vs. background}_i) = \frac{P_{\text{signal}}}{P_{\text{signal}} + P_{\text{background}_i}}
\end{equation}
where $P$ denotes the classifier output probability for each class.

These likelihood ratios provide discrimination between signal and specific background categories, with values near 1 indicating signal-like events and values near 0 indicating background-like events.

\subsection{Validation}

The ParticleNet classifier is validated using a $\ttbar$+\PZ control region defined with $\Pe\Pe\Pgm$ lepton configuration. This region is kinematically similar to the signal region but essentially signal-free, as the $\PA \to \Pe\Pe$ decay is suppressed by $(m_e/m_\mu)^2$ due to Yukawa coupling. During inference on this region, electron and muon labels are swapped so the model interprets these events as $\ttbar$+\PZ with $\PZ \to \Pgm\Pgm$.

Good agreement between data and prediction is observed in the classifier output distributions, confirming the validity of the ParticleNet modeling.

\section{Signal Extraction Strategy}
\label{sec:extraction}

Signal extraction is performed through a binned maximum likelihood fit to the dimuon invariant mass distribution. The fit is performed separately for each $(\mHc, \mA)$ mass hypothesis, using appropriate templates for signal and background contributions.

\subsection{Mass-Dependent Strategy}

The signal extraction strategy varies depending on the \PA mass relative to the \PZ mass:

\textbf{Off-Z region} ($\mA$ far from $\mZ$): For mass points where the signal dimuon mass peak is well-separated from the \PZ peak, the dimuon invariant mass alone provides excellent discrimination. In this region, the analysis relies primarily on the narrow signal resonance structure in the mass distribution.

\textbf{On-Z region} ($\mA$ near $\mZ$): For mass points where the signal overlaps with the \PZ peak, the ParticleNet classifier output is used to enhance discrimination. Events are categorized based on the likelihood ratio value, and the fit is performed in bins of both dimuon mass and classifier score.

\subsection{Template Construction}

Signal templates are constructed from MC simulation, accounting for the narrow intrinsic width of the \PA boson (set to 1~MeV in simulation) and the experimental dimuon mass resolution. Background templates are derived from:
\begin{itemize}
\item MC simulation for prompt backgrounds (diboson, $\ttbar$+V/H)
\item Data-driven methods for nonprompt backgrounds (matrix method)
\item MC simulation with data-driven scale factors for conversion backgrounds
\end{itemize}

\subsection{Statistical Analysis}

The statistical analysis uses the asymptotic CL$_s$ method~\cite{Read:2002hq,Junk:1999kv} implemented in the Combine tool~\cite{CMS-NOTE-2011-005}. Upper limits are set at 95\% confidence level on the signal cross section for each $(\mHc, \mA)$ mass hypothesis.

Systematic uncertainties are incorporated as nuisance parameters with appropriate correlations across data-taking periods and channels. The dominant systematic uncertainties include the nonprompt background normalization, theoretical cross sections for prompt backgrounds, and experimental uncertainties on lepton identification and jet energy scale.

\section{Expected Sensitivity}
\label{sec:sensitivity}

The ParticleNet classifier provides significant sensitivity improvements compared to using the dimuon mass distribution alone, particularly in the on-Z region. The improvement is quantified as the ratio of expected limits with and without the classifier:
\begin{itemize}
\item $\Pe\Pgm\Pgm$ channel: 25--50\% improvement in expected limit
\item $\Pgm\Pgm\Pgm$ channel: 10--20\% improvement in expected limit
\end{itemize}

The larger improvement in the $\Pe\Pgm\Pgm$ channel reflects the more challenging background composition in this channel, where the classifier provides greater discrimination power.

\chapter{Background Estimation}
\label{ch:background}

This chapter describes the methods used to estimate the various background contributions in the signal region. The backgrounds are categorized into three main types: nonprompt leptons from jet misidentification or heavy-flavor decays, conversion backgrounds from photon interactions, and prompt backgrounds from irreducible SM processes. Data-driven methods are employed for the first two categories, while prompt backgrounds are estimated from Monte Carlo simulation.

\section{Overview of Background Composition}
\label{sec:bkg_overview}

The background composition in the signal region varies between the two analysis channels:

\textbf{$\Pe\Pgm\Pgm$ channel}: The dominant backgrounds are nonprompt leptons ($\sim$46\%), $\ttbar$+X production ($\sim$32\%), and diboson production ($\sim$11\%). Conversion backgrounds contribute approximately 6\%.

\textbf{$\Pgm\Pgm\Pgm$ channel}: Similar composition with nonprompt leptons ($\sim$46\%), $\ttbar$+X ($\sim$32\%), diboson ($\sim$13\%), and smaller conversion contributions ($\sim$1\%).

The nonprompt background is estimated using a data-driven matrix method, conversion backgrounds use MC simulation with data-driven scale factors, and prompt backgrounds are taken directly from MC simulation.

\section{Nonprompt Lepton Background}
\label{sec:nonprompt}

Nonprompt leptons arise from jets misidentified as leptons or from genuine leptons produced in heavy-flavor hadron decays within jets. These leptons typically fail tight isolation requirements but can pass identification criteria, making them a significant background for trilepton analyses.

\subsection{The Matrix Method}

The matrix method (also known as the fake rate method) extrapolates from control regions enriched in nonprompt leptons to estimate their contribution in the signal region. The method relies on defining two lepton identification criteria: ``tight'' (the signal selection) and ``loose'' (relaxed requirements).

For a sample of events containing one lepton of interest (with other leptons fixed), the number of events passing tight ($N_T$) and loose ($N_L$) identification can be written as:
\begin{align}
N_T &= \epsilon_p N_p + \epsilon_f N_f \\
N_L &= N_p + N_f
\end{align}
where $N_p$ and $N_f$ are the numbers of prompt and fake (nonprompt) leptons, $\epsilon_p$ is the tight-to-loose efficiency for prompt leptons, and $\epsilon_f$ is the corresponding efficiency for fake leptons (the ``fake rate'').

Solving for the number of fake leptons passing tight identification:
\begin{equation}
N_f^T = \epsilon_f N_f = \frac{\epsilon_f}{\epsilon_p - \epsilon_f}(\epsilon_p N_L - N_T)
\end{equation}

\subsection{Simplification for Tight Identification}

For the tight identification criteria used in this analysis, the prompt lepton efficiency $\epsilon_p$ is very close to unity. Under this approximation, the expression simplifies to:
\begin{equation}
N_f^T \approx \frac{\epsilon_f}{1 - \epsilon_f}(N_L - N_T) = \frac{f}{1-f} N_{L\bar{T}}
\end{equation}
where $f \equiv \epsilon_f$ is the fake rate and $N_{L\bar{T}}$ is the number of leptons passing loose but failing tight identification.

\subsection{Extension to Multiple Leptons}

For events with multiple leptons, the matrix method is extended by considering all possible combinations of prompt and fake leptons. For a trilepton event with leptons $\ell_1$, $\ell_2$, $\ell_3$, the nonprompt background contribution is:
\begin{equation}
N_{\text{fake}}^{TTT} = \sum_{\text{combinations}} w(\ell_1) \cdot w(\ell_2) \cdot w(\ell_3) \cdot N_{\text{data}}
\end{equation}
where the weights $w$ depend on whether each lepton passes ($T$) or fails ($\bar{T}$) tight identification:
\begin{equation}
w(\ell) = \begin{cases}
-f/(1-f) & \text{if lepton passes tight ID} \\
+f/(1-f) & \text{if lepton fails tight ID}
\end{cases}
\end{equation}

Events where all three leptons pass tight identification contribute with alternating signs depending on the number of assumed fake leptons, ensuring proper subtraction of the prompt component.

\subsection{Fake Rate Measurement}

The fake rate $f$ is measured in control regions enriched in QCD multijet events, where the lepton is likely to be nonprompt. Single-lepton events selected with prescaled triggers are used, requiring:
\begin{itemize}
\item Exactly one lepton passing loose identification
\item Low missing transverse momentum ($\ptmiss < 20\GeV$) to suppress \PW boson contributions
\item Low transverse mass ($M_T < 20\GeV$) as an additional \PW veto
\item Away-side jet requirement to ensure a well-defined QCD topology
\end{itemize}

The fake rate is measured as a function of lepton $\pT$ and $|\eta|$, accounting for the kinematic dependence of the misidentification probability. Contamination from prompt leptons (primarily from \PW and \PZ decays) is subtracted using MC simulation.

\subsection{Fake Rate Parametrization}

The measured fake rates show strong dependence on lepton $\pT$, typically decreasing at higher momenta where isolation requirements become more discriminating. The fake rate also depends on the pseudorapidity region due to varying detector conditions.

For electrons, typical fake rates range from 10--30\% at low $\pT$ to a few percent at high $\pT$. For muons, fake rates are generally lower, ranging from 5--20\% at low $\pT$ to below 5\% at high $\pT$.

\subsection{Systematic Uncertainties}

Systematic uncertainties on the fake rate measurement arise from several sources:
\begin{itemize}
\item \textbf{Prompt subtraction}: Uncertainty in the MC modeling of prompt contamination in the measurement region
\item \textbf{Source dependence}: The fake rate can depend on the parent process (e.g., b jets vs. light jets)
\item \textbf{Jet energy scale}: The fake rate depends on the mother jet properties, introducing sensitivity to jet energy calibration
\end{itemize}

A MC closure test is performed to validate the method. The fake rate measured in simulation is applied to simulated events with known composition, and the predicted nonprompt yield is compared to the truth-level expectation. The closure is typically within 25--35\%, and this non-closure is included in the systematic uncertainty.

Based on these studies, a flat systematic uncertainty of 30\% is assigned to the nonprompt background normalization. This uncertainty is treated as uncorrelated between data-taking periods, as the fake rate behavior depends on the specific lepton identification criteria and detector conditions of each era.

\section{Conversion Background}
\label{sec:conversion}

Photon conversions can produce additional electrons that mimic the signal topology. Two types of conversions are considered:

\textbf{External conversions}: Photons converting to electron-positron pairs in detector material (primarily the silicon tracker and beam pipe).

\textbf{Internal conversions} (Dalitz decays): Virtual photons from meson decays ($\pi^0 \to \gamma^* \gamma \to e^+e^- \gamma$) producing electron pairs.

\subsection{Modeling of Conversion Backgrounds}

Conversion backgrounds are modeled using MC simulation of processes that produce prompt photons, primarily $\PZ\gamma$ and $\ttbar\gamma$ production. The photon radiation is included in the simulation through QED showering on the generated particles.

Conversion electrons are identified at generator level as reconstructed electrons that:
\begin{itemize}
\item Do not match a generator-level final-state electron within $\Delta R < 0.1$
\item Match a generator-level prompt photon with $\Delta\pT/\pT < 0.5$ and $\Delta R < 0.2$
\item Have a photon origin without hadronic ancestors
\end{itemize}

\subsection{Conversion Control Region}

A dedicated control region enriched in conversion events is defined to validate the MC modeling and derive data-driven scale factors. The selection requires:
\begin{itemize}
\item Trilepton events with the dimuon mass away from the \PZ peak
\item Trilepton invariant mass consistent with the \PZ mass (targeting $\PZ \to \ell\ell\gamma^{(*)}$)
\item Low $\ptmiss$ to suppress backgrounds with genuine neutrinos
\item Zero b-tagged jets
\end{itemize}

\subsection{Scale Factor Derivation}

The ratio of observed to predicted events in the conversion control region is used to derive scale factors for the conversion background. Separate scale factors are derived for electron and muon conversions:

\textbf{Electron conversions}: The MC simulation is observed to underestimate the electron conversion rate, with data/MC ratios varying between 0.5--0.8 depending on the data-taking period. Scale factors are derived separately for each era to account for different detector conditions.

\textbf{Muon conversions} (internal): The MC simulation overestimates the muon internal conversion rate by approximately 20\% across all eras. A single correlated scale factor is applied, as this disagreement originates from generator-level modeling rather than detector effects.

\subsection{Systematic Uncertainties}

A systematic uncertainty of 20\% is assigned to the conversion background rate based on the agreement between data and prediction in the control region. For electron conversions, this uncertainty is treated as uncorrelated between eras due to its detector-dependent nature. For muon conversions, the uncertainty is fully correlated across eras.

\section{Prompt Backgrounds}
\label{sec:prompt}

Prompt backgrounds arise from SM processes that produce three or more genuine leptons in the final state. These irreducible backgrounds are estimated from MC simulation.

\subsection{Diboson Production}

The dominant prompt background is \PW\PZ production with both bosons decaying leptonically:
\begin{equation}
\PW\PZ \to \ell\nu\ell\ell
\end{equation}

This process has a similar final state to the signal (three leptons plus $\ptmiss$) but differs in the dimuon invariant mass distribution, which peaks at the \PZ mass rather than at $\mA$.

The \PZ\PZ $\to 4\ell$ process contributes when one lepton fails reconstruction or identification. These backgrounds are simulated at NLO accuracy with NNLO k-factors applied where available.

\subsection{$\ttbar$ + V/H Production}

Top pair production in association with vector bosons (\PW, \PZ) or Higgs bosons produces multilepton final states:
\begin{itemize}
\item $\ttbar\PZ \to \ell\nu\PQb\, \Pq\Pq\PQb\, \ell\ell$ and similar final states
\item $\ttbar\PW \to \ell\nu\PQb\, \Pq\Pq\PQb\, \ell\nu$
\item $\ttbar\PH$ with $\PH \to \PW\PW^*/\PZ\PZ^*/\tau\tau$
\end{itemize}

These processes share similar jet and b-jet multiplicity with the signal, making them challenging backgrounds in the signal region.

\subsection{Single Top + Z}

The $\cPqt\PZ\cPq$ process produces events with one top quark and a \PZ boson, contributing to the trilepton final state when the top decays leptonically.

\subsection{Rare Processes}

Several rare processes with small cross sections contribute at the percent level:
\begin{itemize}
\item Triboson production ($\PW\PW\PW$, $\PW\PW\PZ$, $\PW\PZ\PZ$, $\PZ\PZ\PZ$)
\item SM Higgs production through gluon fusion, vector boson fusion, and associated production with $\PH \to \PZ\PZ^* \to 4\ell$
\item $\ttbar\ttbar$ (four-top) production
\end{itemize}

A conservative 50\% uncertainty is assigned to these rare backgrounds.

\subsection{Normalization Uncertainties}

Theoretical uncertainties on the prompt background cross sections are taken from calculations in the literature:
\begin{itemize}
\item \PW\PZ: 12\% total (including scale variation, PDF, and radiative zero effects)
\item \PZ\PZ: 6.4\% total
\item $\ttbar\PZ$: 12\% total
\item $\ttbar\PW$: 13\% total
\item $\ttbar\PH$: 10\% total
\item $\cPqt\PZ\cPq$: 5\% total
\end{itemize}

\section{Background Validation}
\label{sec:bkg_validation}

\subsection{Z+Fake Control Region}

The nonprompt background estimation is validated in the \PZ+fake control region, which requires an on-\PZ dimuon pair, at least two jets, and zero b-tagged jets. This region is enriched in events with a genuine \PZ boson plus a nonprompt lepton.

Good agreement between data and prediction is observed, with the nonprompt contribution validated to within the assigned 30\% systematic uncertainty.

\subsection{Conversion Control Region}

The conversion background modeling is validated in the $\PZ\gamma$ control region. The scale factors derived from this region bring the MC prediction into agreement with data within the 20\% systematic uncertainty.

\subsection{WZ Control Region}

For Run~3 data, an additional \PW\PZ control region is used to validate the diboson background modeling following changes in the MC production. Good agreement between data and prediction confirms the validity of the diboson background estimate.

\section{Summary of Background Estimates}
\label{sec:bkg_summary}

Table~\ref{tab:bkg_summary} summarizes the expected background yields in the signal region for Run~2 data.

\begin{table}[htbp]
\centering
\caption{Expected background yields in the signal region for Run~2 data. Uncertainties include statistical uncertainties only.}
\label{tab:bkg_summary}
\begin{tabular}{lcc}
\hline\hline
Background & $\Pe\Pgm\Pgm$ & $\Pgm\Pgm\Pgm$ \\
\hline
Nonprompt & $496 \pm 15$ & $554 \pm 17$ \\
Diboson & $116 \pm 2$ & $154 \pm 3$ \\
$\ttbar$+X & $348 \pm 1$ & $383 \pm 1$ \\
Conversion & $61 \pm 9$ & $17 \pm 4$ \\
Others & $63 \pm 1$ & $85 \pm 1$ \\
\hline
Total & $1083 \pm 17$ & $1193 \pm 18$ \\
\hline\hline
\end{tabular}
\end{table}

The background composition is dominated by nonprompt leptons ($\sim$46\%) and $\ttbar$+X production ($\sim$32\%), followed by diboson production ($\sim$11--13\%) and smaller contributions from conversions and rare processes.

\chapter{Systematic Uncertainties}
\label{ch:systematics}

This chapter describes the systematic uncertainties considered in the analysis. The uncertainties are categorized into several groups: uncertainties related to data-driven background estimation methods, experimental uncertainties on MC corrections and luminosity, and theoretical uncertainties on signal and background modeling. A summary of all systematic uncertainty sources is provided at the end of the chapter.

\section{Overview}
\label{sec:syst_overview}

Systematic uncertainties are incorporated into the statistical analysis as nuisance parameters that modify either the normalization (rate uncertainties) or the shape (shape uncertainties) of the signal and background templates. These nuisance parameters are constrained by auxiliary measurements or theoretical calculations and are profiled in the maximum likelihood fit.

The correlation structure of systematic uncertainties across data-taking periods is carefully considered. Some uncertainties, such as theoretical cross sections, are fully correlated across all Run~2 data-taking periods (2016, 2017, 2018), while others, such as certain detector-related effects, are treated as uncorrelated due to different detector conditions in each period.

\section{Data-Driven Background Uncertainties}
\label{sec:syst_datadriven}

\subsection{Nonprompt Lepton Background}

The systematic uncertainty on the nonprompt lepton background estimated using the matrix method is assessed through several complementary studies. A flat uncertainty of 30\% is assigned to the normalization of this background, based on the following considerations:

\textbf{MC closure test}: The matrix method is applied to MC simulation with known composition to validate the method. The fake rate measured in simulation is applied to predict the nonprompt yield, which is then compared to the truth-level expectation. The observed non-closure is typically within 25--35\%.

\textbf{Fake rate variations}: Systematic variations of the measured fake rate are propagated through the background estimation. Sources of fake rate uncertainty include:
\begin{itemize}
\item Accuracy of MC simulation in subtracting prompt lepton contamination from the measurement region
\item Source dependence: the fake rate depends on the parent process (e.g., b-jets versus light-flavor jets)
\item Jet energy scale: the fake rate depends on the properties of the mother jet
\end{itemize}

The resulting variation in the fake rate is 10--30\%, and a similar magnitude of variation is observed in the predicted yields in the signal region.

The nonprompt background uncertainty is treated as uncorrelated between data-taking periods. This conservative treatment is motivated by the fact that the fake rate depends on the specific lepton identification criteria, which varied between data-taking eras for both tight and loose definitions. The correlation structure between eras is difficult to determine without using signal region data, and the impact of this nuisance parameter on the signal strength is at the order of 10\%, making detailed correlation studies of secondary importance.

\subsection{Conversion Background}

For the conversion background estimated from MC simulation with data-driven scale factors, an uncertainty of 20\% is assigned based on the agreement between data and prediction in the $\PZ\gamma$ control region.

The correlation structure differs between electron and muon conversions:

\textbf{Electron conversions}: The data/MC agreement varies between 0.5--0.8 depending on the data-taking period, reflecting detector-dependent effects. Since external conversions (the dominant source for electrons) occur in detector material and are sensitive to detector conditions, the uncertainty is treated as uncorrelated between data-taking eras.

\textbf{Muon conversions}: The MC simulation overestimates the internal conversion rate by approximately 20\% consistently across all eras. Since this disagreement originates from generator-level modeling rather than detector effects, the uncertainty is treated as fully correlated across all data-taking eras.

\section{Experimental Uncertainties}
\label{sec:syst_experimental}

Experimental uncertainties arise from corrections applied to MC simulation to account for differences between data and simulation. These uncertainties are considered for all processes estimated from MC simulation, including prompt backgrounds, signal, and conversion backgrounds.

\subsection{Integrated Luminosity}

The integrated luminosity measurement uncertainty is applied following recommendations from the CMS Luminosity Physics Object Group. The uncertainty has both uncorrelated and correlated components:

\textbf{Uncorrelated components}: Uncertainties of 1.0\%, 2.0\%, and 1.5\% are assigned to the integrated luminosity values in 2016, 2017, and 2018, respectively.

\textbf{Fully correlated components}: Uncertainties of 0.6\%, 0.9\%, and 2.0\% are assigned for 2016, 2017, and 2018, respectively, for sources that are fully correlated across all of Run~2.

\textbf{Partially correlated components}: Additional uncertainties of 0.6\% and 0.2\% are assigned to 2017 and 2018, respectively, for sources related to beam-beam interactions that affect only those data-taking eras.

\subsection{Pileup Reweighting}

The effect of pileup interactions is included through reweighting of the simulated pileup multiplicity distribution to match the expected distribution at a minimum bias cross section of 69.2~mb. An uncertainty of 4.6\% on this cross section is considered, and its effect is evaluated by reweighting to distributions at varied cross section values.

Since the minimum bias cross section is a physics quantity that should be identical across all data-taking periods, this uncertainty is treated as fully correlated between eras.

\subsection{L1 Prefiring}

During 2016 and 2017 data taking, a timing issue in the ECAL trigger primitives caused some events to be incorrectly assigned to the previous bunch crossing, resulting in event losses. Event weights are applied to correct for this L1 prefiring inefficiency.

A systematic uncertainty of 20\% on the prefiring probability per object is considered, along with the statistical uncertainty of the measurement. Since the prefiring effect is related to detector conditions that varied between data-taking periods, this uncertainty is treated as uncorrelated between 2016 and 2017 (the effect was resolved before 2018 data taking).

\subsection{Trigger Efficiency}

The trigger efficiency in simulation is corrected using scale factors measured in Drell-Yan events. The measurement uncertainty is observed to be negligible. Conservatively, a fully correlated uncertainty of 1\% is assigned across all data-taking periods.

\subsection{Lepton Reconstruction and Identification}

\subsubsection{Electron Reconstruction}

The electron reconstruction efficiency is corrected using scale factors provided by the CMS Electron and Photon Physics Object Group. The uncertainty in these corrections is propagated following the group recommendations and is treated as correlated between data-taking eras.

\subsubsection{Muon Reconstruction}

For the muon reconstruction efficiency in the considered $\pT$ and $\eta$ range, the efficiency is close to unity, and no statistically significant difference between data and simulation is observed. Therefore, no uncertainty is assigned for muon reconstruction efficiency.

\subsubsection{Lepton Energy Scale and Resolution}

The lepton energy scale corrections and their uncertainties follow the recommendations from the relevant physics object groups. For electrons, the uncertainty is propagated through the EGM corrections. For muons, the Rochester corrections are applied with their associated uncertainties.

The classifier output distributions are re-evaluated with lepton momenta varied by the correction uncertainties. The nuisance parameters are treated as fully correlated across data-taking eras following official recommendations.

\subsubsection{Lepton Identification}

The lepton identification efficiency is corrected using scale factors measured in Drell-Yan events via the tag-and-probe method. The uncertainty in the measured correction factors ranges from 1--4\% depending on the kinematic region. These uncertainties are treated as fully correlated across data-taking eras.

\subsection{Jet Energy Corrections}

\subsubsection{Jet Energy Scale}

The jet energy scale corrections from the JME Physics Object Group are applied to simulation. The impact of jet energy scale uncertainty is assessed by re-evaluating observables with the jet energy shifted by its uncertainty. This uncertainty is treated as correlated across data-taking periods.

\subsubsection{Jet Energy Resolution}

The jet energy resolution in simulation is smeared to match the resolution observed in data. The uncertainty on the smearing factors is propagated through the analysis. Since the resolution depends on detector conditions, this uncertainty is treated as uncorrelated between data-taking periods.

\subsection{B-Tagging Efficiency}

The b-tagging efficiency corrections and their uncertainties follow the recommendations from the BTV Physics Object Group. The uncertainty is evaluated from variations of the scale factors for each jet flavor:

\textbf{Heavy flavor (b, c)}: The uncertainty on b-tagging efficiency and c-to-b mistag efficiency is treated with 100\% correlation.

\textbf{Light flavor (u, d, s, g)}: The light jet mistag efficiency uncertainty is treated independently.

The uncertainty sources are decomposed into components related to detector conditions and measurement methods (correlated across eras) and components related to statistical uncertainty of the measurement samples (uncorrelated across eras). Both correlated and uncorrelated components are separately considered following the official recommendations.

\subsection{Unclustered Energy}

The uncertainty in the energy scale of particles not clustered into jets or identified as leptons (unclustered energy) affects the $\ptmiss$ calculation. The energy of unclustered particles is varied by their resolution in the tracker, HCAL, ECAL, and HF for charged hadrons, neutral hadrons, photons, and forward particles, respectively.

This uncertainty is treated as uncorrelated between data-taking periods following the POG recommendations.

\section{Theoretical Uncertainties}
\label{sec:syst_theory}

Theoretical uncertainties affect both the signal acceptance and the normalization of prompt backgrounds estimated from MC simulation. All theoretical uncertainties are treated as fully correlated across all data-taking eras of Run~2.

\subsection{Signal Acceptance Uncertainties}

\subsubsection{Parton Distribution Functions}

The uncertainty on signal acceptance arising from parton distribution function (PDF) uncertainties is evaluated following the PDF4LHC recommendations. The acceptance is calculated for each of the 100 replicas in the PDF set, and the deviations are added in quadrature to obtain the total PDF uncertainty.

\subsubsection{Renormalization and Factorization Scales}

The uncertainty from missing higher-order corrections is estimated by varying the renormalization ($\mu_R$) and factorization ($\mu_F$) scales by factors of 0.5 and 2 relative to the nominal values. The envelope of these variations, excluding anti-correlated variations ($\mu_R/\mu_F = 2/0.5$ or $0.5/2$), is taken as the systematic uncertainty.

Additionally, the scale of strong interaction at additional parton emission vertices is varied by changing the ``alpsfact'' parameter in MadGraph5\_aMC@NLO by factors of 0.5 and 2.

\subsubsection{Parton Shower Modeling}

The signal process involves multiple energetic partons from the cascade decays, and the ParticleNet classifier utilizes various aspects of the resulting jets. The uncertainty in parton shower modeling is evaluated by varying the scale of the strong coupling constant in the parton shower by factors of 0.5 and 2.

Separate variations are performed for Initial State Radiation (ISR) and Final State Radiation (FSR) following the recommendations from the TOP Physics Analysis Group. In these variations, the interaction scale at different splitting types ($\Pg \to \PQq\PAQq$, $\PQq \to \PQq\Pg$, $\Pg \to \Pg\Pg$, and X $\to$ X$\Pg$) are varied simultaneously.

Studies confirm that the parton shower uncertainty is subdominant in this analysis and is not expected to be strongly constrained by the data.

\subsection{Background Cross Section Uncertainties}

Theoretical uncertainties on the cross sections of prompt backgrounds are taken from precision calculations in the literature:

\subsubsection{Diboson Production}

For \PW\PZ production, the cross section uncertainty is derived from NNLO calculations:
\begin{itemize}
\item Scale variation ($Q^2$): $+4.9\%$, $-3.9\%$
\item PDF and $\alpha_S$: $\pm 1.5\%$
\item Radiative zero interference: $10.9\%$ (from the difference between NNLO and NLO calculations)
\end{itemize}

The radiative zero effect arises from destructive interference between diagrams $\PW \to \PW\PZ \to \ell\nu\PZ$ and $\PW \to \ell\nu \to \ell\nu\PZ$. This interference diminishes in higher-order QCD due to the presence of additional partons, an effect not captured by standard scale variations. The total \PW\PZ cross section uncertainty, obtained by adding all sources in quadrature, is 12\%.

For \PZ\PZ production:
\begin{itemize}
\item Scale variation ($Q^2$): $+4.3\%$, $-6.2\%$
\item PDF and $\alpha_S$: $\pm 1.7\%$
\end{itemize}
The total \PZ\PZ cross section uncertainty is 6.4\%.

\subsubsection{Top Quark Associated Production}

For $\ttbar$+V/H production, uncertainties are taken from the LHC Higgs Cross Section Working Group Yellow Report:

\textbf{$\ttbar\PZ$}:
\begin{itemize}
\item Scale variation: $+9.6\%$, $-11.2\%$
\item PDF: $\pm 2.8\%$
\item $\alpha_S$: $\pm 2.8\%$
\item Total: 12\%
\end{itemize}

\textbf{$\ttbar\PW$}:
\begin{itemize}
\item Scale variation: $+12.9\%$, $-11.5\%$
\item PDF: $\pm 2.0\%$
\item $\alpha_S$: $\pm 2.7\%$
\item Total: 13\%
\end{itemize}

\textbf{$\ttbar\PH$}:
\begin{itemize}
\item Scale variation: $+5.8\%$, $-9.2\%$
\item PDF: $\pm 3.0\%$
\item $\alpha_S$: $\pm 2.0\%$
\item Higgs branching fraction: 1.8\%
\item Total: 10\%
\end{itemize}

\subsubsection{Single Top + Z}

For $\cPqt\PZ\cPq$ production:
\begin{itemize}
\item Scale variation: $+1.1\%$, $-5.1\%$
\item PDF and $\alpha_S$: $\pm 1.0\%$
\item Total: 5\%
\end{itemize}

\subsubsection{Rare Processes}

A conservative uncertainty of 50\% is assigned to rare background processes, including triboson production (\PW\PW\PW, \PW\PW\PZ, \PW\PZ\PZ, \PZ\PZ\PZ), SM Higgs production via gluon fusion, vector boson fusion, and associated production.

For simplicity, asymmetric scale uncertainties are symmetrized using the larger of the upward and downward variations. This simplification has minimal impact on the analysis results due to the relatively small contribution of prompt backgrounds to the total background.

\section{Summary of Systematic Uncertainties}
\label{sec:syst_summary}

Table~\ref{tab:syst_summary} provides a comprehensive summary of all systematic uncertainty sources considered in this analysis, along with their correlation structure across data-taking eras and the processes to which they apply.

\begin{table}[htbp]
\centering
\caption{Summary of systematic uncertainty sources. The ``Correlation'' column indicates whether the uncertainty is correlated (yes) or uncorrelated (no) between data-taking eras. ``Yes/no'' indicates partial correlation where some components are correlated and others are not.}
\label{tab:syst_summary}
\begin{tabular}{lcc}
\hline\hline
Source & Correlation & Affected Processes \\
\hline
\multicolumn{3}{c}{\textit{Luminosity and Pileup}} \\
\hline
Integrated luminosity & yes/no & Prompt, signal, conversion \\
Pileup reweighting & yes & Prompt, signal, conversion \\
\hline
\multicolumn{3}{c}{\textit{Trigger and Prefiring}} \\
\hline
L1 prefiring rate & no & Prompt, signal, conversion \\
Trigger efficiency & yes & Prompt, signal, conversion \\
\hline
\multicolumn{3}{c}{\textit{Electron Uncertainties}} \\
\hline
Reconstruction efficiency & yes & Prompt, signal, conversion \\
Energy scale & yes & Prompt, signal, conversion \\
Energy resolution & yes & Prompt, signal, conversion \\
Identification efficiency & yes & Prompt, signal, conversion \\
\hline
\multicolumn{3}{c}{\textit{Muon Uncertainties}} \\
\hline
$\pT$ scale and resolution & yes & Prompt, signal, conversion \\
Identification efficiency & yes & Prompt, signal, conversion \\
\hline
\multicolumn{3}{c}{\textit{Jet and MET Uncertainties}} \\
\hline
Jet energy scale & yes & Prompt, signal, conversion \\
Jet energy resolution & no & Prompt, signal, conversion \\
Unclustered energy scale & no & Prompt, signal, conversion \\
B-tagging efficiency (b, c) & yes/no & Prompt, signal, conversion \\
B-tagging efficiency (light) & yes/no & Prompt, signal, conversion \\
\hline
\multicolumn{3}{c}{\textit{Data-Driven Background Uncertainties}} \\
\hline
Fake rate (nonprompt) & no & Nonprompt background \\
Conversion rate (electron) & no & Conversion (electron) \\
Conversion rate (muon) & yes & Conversion (muon) \\
\hline
\multicolumn{3}{c}{\textit{Theoretical Uncertainties}} \\
\hline
Cross section & yes & Prompt backgrounds \\
PDF (signal acceptance) & yes & Signal \\
QCD scales (signal) & yes & Signal \\
Parton shower modeling & yes & Signal \\
\hline\hline
\end{tabular}
\end{table}

\section{Impact of Systematic Uncertainties}
\label{sec:syst_impact}

The impact of systematic uncertainties on the final results depends on the signal mass hypothesis and the analysis channel. In general, the dominant systematic uncertainties are:

\textbf{Nonprompt background normalization} (30\%): This is one of the largest systematic uncertainties in the analysis due to the significant contribution of nonprompt leptons to the total background.

\textbf{Prompt background cross sections} (5--13\%): The theoretical uncertainties on diboson and $\ttbar$+X production cross sections contribute to the total uncertainty, particularly in regions where these backgrounds are significant.

\textbf{Conversion background} (20\%): The conversion background uncertainty is important in the $\Pe\Pgm\Pgm$ channel where electron conversions contribute more substantially.

\textbf{Experimental uncertainties}: Among the experimental uncertainties, the b-tagging efficiency and jet energy scale uncertainties have the largest impact due to the analysis requirements on jet multiplicity and b-tagging.

The statistical uncertainty from the limited size of the dataset remains significant across most of the explored mass range, particularly for signal hypotheses at the edges of the kinematically allowed region.


\chapter{Results and Interpretation}
\label{ch:results}

This chapter presents the statistical analysis procedure and results of the search for charged Higgs bosons. The expected and observed limits on the signal cross section are presented as functions of the charged Higgs and pseudoscalar mass hypotheses, and the results are interpreted in the context of the Two-Higgs-Doublet Model.

\section{Statistical Methodology}
\label{sec:stat_method}

The statistical analysis is performed using the asymptotic CL$_s$ method~\cite{Read:2002hq,Junk:1999kv} implemented in the Combine tool~\cite{CMS-NOTE-2011-005}. This modified frequentist approach provides robust limits by constructing confidence intervals that protect against excluding the signal hypothesis in cases of downward background fluctuations.

\subsection{Test Statistic}

The test statistic is based on the profile likelihood ratio:
\begin{equation}
q_\mu = -2 \ln \frac{\mathcal{L}(\mu, \hat{\hat{\boldsymbol{\theta}}})}{\mathcal{L}(\hat{\mu}, \hat{\boldsymbol{\theta}})}
\end{equation}
where $\mu$ is the signal strength parameter (the ratio of the observed cross section to the theoretical prediction), $\boldsymbol{\theta}$ represents the nuisance parameters, $\hat{\mu}$ and $\hat{\boldsymbol{\theta}}$ are the values that maximize the likelihood globally, and $\hat{\hat{\boldsymbol{\theta}}}$ are the values that maximize the likelihood for a fixed value of $\mu$.

The likelihood function is constructed from the product of Poisson probabilities over all bins in the fit:
\begin{equation}
\mathcal{L}(\mu, \boldsymbol{\theta}) = \prod_i \frac{(\mu s_i(\boldsymbol{\theta}) + b_i(\boldsymbol{\theta}))^{n_i}}{n_i!} e^{-(\mu s_i(\boldsymbol{\theta}) + b_i(\boldsymbol{\theta}))} \cdot \prod_j p_j(\theta_j)
\end{equation}
where $n_i$ is the observed count in bin $i$, $s_i$ and $b_i$ are the expected signal and background yields, and $p_j(\theta_j)$ are the constraint terms for the nuisance parameters.

\subsection{CL$_s$ Method}

Upper limits are set using the CL$_s$ criterion:
\begin{equation}
\text{CL}_s = \frac{\text{CL}_{s+b}}{\text{CL}_b} = \frac{P(q_\mu \geq q_\mu^{\text{obs}} | s+b)}{P(q_\mu \geq q_\mu^{\text{obs}} | b)}
\end{equation}

A signal hypothesis is excluded at the 95\% confidence level if CL$_s < 0.05$. The asymptotic approximation, valid for sufficiently large event counts, is used to compute the expected distributions of the test statistic, enabling efficient evaluation of limits across the full mass grid.

\section{Template Construction}
\label{sec:templates}

Signal extraction is performed through a binned maximum likelihood fit to the dimuon invariant mass distribution. The template construction procedure is optimized separately for the ``off-Z'' and ``on-Z'' mass regions.

\subsection{Mass Window Selection}

For each signal mass hypothesis $\mA$, events are selected within a mass window centered on the signal peak:
\begin{equation}
|m(\mu^+\mu^-) - \mA| < 5\sqrt{\Gamma_A^2 + \sigma_A^2}
\end{equation}
where $\Gamma_A$ is the natural width of the pseudoscalar \PA and $\sigma_A$ is the detector resolution. These parameters are extracted by fitting a Voigtian function (the convolution of a Breit-Wigner and a Gaussian) to the dimuon mass distribution in simulated signal events. The Gaussian component represents the experimental resolution, while the Breit-Wigner component represents the intrinsic width.

\subsection{Binning Strategy}

Following the recommendations of the Combine framework, signal and background templates are divided into 15 bins within the mass window. This binning preserves the distinctive shape information from the narrow signal resonance while ensuring sufficient statistical precision in each bin.

The autoMCStats method is employed with a threshold of 10 events to automatically account for per-bin statistical fluctuations in both signal and background templates. This approach introduces additional nuisance parameters for bins with limited MC statistics, ensuring robust treatment of template shape uncertainties.

\subsection{Dimuon Pair Selection in the $\Pgm\Pgm\Pgm$ Channel}

In the $\Pgm\Pgm\Pgm$ channel, two opposite-sign dimuon pairs can be formed, introducing ambiguity in identifying the pair originating from the \PA decay. Several selection criteria were studied using truth-matched simulation:

\textbf{Direct mass comparison}: Selecting the pair with the smaller or larger invariant mass based on the mass hypothesis.

\textbf{Gamma factor}: The Lorentz boost factor $\gamma = \pT(\mu\mu)/m(\mu\mu)$ of the dimuon system. For highly boosted \PA bosons (when $\mHc \gg \mA$), the correct pair tends to have a larger $\gamma$ factor.

\textbf{Transverse mass}: The transverse mass of same-sign muon pairs with the missing transverse momentum, exploiting the fact that the ``wrong'' muon originates from the \PW boson decay.

Studies show that direct mass comparison provides the best discrimination. For signal mass points with $\mHc > 110\GeV$ and $\mA > 60\GeV$, the pair with the larger mass is selected, achieving correct assignment in 50--70\% of events. For other mass hypotheses, the pair with the smaller mass is selected, with correct assignment rates of 60--95\%.

\subsection{ParticleNet-Enhanced Templates}

In the on-Z region ($60 < \mA < 120\GeV$), where the signal peak overlaps with abundant \PZ boson backgrounds, the ParticleNet classifier output is used to enhance discrimination. A cross-section-weighted likelihood ratio (LR) score is constructed:
\begin{equation}
\text{LR} = \frac{s_{\text{signal}}}{s_{\text{signal}} + w_{\text{nonprompt}} \times s_{\text{nonprompt}} + w_{\text{diboson}} \times s_{\text{diboson}} + w_{\ttbar+X} \times s_{\ttbar+X}}
\end{equation}
where $s_i$ are the classifier output scores for each class and $w_i$ are the relative yield weights calculated from the observed event yields in the analysis mass window.

Events are categorized based on the LR score, and a cut value is optimized for each mass hypothesis by maximizing the sensitivity metric:
\begin{equation}
\sqrt{2\left[(s+b)\ln\left(1 + \frac{s}{b}\right) - s\right]}
\end{equation}
where $s$ and $b$ denote the number of signal and background events within the selected window. This metric is appropriate for scenarios with low background rates.

The optimization yields sensitivity improvements of 25--50\% in the $\Pe\Pgm\Pgm$ channel and 10--20\% in the $\Pgm\Pgm\Pgm$ channel compared to using the dimuon mass distribution alone.

\section{Signal Branching Ratio Definition}
\label{sec:branching_ratio}

The results are presented in terms of limits on the signal branching ratio, defined as:
\begin{equation}
\mathcal{B}_{\text{sig}} = \mathcal{B}(\PQt \to \PW\PQb) \times \mathcal{B}(\PQt \to \PHc\PQb) \times \mathcal{B}(\PHc \to \PW\PA) \times \mathcal{B}(\PA \to \mu^+\mu^-)
\end{equation}

This definition accounts for the complete decay chain in the signal topology. The relationship between the signal branching ratio and the signal cross section is:
\begin{equation}
\sigma_{\text{sig}} = \sigma(\Pp\Pp \to \ttbar) \times \mathcal{B}_{\text{sig}} \times 2
\end{equation}
where the factor of 2 accounts for both charge conjugate configurations (either the top or antitop quark can decay via the charged Higgs).

\section{Expected Sensitivity}
\label{sec:expected}

Before examining the data, the expected sensitivity of the analysis is evaluated using the Asimov dataset, which represents the expected event yields under the background-only hypothesis. The expected upper limits on $\mathcal{B}_{\text{sig}}$ at 95\% confidence level are computed across the full grid of mass hypotheses.

\subsection{Mass Dependence}

The expected sensitivity varies significantly across the $(\mHc, \mA)$ mass plane due to several factors:

\textbf{Signal acceptance}: The signal acceptance depends on the kinematics of the decay products, which varies with the mass splitting between $\mHc$ and $\mA$. Larger mass splittings generally lead to more boosted decay products and higher acceptance.

\textbf{Background levels}: The background composition and rate depend on the dimuon mass region. In the off-Z regions, backgrounds are dominated by nonprompt leptons and continuum processes, while in the on-Z region, resonant \PZ backgrounds become significant.

\textbf{Off-shell effects}: For mass hypotheses where the $\PHc \to \PW\PA$ decay proceeds through an off-shell \PW boson ($\mA > \mHc - m_\PW$), the signal acceptance is reduced due to the suppressed decay rate, partially compensated by softer kinematic distributions.

\subsection{Channel Comparison}

The $\Pe\Pgm\Pgm$ and $\Pgm\Pgm\Pgm$ channels provide complementary sensitivity across the mass plane:

The $\Pgm\Pgm\Pgm$ channel benefits from higher muon reconstruction efficiency and lower conversion backgrounds, providing better sensitivity for most mass hypotheses.

The $\Pe\Pgm\Pgm$ channel provides additional statistical power and cross-checks, with slightly different systematic uncertainty correlations.

The combination of both channels improves the overall sensitivity by approximately 20--40\% compared to the individual channels.

\section{Observed Results}
\label{sec:observed}

The analysis is performed as a blind search, with the signal region examined only after all selection criteria, background estimation methods, and systematic uncertainties are finalized. The observed data in the signal region are compared to the background predictions, and limits are extracted using the statistical methodology described above.

\subsection{Data-Background Comparison}

Good agreement is observed between the data and the background predictions in the signal region. No statistically significant excess above the Standard Model expectation is observed for any of the tested mass hypotheses.

The largest local deviations from the background-only hypothesis are consistent with statistical fluctuations expected from the large number of mass hypotheses tested (look-elsewhere effect).

\subsection{Upper Limits}

In the absence of a significant signal, upper limits at 95\% confidence level are set on the signal branching ratio $\mathcal{B}_{\text{sig}}$ for each $(\mHc, \mA)$ mass hypothesis. The observed limits are generally consistent with the expected limits within the $\pm 1\sigma$ and $\pm 2\sigma$ uncertainty bands.

The strongest limits are obtained in the off-Z mass regions where backgrounds are lowest. In the on-Z region, the ParticleNet classifier recovers sensitivity that would otherwise be significantly degraded by the resonant \PZ background.

\section{Interpretation in the 2HDM}
\label{sec:interpretation}

The results can be interpreted in the context of specific Two-Higgs-Doublet Model scenarios by combining the limits on $\mathcal{B}_{\text{sig}}$ with theoretical predictions for the relevant branching ratios.

\subsection{Type-I 2HDM at Large $\tanb$}

In a Type-I 2HDM with large $\tanb$, the decay $\PHc \to \PW\PA$ can become dominant as the fermionic decay modes are suppressed. The branching ratio $\mathcal{B}(\PHc \to \PW\PA)$ approaches unity in this regime, making this search channel particularly sensitive.

The branching ratio $\mathcal{B}(\PA \to \mu^+\mu^-)$ depends on the pseudoscalar mass and the 2HDM type:
\begin{equation}
\mathcal{B}(\PA \to \mu^+\mu^-) \propto \frac{m_\mu^2 \tan^2\beta}{\sum_f N_c^f m_f^2 \xi_f^2}
\end{equation}
where the sum runs over all kinematically accessible fermions, $N_c^f$ is the color factor, and $\xi_f$ depends on the 2HDM type.

For $\mA < 2m_b \approx 10\GeV$, the dimuon channel provides the dominant sensitivity. For larger $\mA$, the dimuon branching ratio decreases but remains non-negligible.

\subsection{Parameter Space Constraints}

The upper limits on $\mathcal{B}_{\text{sig}}$ can be translated into constraints on the 2HDM parameter space, particularly on:

The branching ratio $\mathcal{B}(\PQt \to \PHc\PQb)$, which is related to $\tanb$ and the charged Higgs mass.

The combination of branching ratios $\mathcal{B}(\PHc \to \PW\PA) \times \mathcal{B}(\PA \to \mu^+\mu^-)$, which depends on the 2HDM type and the masses of the additional Higgs bosons.

These constraints are complementary to those from other charged Higgs searches (e.g., $\PHc \to \tau\nu$, $\PHc \to \PQt\PQb$) and from neutral Higgs searches.

\section{Comparison with Previous Results}
\label{sec:comparison}

This analysis represents a significant improvement over previous searches for charged Higgs bosons in the $\PHc \to \PW\PA \to \PW\mu^+\mu^-$ channel:

\textbf{Extended mass range}: Previous CMS searches were limited to on-shell $\PHc \to \PW\PA$ decays, requiring $\mA < \mHc - m_\PW$. This analysis extends coverage to the off-shell region, significantly expanding the accessible parameter space.

\textbf{Full Run~2 dataset}: The analysis uses the complete Run~2 dataset corresponding to 138\fbinv, compared to 36\fbinv in the previous CMS publication using 2016 data.

\textbf{Machine learning enhancement}: The ParticleNet-based classifier provides substantially improved sensitivity in the challenging on-Z region where traditional cut-based analyses suffer from large \PZ boson backgrounds.

\textbf{Improved background estimation}: The data-driven methods for nonprompt and conversion backgrounds have been refined with additional control regions and systematic studies.

\section{Summary of Results}
\label{sec:results_summary}

No significant excess above the Standard Model prediction is observed in the search for charged Higgs bosons produced in top quark decays with subsequent decays $\PHc \to \PW\PA \to \PW\mu^+\mu^-$. Upper limits at 95\% confidence level are set on the signal branching ratio across a two-dimensional grid of charged Higgs masses (70--160\GeV) and pseudoscalar masses (15\GeV to $\mHc - 5\GeV$).

This represents the first search for the decay $\PHc \to \PW\PA$ including off-shell effects, providing comprehensive coverage of the 2HDM parameter space accessible at the LHC for light charged Higgs bosons.


\chapter{Summary and Conclusions}
\label{ch:conclusion}

\section{Summary}

This thesis has presented a search for light charged Higgs bosons produced in top quark decays at the Large Hadron Collider. The search targets the decay chain $\PQt \to \PHc\PQb$ followed by $\PHc \to \PW\PA$ and $\PA \to \mu^+\mu^-$, where the charged Higgs boson (\PHc) and CP-odd pseudoscalar (\PA) are predicted by Two-Higgs-Doublet Model extensions of the Standard Model.

The analysis uses proton-proton collision data collected by the CMS experiment during LHC Run~2 (2016--2018) at a center-of-mass energy of $\sqrt{s} = 13\TeV$, corresponding to an integrated luminosity of 138\fbinv. Events are selected in two final state channels: the $\Pe\Pgm\Pgm$ channel containing one electron and two muons, and the $\Pgm\Pgm\Pgm$ channel containing three muons. The selection requires at least two jets with at least one identified as originating from a bottom quark, consistent with the $\ttbar$ production topology.

A key advancement of this analysis compared to previous searches is the inclusion of off-shell $\PHc \to \PW\PA$ decays, extending coverage to mass configurations where the decay proceeds through a virtual \PW boson. This significantly expands the accessible region of the 2HDM parameter space. The search covers charged Higgs boson masses from 70 to 160\GeV and pseudoscalar masses from 15\GeV up to $\mHc - 5\GeV$.

The dominant backgrounds arise from nonprompt leptons (from jet misidentification or heavy-flavor decays), photon conversions, and irreducible prompt Standard Model processes including diboson production and $\ttbar$+V/H. The nonprompt lepton background is estimated using a data-driven matrix method, while conversion backgrounds are constrained using control regions enriched in $\PZ\gamma$ events. Prompt backgrounds are estimated from Monte Carlo simulation with appropriate theoretical and experimental systematic uncertainties.

Machine learning techniques play a central role in enhancing the sensitivity of the search. A ParticleNet-based graph neural network classifier is trained to discriminate signal events from \PZ boson associated backgrounds, which become particularly challenging in the ``on-Z'' mass region where the signal dimuon mass peak overlaps with the \PZ resonance. The classifier exploits the distinct kinematic and topological features of the signal process, yielding sensitivity improvements of 25--50\% in the $\Pe\Pgm\Pgm$ channel and 10--20\% in the $\Pgm\Pgm\Pgm$ channel.

Signal extraction is performed through a binned maximum likelihood fit to the dimuon invariant mass distribution using the asymptotic CL$_s$ method. The analysis scans over a two-dimensional grid of mass hypotheses, providing comprehensive coverage of the kinematically accessible parameter space.

\section{Results}

No statistically significant excess above the Standard Model prediction is observed. The data are consistent with the background-only hypothesis across all tested mass configurations. Upper limits at 95\% confidence level are set on the signal branching ratio:
\begin{equation}
\mathcal{B}_{\text{sig}} = \mathcal{B}(\PQt \to \PW\PQb) \times \mathcal{B}(\PQt \to \PHc\PQb) \times \mathcal{B}(\PHc \to \PW\PA) \times \mathcal{B}(\PA \to \mu^+\mu^-)
\end{equation}

These limits provide constraints on Two-Higgs-Doublet Model scenarios, particularly the Type-I 2HDM at large $\tanb$ where the bosonic decay mode $\PHc \to \PW\PA$ can become dominant.

\section{Significance and Impact}

This analysis represents the first search for the decay $\PHc \to \PW\PA$ including off-shell effects. The inclusion of off-shell decays significantly extends the physics reach beyond previous searches that were restricted to on-shell configurations. Combined with the full Run~2 dataset and advanced machine learning techniques, this provides the most comprehensive exploration of this charged Higgs decay channel to date.

The results complement other charged Higgs searches at the LHC that target fermionic decay modes ($\PHc \to \tau\nu$, $\PHc \to \PQt\PQb$, $\PHc \to \PQc\PQs$). Together, these searches probe different regions of the 2HDM parameter space and different mass ranges for the additional Higgs bosons.

\section{Future Prospects}

Several avenues exist for improving the sensitivity of this search in the future:

\textbf{Larger datasets}: The High-Luminosity LHC (HL-LHC) is expected to deliver an integrated luminosity of approximately 3000\fbinv, representing a factor of 20 increase compared to the current analysis. This dramatic increase in statistics will substantially improve the sensitivity to rare processes.

\textbf{Improved machine learning}: The field of machine learning continues to advance rapidly, with new architectures and training techniques offering potential improvements in signal-background discrimination. The graph neural network approach used in this analysis could be further refined or replaced with more sophisticated methods.

\textbf{Additional decay channels}: While this analysis focuses on the $\PA \to \mu^+\mu^-$ decay, other pseudoscalar decay modes such as $\PA \to \tau^+\tau^-$ or $\PA \to \PQb\PAQb$ could provide complementary sensitivity in different mass regions.

\textbf{Combination with other searches}: The combination of results from multiple charged and neutral Higgs searches at ATLAS and CMS will provide the most stringent constraints on extended Higgs sector models.

\section{Conclusions}

The search for new physics beyond the Standard Model remains one of the primary objectives of the LHC physics program. Extended Higgs sectors, such as the Two-Higgs-Doublet Model, represent well-motivated theoretical extensions that predict additional scalar particles potentially accessible at current collider energies.

This thesis has demonstrated that precision searches for charged Higgs bosons in complex final states are feasible with the Run~2 dataset, and that machine learning techniques can significantly enhance sensitivity in challenging background environments. While no evidence for new physics has been observed in this analysis, the comprehensive exploration of the parameter space and the development of advanced analysis techniques provide a foundation for future discoveries.

The absence of a signal at the current sensitivity level places meaningful constraints on 2HDM scenarios and motivates continued searches with larger datasets and improved techniques. The discovery of an extended Higgs sector would fundamentally transform our understanding of electroweak symmetry breaking and provide crucial guidance for physics beyond the Standard Model.



\printbibliography

\appendix
\input{Sections/appendices/First.tex}

\keywordalt{하전 힉스 보존, 이중 힉스 이중항 모형, CMS, LHC, 표준모형 너머 물리}
\begin{abstractalt}
이 논문은 LHC의 CMS 검출기에서 수집한 양성자-양성자 충돌 데이터를 사용하여 희귀 탑 쿼크 붕괴에서 가벼운 하전 힉스 보존 탐색을 제시한다. 데이터는 13~\TeV에서 138~\fbinv, 13.6~\TeV에서 62~\fbinv의 적분 광도량에 해당한다. 탐색은 붕괴 연쇄 $\PQt \to \PHc\PQb$와 이어지는 $\PHc \to \PWp\PA$, $\PA \to \Pgmp\Pgmm$를 대상으로 하며, 여기서 $\PHc$와 $\PA$는 이중 힉스 이중항 모형(2HDM)에서 예측되는 하전 힉스 보존과 CP-홀수 힉스 보존이다. 최종 상태는 세 개의 하전 경입자($\Pe\Pgm\Pgm$ 또는 $\Pgm\Pgm\Pgm$), 결손 횡운동량, 그리고 b-태그된 젯 하나를 포함한 최소 두 개의 젯으로 구성된다. 탐색은 비껍질 $\PHc \to \PWp\PA$ 붕괴를 포함하여 하전 힉스 보존 질량($\mHc$) 70--160\GeV와 CP-홀수 힉스 보존 질량($\mA$) 15\GeV에서 $\mHc - 5\GeV$까지의 범위를 다룬다. ParticleNet 기반 분류기가 \PZ 보존 배경으로부터 신호를 구별하는 데 사용되며, 쌍뮤온 불변 질량 분포에 대한 구간화된 최대 우도 맞춤을 통해 신호가 추출된다. 표준모형 예측을 초과하는 통계적으로 유의미한 초과가 관찰되지 않았으며, 신호 분기비에 대해 95\% 신뢰 수준에서 상한이 설정되었다. 이 결과는 비껍질 효과를 포함한 $\PHc \to \PW\PA$ 붕괴에 대한 최초의 탐색을 나타낸다.
\end{abstractalt}

\acknowledgement
Thanks.

\end{document}

