\chapter{Introduction}
\label{ch:introduction}

The Standard Model (SM) of particle physics represents one of the most successful scientific theories ever developed, providing an elegant mathematical framework that describes the fundamental constituents of matter and their interactions. Since its formulation in the 1960s and 1970s, the SM has withstood rigorous experimental scrutiny, culminating in the historic discovery of the Higgs boson by the ATLAS and CMS collaborations at the Large Hadron Collider (LHC) in 2012~\cite{HiggsDiscoveryATLAS,HiggsDiscoveryCMS}. This discovery confirmed the Brout-Englert-Higgs mechanism as the source of electroweak symmetry breaking and provided a compelling explanation for the origin of mass for fundamental particles.

Despite its remarkable success, the SM is widely regarded as an incomplete description of nature. Several observed phenomena remain unexplained within its framework: the existence and nature of dark matter, the matter-antimatter asymmetry in the universe, the hierarchy problem associated with the Higgs boson mass, and the origin of neutrino masses. These unresolved questions strongly suggest the existence of physics beyond the Standard Model (BSM), motivating extensive theoretical and experimental efforts to uncover new fundamental principles.

Among the various BSM extensions, models with extended Higgs sectors have attracted considerable attention. The Two-Higgs-Doublet Model (2HDM) represents one of the simplest and most well-motivated extensions, introducing a second Higgs doublet to the SM scalar sector~\cite{Branco:2011iw}. This extension is particularly compelling because it arises naturally in several theoretical frameworks, including supersymmetric theories~\cite{Gunion:1989we}, models addressing the hierarchy problem~\cite{Martin:1997ns}, and the Peccei-Quinn mechanism proposed to solve the strong CP problem~\cite{Peccei:1977hh,Peccei:1977ur,Peccei:2006as}.

The 2HDM predicts a rich phenomenology with five physical Higgs bosons: two CP-even neutral scalars (\Ph and \PH), one CP-odd neutral pseudoscalar (\PA), and a pair of charged Higgs bosons (\PHpm). The discovery of any of these additional Higgs bosons would constitute unambiguous evidence for new physics and provide crucial insights into the structure of electroweak symmetry breaking beyond the SM.

\section{Motivation for Charged Higgs Searches}

The charged Higgs boson (\PHpm) is a particularly distinctive signature of extended Higgs sectors, as its existence would definitively establish physics beyond the SM. Unlike neutral scalars, which could potentially be confused with the SM Higgs boson, charged Higgs bosons have no SM counterpart, making their detection a clear discovery channel for new physics.

Searches for charged Higgs bosons at hadron colliders are typically categorized based on the relationship between the charged Higgs mass (\mHc) and the top quark mass (\mt). For $\mHc < \mt$, the dominant production mechanism is through top quark decays: $\PQt \to \PHc\PQb$. In this scenario, charged Higgs bosons are produced in abundance from $\ttbar$ pair production, one of the most copious processes at the LHC. For $\mHc > \mt$, production occurs primarily through gluon fusion or in association with a top quark.

Previous searches have predominantly focused on fermionic decay modes of the charged Higgs boson, such as $\PHpm \to \PQt\PAQb$, $\PHpm \to \PGtp\PGn$, and $\PHpm \to \PQc\PAQs$. However, bosonic decay modes, particularly $\PHpm \to \PWpm\PA$, can become dominant in specific regions of the 2HDM parameter space. In a Type-I 2HDM with large values of \tanb (the ratio of vacuum expectation values of the two Higgs doublets), the decay $\PHpm \to \PWpm\PA$ can achieve branching fractions approaching unity, as the fermionic decay modes become suppressed.

\section{The Search Channel: $\PHpm \to \PWpm\PA \to \PWpm\PGmp\PGmm$}

This thesis presents a search for light charged Higgs bosons produced in top quark decays, with subsequent decay through the chain $\PHpm \to \PWpm\PA$ followed by $\PA \to \PGmp\PGmm$. The complete signal process can be written as:
\begin{equation}
\Pp\Pp \to \ttbar \to (\PHpm\PQb)(\PWmp\PQb) \to (\PWpm\PA\PQb)(\PWmp\PQb) \to (\PWpm\PGmp\PGmm\PQb)(\PWmp\PQb)
\end{equation}
where one top quark decays via the charged Higgs while the other undergoes the SM decay $\PQt \to \PWp\PQb$.

The choice of the dimuon decay channel for the pseudoscalar \PA offers several experimental advantages. The Compact Muon Solenoid (CMS) detector, as its name suggests, provides exceptional muon reconstruction capabilities with excellent momentum resolution. This enables precise reconstruction of the dimuon invariant mass, which is crucial for identifying narrow resonance structures over the continuum background. The analysis can therefore scan fine mass bins across a wide range of \mA values, maximizing sensitivity to potential signal contributions.

The final state topology consists of three charged leptons (from the two \PW bosons and the dimuon pair from \PA decay), missing transverse momentum from neutrinos, and at least two jets from bottom quark hadronization. Events are categorized into two channels based on the lepton flavors: the $\Pe\Pgm\Pgm$ channel (one electron, two muons) and the $\Pgm\Pgm\Pgm$ channel (three muons).

\section{Extension to Off-Shell Decays}

A significant advancement of this analysis compared to previous searches is the inclusion of off-shell $\PHpm \to \PWpm\PA$ decays. Previous CMS and ATLAS searches~\cite{CMSRun2ChargedHiggs,ATLASRun2ChargedHiggs} restricted their scope to on-shell decays, requiring $\mA < \mHc - m_{\PW}$. This constraint excluded a substantial portion of the 2HDM parameter space where the decay proceeds through off-shell \PW bosons.

Calculations using the 2HDMC program~\cite{2HDMC} demonstrate that the targeted decay chain maintains significant branching ratios even in the off-shell region, extending the physics reach of the search. The inclusion of off-shell decays, however, introduces additional challenges, particularly from backgrounds involving \PZ boson decays to muon pairs. To address this, the analysis employs advanced machine learning techniques based on graph neural networks for effective background discrimination.

\section{Analysis Strategy Overview}

The search employs a combination of data-driven methods and Monte Carlo (MC) simulation to estimate the various background contributions. The dominant backgrounds include:
\begin{itemize}
\item Nonprompt leptons from jets misidentified as leptons or from heavy-flavor hadron decays
\item Photon conversions, both external (in detector material) and internal (Dalitz decays)
\item Irreducible prompt backgrounds from diboson (\PW\PZ, \PZ\PZ) and $\ttbar$+V production
\end{itemize}

The nonprompt lepton background is estimated using the matrix method (fake rate method), which extrapolates from control regions enriched in misidentified leptons to the signal region. Conversion backgrounds are constrained using scale factors derived from $\PZ\Pgg$ control regions. Prompt backgrounds are estimated from MC simulation with appropriate theoretical and experimental systematic uncertainties.

Signal extraction is performed using a binned maximum likelihood fit to the dimuon invariant mass distribution. The analysis scans over a two-dimensional grid of charged Higgs masses (70--160\GeV) and pseudoscalar masses (15\GeV to $\mHc-5\GeV$), providing comprehensive coverage of the accessible parameter space.

Machine learning techniques play a central role in enhancing the sensitivity of the search. A ParticleNet-based~\cite{ParticleNet} classifier is trained to discriminate signal events from the dominant \PZ boson backgrounds, exploiting the distinct kinematic and topological features of the signal process. This approach yields significant improvements in sensitivity, particularly in the challenging off-shell decay regions.

\section{Dataset and Experimental Context}

The analysis utilizes proton-proton collision data collected by the CMS experiment at the LHC during Run~2 (2016--2018) at a center-of-mass energy of $\sqrt{s} = 13\TeV$, corresponding to an integrated luminosity of 138\fbinv. The unprecedented size of this dataset, combined with the high $\ttbar$ production cross section at the LHC, provides substantial statistical power for the search.

The CMS detector, with its precise tracking system, high-resolution electromagnetic calorimeter, and powerful muon detection capabilities, is ideally suited for this analysis. The combination of excellent lepton identification, efficient b-jet tagging, and accurate missing transverse momentum reconstruction enables effective selection of the signal topology while suppressing backgrounds.

\section{Thesis Structure}

This thesis is organized as follows. Chapter~\ref{ch:theory} provides the theoretical foundation, reviewing the Standard Model, the Two-Higgs-Doublet Model, and the phenomenology of charged Higgs bosons relevant to this search. Chapter~\ref{ch:detector} describes the LHC accelerator complex and the CMS detector, with emphasis on the subsystems most relevant for this analysis.

Chapter~\ref{ch:datasets} details the datasets and Monte Carlo samples used, including the signal modeling approach. Chapter~\ref{ch:objects} discusses the reconstruction and identification of physics objects: electrons, muons, jets, b-tagged jets, and missing transverse momentum. The event selection criteria and signal extraction strategy are presented in Chapter~\ref{ch:selection}.

Chapter~\ref{ch:background} describes the methods used for background estimation, including the matrix method for nonprompt leptons and control region techniques for conversion backgrounds. Chapter~\ref{ch:systematics} summarizes the systematic uncertainties considered in the analysis.

The results of the search, including observed and expected limits on the signal cross section, are presented in Chapter~\ref{ch:results}. Finally, Chapter~\ref{ch:conclusion} provides conclusions and discusses prospects for future searches.
