\chapter{Introduction}
\label{ch:Introduction}

The Standard Model (SM) of particle physics stands as one of the most successful scientific theories ever developed. Built upon the gauge symmetry group $\SU(3)_C \times \SU(2)_L \times \U(1)_Y$, it provides a unified description of the strong, weak, and electromagnetic interactions among fundamental particles. Since its formulation in the 1960s and 1970s~\cite{Glashow:1961tr,Weinberg:1967tq,Salam:1968rm}, the SM has demonstrated remarkable predictive power, with theoretical predictions consistently confirmed by precision experiments. The discovery of the Higgs boson by the ATLAS and CMS collaborations at the Large Hadron Collider (LHC) in 2012~\cite{ATLAS:2012yda,CMS:2012qbp} marked the completion of the SM particle spectrum. This discovery confirmed the Brout-Englert-Higgs mechanism~\cite{Englert:1964et,Higgs:1964pj} as the origin of electroweak symmetry breaking, providing a compelling explanation for how gauge bosons and fermions acquire mass through their interactions with the Higgs field.

Despite this remarkable success, the SM leaves several fundamental questions unanswered. The \textit{hierarchy problem} asks why the Higgs boson mass ($\sim 125\GeV$) is so much lighter than the Planck scale ($\sim 10^{19}\GeV$). In the SM, quantum corrections from virtual particles would naturally drive the Higgs mass toward the highest energy scale in the theory, requiring an extraordinary fine-tuning to maintain its observed value. Supersymmetric extensions~\cite{Martin:1997ns} offer an elegant solution through systematic cancellations between SM particles and their superpartners, and these theories necessarily require a two-Higgs-doublet structure. The \textit{origin of neutrino masses} poses another fundamental challenge. Neutrino oscillation experiments have conclusively demonstrated that neutrinos possess small but nonzero masses~\cite{SuperK:1998kpq,SNO:2002tuh}, yet the SM contains only left-handed neutrinos and therefore cannot generate their masses through the standard Higgs mechanism, which requires both left- and right-handed fields. The seesaw mechanism~\cite{Minkowski:1977sc,Mohapatra:1979ia} provides an elegant explanation for the smallness of neutrino masses, and its implementations often involve extended scalar sectors such as Higgs triplets or additional doublets.

The \textit{strong CP problem}~\cite{Kim:2008hd} concerns why QCD preserves CP symmetry to such extraordinary precision. The QCD Lagrangian permits a CP-violating term, yet experimental bounds on the neutron electric dipole moment~\cite{Abel:2020pzs} constrain the relevant parameter to $|\bar{\theta}| < 10^{-10}$, an unnaturally small value that demands explanation. The Peccei-Quinn mechanism~\cite{Peccei:1977hh,Peccei:1977ur} addresses this puzzle by introducing a new global $\U(1)_{PQ}$ symmetry, and many implementations of this mechanism require an extended Higgs sector. The \textit{matter-antimatter asymmetry} of the universe presents a cosmological puzzle: the observed dominance of matter over antimatter requires sources of CP violation beyond what the SM provides through the CKM matrix~\cite{Sakharov:1967dj}. Extended Higgs models can supply the necessary additional CP violation while also enabling a strongly first-order electroweak phase transition, both essential ingredients for successful electroweak baryogenesis~\cite{Morrissey:2012db}. Finally, \textit{dark matter}, which constitutes approximately 27\% of the universe's energy density according to cosmological observations~\cite{Planck:2018vyg}, has no viable candidate within the SM. Extended scalar sectors can naturally accommodate dark matter through mechanisms such as the inert doublet model~\cite{Deshpande:1977rw}, where a discrete $Z_2$ symmetry stabilizes the lightest neutral scalar, making it a weakly interacting massive particle (WIMP) candidate.

These diverse anomalies, spanning mass generation, CP violation, and cosmology, can all be addressed by extending the scalar sector. The Two-Higgs-Doublet Model (2HDM) represents the minimal such extension, introducing only one additional Higgs doublet to the SM~\cite{Branco:2011iw}. This economy, where a single structural modification can provide mechanisms for multiple fundamental puzzles, makes the 2HDM a particularly compelling framework for physics beyond the Standard Model (BSM). The 2HDM predicts five physical Higgs bosons: two CP-even neutral scalars (\Ph and \PH), one CP-odd pseudoscalar (\PA), and a pair of charged Higgs bosons (\PHpm). The discovery of any of these additional states would constitute unambiguous evidence for new physics and provide crucial insights into the structure of electroweak symmetry breaking.

The Large Hadron Collider~\cite{Evans:2008zzb} at CERN is the premier facility for directly testing extended Higgs sectors. Operating at center-of-mass energies of 13--14\TeV, it can produce BSM particles that would be inaccessible at lower-energy machines, enabling direct searches for the additional Higgs bosons predicted by models like the 2HDM. The high luminosity achieved by the LHC is equally essential: rare processes such as charged Higgs production through top quark decays require enormous datasets to accumulate sufficient signal events above the SM backgrounds. With the High-Luminosity LHC era beginning in 2030~\cite{HLLHC:TDR}, the program will ultimately deliver over 3000~\fbinv, providing unprecedented sensitivity to BSM physics.

This thesis presents a search for light charged Higgs bosons produced in top quark decays, targeting the decay chain $\PHpm \to \PWpm\PA$ followed by $\PA \to \PGmp\PGmm$. The search utilizes proton-proton collision data collected by the CMS experiment~\cite{CMS:2008xjf}, corresponding to integrated luminosities of 138~\fbinv at $\sqrt{s} = 13\TeV$ (Run~2, 2016--2018) and 62~\fbinv at $\sqrt{s} = 13.6\TeV$ (Run~3, 2022--2023). A key advancement of this analysis is the inclusion of off-shell $\PHpm \to \PWpm\PA$ decays, extending the physics reach beyond previous searches that were restricted to the on-shell regime. The thesis is organized as follows: Chapters~\ref{ch:theory}--\ref{ch:detector} establish the theoretical and experimental foundations; Chapters~\ref{ch:datasets}--\ref{ch:objects} describe the data samples and physics object reconstruction; Chapters~\ref{ch:selection}--\ref{ch:systematics} detail the event selection, background estimation methods, and systematic uncertainties; and Chapters~\ref{ch:results}--\ref{ch:conclusion} present the results and conclusions.
