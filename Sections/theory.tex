\chapter{Theoretical Background}
\label{ch:theory}

This chapter provides the theoretical foundation for the search presented in this thesis. We begin with an overview of the Standard Model of particle physics, followed by a discussion of its limitations and the motivation for extended Higgs sectors. The Two-Higgs-Doublet Model is then introduced, with particular emphasis on the phenomenology of charged Higgs bosons relevant to this analysis.

\section{The Standard Model of Particle Physics}
\label{sec:theory_sm}

The Standard Model (SM) is a quantum field theory that describes the fundamental particles and their interactions through three of the four known fundamental forces: the electromagnetic, weak, and strong interactions. Gravity, while well-described by general relativity at macroscopic scales, is not incorporated into the SM framework.

\subsection{Spacetime Symmetries and Particle Classification}
\label{sec:spacetime_symmetries}

The construction of the Standard Model begins not with a list of particles, but with a principle: the laws of physics must take the same form regardless of where or when an experiment is performed, or which inertial frame the observer occupies. As Weinberg emphasized, the result of an experiment on Earth must be predicted by the same laws that would apply on the Moon---not because of any dynamical mechanism, but because the laws themselves respect the symmetries of spacetime~\cite{Weinberg:1995mt}.

This requirement of invariance under spacetime transformations is encoded in the Poincar\'{e} group---the group of all isometries of Minkowski spacetime, comprising translations, rotations, and Lorentz boosts. By Noether's theorem, each continuous symmetry implies a conserved quantity: translational invariance in space and time yields conservation of momentum and energy, while rotational invariance yields conservation of angular momentum.

The most profound consequence of these symmetries for particle physics is Wigner's classification~\cite{Wigner:1939cj}: elementary particles correspond to irreducible unitary representations of the Poincar\'{e} group, labeled by two invariants---mass $m \geq 0$ and spin $s = 0, \frac{1}{2}, 1, \frac{3}{2}, \ldots$. This is not merely a convenient labeling scheme; it is the \emph{definition} of what constitutes an elementary particle.

The structure of the Lorentz group further constrains what types of fields can describe particles. The Lie algebra of the proper orthochronous Lorentz group $\mathrm{SO}^+(3,1)$ admits, upon complexification, a decomposition into two copies of the $\mathfrak{su}(2)$ algebra:
\begin{equation}
\mathfrak{so}(3,1)_{\mathbb{C}} \cong \mathfrak{su}(2) \oplus \mathfrak{su}(2)
\end{equation}
Finite-dimensional representations are therefore labeled by a pair $(j_L, j_R)$, where each index independently takes values $0, \frac{1}{2}, 1, \ldots$. The familiar fields of particle physics correspond to specific representations in this classification:
\begin{itemize}
\item $(0, 0)$: scalar field (e.g., the Higgs boson)
\item $(\frac{1}{2}, 0)$: left-handed Weyl spinor
\item $(0, \frac{1}{2})$: right-handed Weyl spinor
\item $(\frac{1}{2}, 0) \oplus (0, \frac{1}{2})$: Dirac spinor (massive fermions)
\item $(\frac{1}{2}, \frac{1}{2})$: four-vector field (gauge bosons)
\end{itemize}
The existence of two distinct spinor representations---left-handed and right-handed---is not imposed by hand but follows from the structure of spacetime itself. Their physical distinction through the weak interaction is discussed in Section~\ref{sec:gauge_structure}.

The union of quantum mechanics with special relativity carries a further consequence. Consistent quantization of fields on Minkowski spacetime---requiring Lorentz invariance, unitarity, and locality---leads to the CPT theorem: every particle must have a corresponding antiparticle with identical mass and spin but opposite internal quantum numbers. The spin-statistics theorem, arising from the same requirements, dictates that the half-integer-spin fermions described by spinor representations obey Fermi--Dirac statistics, while integer-spin bosons obey Bose--Einstein statistics. Matter and antimatter are thus not independent postulates but inevitable consequences of relativistic quantum field theory~\cite{Streater:1989vi}.

The experimentally observed matter content of the SM consists of twelve spin-$\frac{1}{2}$ fermions organized into three generations, each containing two quarks and two leptons, summarized in Table~\ref{tab:sm_fermions}.

\begin{table}[htbp]
\centering
\caption{The three generations of fermions in the Standard Model with their electric charges and approximate masses~\cite{PDG2024}.}
\label{tab:sm_fermions}
\begin{tabular}{lllcl}
\hline\hline
Generation & Particle & Type & Charge & Mass \\
\hline
First & up (\PQu) & quark & $+2/3$ & $\sim$2.2 MeV \\
      & down (\PQd) & quark & $-1/3$ & $\sim$4.7 MeV \\
      & electron (\Pe) & lepton & $-1$ & 0.511 MeV \\
      & $\nu_e$ & neutrino & $0$ & $<$1 eV \\
\hline
Second & charm (\PQc) & quark & $+2/3$ & $\sim$1.27 GeV \\
       & strange (\PQs) & quark & $-1/3$ & $\sim$93 MeV \\
       & muon (\Pgm) & lepton & $-1$ & 105.7 MeV \\
       & $\nu_\mu$ & neutrino & $0$ & $<$1 eV \\
\hline
Third & top (\PQt) & quark & $+2/3$ & $\sim$172.8 GeV \\
      & bottom (\PQb) & quark & $-1/3$ & $\sim$4.18 GeV \\
      & tau (\Pgt) & lepton & $-1$ & 1.777 GeV \\
      & $\nu_\tau$ & neutrino & $0$ & $<$1 eV \\
\hline\hline
\end{tabular}
\end{table}

Quarks carry color charge and participate in all three SM interactions, while leptons do not carry color charge and are thus absent from the strong interaction. This generational structure was not predicted from first principles but emerged from successive experimental discoveries---from the muon (1936) and strange quark physics, through the electroweak unification by Glashow, Weinberg, and Salam~\cite{Glashow:1961tr,Weinberg:1967tq,Salam:1968rm}, to the third-generation particles discovered between 1975 and 1995.

\subsection{Gauge Structure and Interactions}
\label{sec:gauge_structure}

The SM is based on the gauge symmetry group:
\begin{equation}
G_{\text{SM}} = \SU(3)_C \times \SU(2)_L \times \U(1)_Y
\end{equation}
where $\SU(3)_C$ describes the strong interaction (quantum chromodynamics, QCD), and $\SU(2)_L \times \U(1)_Y$ describes the electroweak interaction before symmetry breaking.

\subsubsection{The Gauge Principle and Local Symmetry}

Quantum field theory is fundamentally local: the physics at one spacetime point should not depend on arbitrary phase conventions chosen at distant points. A global symmetry transformation rotates all particle fields by the same phase everywhere in spacetime. Promoting this to a local symmetry---allowing independent phase rotations at each spacetime point---is a powerful constraint. Demanding that the theory remain invariant under such local gauge transformations requires the introduction of new physics to maintain consistency between neighboring points.

The ordinary derivative $\partial_\mu$ fails to be gauge-covariant under local transformations: it cannot distinguish between a change in the field due to dynamics versus a change due to local phase redefinition. The unique solution is to introduce a connection---the covariant derivative $D_\mu$---that parallel-transports symmetry charges between spacetime points. This connection is precisely the gauge field, and the requirement of locality thus mandates the existence of force-carrying gauge bosons. Each generator of the symmetry group requires one gauge boson: three for $\SU(2)_L$ (the $W^1$, $W^2$, $W^3$ bosons) and one for $\U(1)_Y$ (the $B$ boson). Interactions arise not as ad hoc additions to the theory, but as geometric necessities of maintaining local symmetry.

\subsubsection{Parity Violation and the Chiral Structure of Weak Interactions}

As discussed in Section~\ref{sec:spacetime_symmetries}, the Lorentz group admits two distinct spinor representations---left-handed $(\frac{1}{2}, 0)$ and right-handed $(0, \frac{1}{2})$. The electromagnetic and strong interactions treat these identically, preserving parity. It was therefore a profound surprise when Wu's 1957 experiment on cobalt-60 beta decay~\cite{Wu:1957my} demonstrated that the weak interaction maximally violates parity, coupling exclusively to left-handed fermions and right-handed antifermions.

This observation was codified in the $V{-}A$ (vector minus axial-vector) theory of Feynman and Gell-Mann~\cite{Feynman:1958ty} and Sudarshan and Marshak~\cite{Sudarshan:1958vf}, which describes the charged weak current as:
\begin{equation}
J^\mu_{\text{CC}} \propto \bar{\psi}\gamma^\mu(1 - \gamma^5)\psi
\end{equation}
The projection operator $(1 - \gamma^5)/2$ selects only the left-handed component of the fermion field, implementing maximal parity violation. This empirical structure finds its natural explanation in the gauge framework: left-handed fermions transform as $\SU(2)_L$ doublets, while right-handed fermions are $\SU(2)_L$ singlets. The subscript ``$L$'' in $\SU(2)_L$ directly encodes this experimental fact.

Left-handed fermions are thus organized into $\SU(2)_L$ doublets:
\begin{equation}
\ell_L = \begin{pmatrix} \nu_e \\ e^- \end{pmatrix}_L, \quad
q_L = \begin{pmatrix} u \\ d \end{pmatrix}_L
\end{equation}
while right-handed fermions are $\SU(2)_L$ singlets: $e_R$, $u_R$, $d_R$ (and similarly for the other generations). In the minimal SM, right-handed neutrinos $\nu_R$ are not included. The $\SU(2)_L$ generators act non-trivially only on left-handed doublets, so that weak charged currents couple exclusively to left-handed fermions (and right-handed antifermions).

\subsubsection{Quantum Number Assignments}

To fully specify how fermions transform under the electroweak gauge group, we assign three quantum numbers to each chirality state. These quantum numbers are not arbitrary labels but arise directly from the symmetry structure. By Noether's theorem, each continuous symmetry possesses conserved charges. The quantum number $I_3$ is the eigenvalue of the $\SU(2)_L$ generator $T^3$---the third component of weak isospin. The quantum number $Y$ is the charge under the $\U(1)_Y$ symmetry, called weak hypercharge. The electric charge $Q$ emerges from the unbroken $\U(1)_{\text{em}}$ subgroup that survives after electroweak symmetry breaking. The Gell-Mann--Nishijima relation $Q = I_3 + Y$ reflects this emergence: electric charge is a specific linear combination of weak isospin and hypercharge determined by the pattern of symmetry breaking.

The Gell-Mann--Nishijima relation is:
\begin{equation}
Q = I_3 + Y
\label{eq:gellmann_nishijima}
\end{equation}

Table~\ref{tab:fermion_quantum_numbers} summarizes the quantum number assignments for the first generation of fermions. The pattern repeats for the second and third generations.

\begin{table}[htbp]
\centering
\caption{Quantum number assignments for the first generation of fermions under $\SU(2)_L \times \U(1)_Y$. The electric charge $Q$ is related to weak isospin $I_3$ and hypercharge $Y$ through $Q = I_3 + Y$. The pattern repeats for the second and third generations.}
\label{tab:fermion_quantum_numbers}
\begin{tabular}{lccccc}
\hline\hline
Particle & Chirality & $I_3$ & $Y$ & $Q$ & Representation \\
\hline
$\nu_e$ & Left & $+1/2$ & $-1/2$ & $0$ & \multirow{2}{*}{$\SU(2)_L$ doublet} \\
$e^-$ & Left & $-1/2$ & $-1/2$ & $-1$ & \\
$u$ & Left & $+1/2$ & $+1/6$ & $+2/3$ & \multirow{2}{*}{$\SU(2)_L$ doublet} \\
$d$ & Left & $-1/2$ & $+1/6$ & $-1/3$ & \\
\hline
$\nu_e$ & Right & $0$ & $0$ & $0$ & $\SU(2)_L$ singlet \\
$e^-$ & Right & $0$ & $-1$ & $-1$ & $\SU(2)_L$ singlet \\
$u$ & Right & $0$ & $+2/3$ & $+2/3$ & $\SU(2)_L$ singlet \\
$d$ & Right & $0$ & $-1/3$ & $-1/3$ & $\SU(2)_L$ singlet \\
\hline\hline
\end{tabular}
\end{table}

Note that left-handed fermions have $I_3 = \pm 1/2$ (doublet members), while right-handed fermions have $I_3 = 0$ (singlets). The hypercharge $Y$ differs from the electric charge $Q$; for example, the left-handed electron has $Y = -1/2$ but $Q = -1$.

\subsubsection{Gauge Bosons and Couplings}

The gauge principle predicts the existence of massless gauge bosons in their natural gauge eigenstate basis. The $\SU(2)_L$ symmetry requires three gauge bosons $A^1_\mu$, $A^2_\mu$, and $A^3_\mu$ (one for each generator), while the $\U(1)_Y$ symmetry requires one gauge boson $B_\mu$---a total of four massless vector bosons. However, experiment observes a different spectrum: the $\PW^\pm$ and $\PZ$ bosons with masses around 80--90\GeV, and the massless photon $\Pgg$. The physical mass eigenstates are mixtures of the gauge eigenstates, with mixing determined by electroweak symmetry breaking. This tension between the gauge structure (predicting massless bosons) and experimental reality (observing massive weak bosons) requires a mechanism to generate masses while preserving the gauge symmetry that ensures renormalizability. This mechanism---electroweak symmetry breaking---is discussed in the following section.

After symmetry breaking, these gauge eigenstates mix to form the physical mass eigenstates: the charged weak bosons $\PWpm$, the neutral weak boson $\PZ$, and the massless photon $\Pgg$. The mixing is characterized by the weak mixing angle $\theta_W$ (or Weinberg angle).

The electroweak sector is characterized by two independent coupling constants: $g$ for the $\SU(2)_L$ gauge coupling and $g'$ for the $\U(1)_Y$ gauge coupling.

The weak mixing angle relates these couplings:
\begin{equation}
\tan\theta_W = \frac{g'}{g}
\end{equation}
with the experimentally measured value $\swsq \approx 0.23$~\cite{PDG2024}. The mixing angle $\theta_W$ determines how the gauge eigenstates $B_\mu$ and $A^3_\mu$ combine to form the physical $\PZ$ boson and photon. It is measured experimentally rather than predicted by the theory, but once measured, it constrains the mass relation $m_Z = m_W/\cos\theta_W$ discussed below---a prediction that serves as a precision test of the symmetry breaking mechanism.

The elementary electric charge $e$ is related to $g$ and $g'$ by:
\begin{equation}
e = g \sin\theta_W = g' \cos\theta_W = \frac{gg'}{\sqrt{g^2 + g'^2}}
\label{eq:electric_charge}
\end{equation}

After symmetry breaking, neutral currents are mediated by both the photon and the $\PZ$ boson. The photon couples to the electric charge $Q$, while the $\PZ$ boson couples to a different combination of quantum numbers:
\begin{equation}
Q_Z = I_3 - \swsq Q
\label{eq:z_coupling}
\end{equation}
This $Z$ coupling quantum number determines the strength of neutral-current weak interactions for each fermion species, and will play a key role when we discuss the masses and interactions of the physical gauge bosons in the next section.

\subsection{Electroweak Symmetry Breaking and the Higgs Mechanism}

As established in the previous section, gauge bosons are observed to be massive, yet direct mass terms of the form $\frac{1}{2}m^2 W_\mu W^\mu$ explicitly break gauge invariance. This is not merely an aesthetic problem: breaking gauge invariance destroys the cancellations that ensure renormalizability, causing the theory to produce nonsensical infinite predictions at high energies. The resolution to this crisis is spontaneous symmetry breaking---a mechanism where the gauge symmetry is preserved in the Lagrangian but broken by the choice of vacuum state. The Brout-Englert-Higgs mechanism~\cite{Higgs:1964pj,Englert:1964et,Guralnik:1964eu} implements this idea: gauge boson masses emerge dynamically from interactions with a symmetry-breaking vacuum, preserving the theoretical structure that ensures predictivity. This elegant solution maintains renormalizability while generating the observed mass spectrum.

\subsubsection{The Higgs Doublet and Scalar Potential}

The SM introduces a complex scalar $\SU(2)_L$ doublet with hypercharge $Y = +1/2$:
\begin{equation}
\Phi = \begin{pmatrix} \phi^+ \\ \phi^0 \end{pmatrix}
\end{equation}
where $\phi^+$ is a charged complex field and $\phi^0$ is a neutral complex field. This doublet has four real degrees of freedom.

The most general renormalizable scalar potential respecting the gauge symmetry is:
\begin{equation}
V(\Phi) = -\mu^2 \Phi^\dagger\Phi + \lambda(\Phi^\dagger\Phi)^2
\label{eq:higgs_potential}
\end{equation}
For $\mu^2 > 0$ and $\lambda > 0$, this potential has the characteristic ``Mexican-hat'' shape: the origin $\Phi = 0$ is a local maximum, and the minimum forms a circle of degenerate vacua with $|\Phi|^2 = \mu^2/(2\lambda)$. The system must choose one vacuum state, spontaneously breaking the $\SU(2)_L \times \U(1)_Y$ symmetry.

When a continuous symmetry breaks spontaneously, Goldstone's theorem predicts the appearance of massless scalar bosons---one for each broken generator. The gauge group $\SU(2)_L \times \U(1)_Y$ has four generators, but the vacuum preserves the $\U(1)_{\text{em}}$ subgroup (one generator), leaving three broken. We thus expect three Goldstone bosons. Simultaneously, massive gauge bosons require three longitudinal polarization states beyond the two transverse polarizations of massless vectors. These are the same degrees of freedom: the three Goldstone bosons are absorbed to become the longitudinal modes of the $\PW^\pm$ and $\PZ$ bosons. The Higgs doublet's four real scalar degrees of freedom rearrange into three longitudinal gauge boson polarizations plus one physical scalar---the Higgs boson. This rearrangement, not a true ``disappearance,'' preserves the total degree-of-freedom count while yielding the physical spectrum.

By convention, we choose the vacuum configuration where the neutral component acquires a non-zero vacuum expectation value (vev):
\begin{equation}
\langle\Phi\rangle = \frac{1}{\sqrt{2}}\begin{pmatrix} 0 \\ v \end{pmatrix}, \quad v = \frac{\mu}{\sqrt{\lambda}} \approx 246\GeV
\label{eq:higgs_vev}
\end{equation}
This choice breaks $\SU(2)_L \times \U(1)_Y$ while preserving a residual $\U(1)_{\text{em}}$ symmetry associated with electric charge $Q = I_3 + Y$.

\subsubsection{Higgs Field Parametrization and Goldstone Bosons}

To study fluctuations around the vacuum, we parametrize the Higgs field as:
\begin{equation}
\Phi(x) = \begin{pmatrix} \chi^+(x) \\ \frac{1}{\sqrt{2}}[v + h(x) + i\chi^3(x)] \end{pmatrix}
\end{equation}
where $h(x)$ is a real scalar field representing excitations along the radial direction of the potential, and $\chi^+(x)$, $\chi^3(x)$ are three real fields associated with the angular directions around the minimum.

The three fields $\chi^+$, $\chi^-$ (the complex conjugate of $\chi^+$), and $\chi^3$ are Goldstone bosons---massless scalar excitations arising from the spontaneous breaking of the gauge symmetry. However, these are not physical degrees of freedom. Through an $\SU(2)_L$ gauge transformation (known as choosing the unitary gauge), we can eliminate the Goldstone fields, leaving:
\begin{equation}
\Phi(x) = \frac{1}{\sqrt{2}}\begin{pmatrix} 0 \\ v + h(x) \end{pmatrix}
\label{eq:unitary_gauge}
\end{equation}
The three Goldstone bosons are ``eaten'' by the gauge bosons, providing the longitudinal polarization states necessary for massive vector bosons. The remaining field $h(x)$ is the physical Higgs boson with mass $m_h = \sqrt{2\lambda}v \approx 125\GeV$~\cite{ATLAS:2012yve,CMS:2012qbp}.

\subsubsection{Gauge Boson Mass Generation}

Gauge boson masses arise from the gauge-covariant kinetic term $|D_\mu\Phi|^2$ evaluated at the vacuum expectation value. This is not a mass term added by hand, but rather emerges dynamically from the kinetic energy of the scalar field. The covariant derivative encodes how gauge bosons couple to the Higgs doublet, and when the Higgs field takes a non-zero vacuum value, these couplings generate quadratic terms in the gauge fields---precisely the structure of mass terms. The following calculation shows how $m_W$ and $m_Z$ depend on the vacuum expectation value $v$ and the gauge couplings $g$ and $g'$.

To see how gauge bosons acquire mass, we examine the covariant derivative acting on the Higgs doublet:
\begin{equation}
D_\mu\Phi = \left(\partial_\mu - ig\frac{\sigma^a}{2}A^a_\mu - ig'Y B_\mu\right)\Phi
\label{eq:covariant_derivative}
\end{equation}
where $\sigma^a$ are the Pauli matrices, $A^a_\mu$ ($a=1,2,3$) are the $\SU(2)_L$ gauge fields, $B_\mu$ is the $\U(1)_Y$ gauge field, and $Y = +1/2$ is the hypercharge of $\Phi$.

The kinetic term for the Higgs field is $|D_\mu\Phi|^2$. When we substitute the vacuum configuration $\langle\Phi\rangle = (0, v/\sqrt{2})^T$ into this expression, the resulting terms quadratic in gauge fields correspond to gauge boson masses.

Consider first the charged gauge bosons. The combinations:
\begin{equation}
W^\pm_\mu = \frac{1}{\sqrt{2}}(A^1_\mu \mp i A^2_\mu)
\label{eq:w_bosons}
\end{equation}
give rise to mass terms:
\begin{equation}
\frac{1}{2}m_W^2 W^+_\mu W^{-\mu} = \frac{g^2 v^2}{8}[(A^1_\mu)^2 + (A^2_\mu)^2]
\end{equation}
from which we identify the $\PW$ boson mass:
\begin{equation}
m_W = \frac{gv}{2}
\label{eq:w_mass}
\end{equation}

\subsubsection{Neutral Gauge Boson Mixing}

The neutral gauge fields $A^3_\mu$ and $B_\mu$ do not have definite masses in the gauge eigenstate basis. Instead, they mix to form the physical mass eigenstates---the photon $A_\mu$ and the $Z$ boson $Z_\mu$. The mixing is described by a rotation through the weak mixing angle $\theta_W$:
\begin{equation}
\begin{pmatrix} A_\mu \\ Z_\mu \end{pmatrix} =
\begin{pmatrix} \cos\theta_W & \sin\theta_W \\ -\sin\theta_W & \cos\theta_W \end{pmatrix}
\begin{pmatrix} B_\mu \\ A^3_\mu \end{pmatrix}
\label{eq:neutral_mixing}
\end{equation}

The photon field $A_\mu$ remains massless because it couples to the unbroken $\U(1)_{\text{em}}$ generator $Q = I_3 + Y$. In contrast, the $Z$ boson acquires mass:
\begin{equation}
m_Z = \frac{v\sqrt{g^2 + g'^2}}{2} = \frac{m_W}{\cos\theta_W}
\label{eq:z_mass}
\end{equation}

The relation $m_Z = m_W/\cos\theta_W$ is a robust prediction of the electroweak symmetry breaking mechanism, not a parameter fit to data. Given measurements of $m_W$, $m_Z$, and $\sin^2\theta_W$, this relation provides a precision test of the theory. Experimental confirmation to sub-percent accuracy~\cite{PDG2024} validates the spontaneous symmetry breaking framework. Deviations from this relation would signal physics beyond the minimal SM, making precision measurements of electroweak observables powerful probes of new physics.

\subsubsection{Covariant Derivative in the Mass Basis}

After symmetry breaking and diagonalizing to mass eigenstates, the covariant derivative describing fermion interactions takes the form:
\begin{equation}
D_\mu = \partial_\mu - ieA_\mu Q - i\frac{g}{\cos\theta_W}Z_\mu Q_Z - ig\left(W^+_\mu T^+ + W^-_\mu T^-\right)
\label{eq:covariant_mass_basis}
\end{equation}
where $T^\pm = (T^1 \pm iT^2)/\sqrt{2}$ are the raising and lowering operators for weak isospin, and the $Z$ coupling quantum number $Q_Z = I_3 - \swsq Q$ was defined in Eq.~(\ref{eq:z_coupling}).

This expression makes manifest the physical content of the electroweak theory. The photon couples universally to electric charge $Q$ with strength $e$. The $\PZ$ boson mediates neutral currents with coupling proportional to $Q_Z$, which differs between fermion species depending on their quantum numbers as shown in Table~\ref{tab:fermion_quantum_numbers}. The $\PWpm$ bosons mediate charged currents, connecting doublet partners such as $u \leftrightarrow d$ and $\nu_e \leftrightarrow e^-$.

\subsubsection{Fermion Masses}

While gauge boson masses emerge from the Higgs kinetic term $|D_\mu\Phi|^2$, fermion masses require a different mechanism. Direct fermion mass terms like $m\bar{\psi}\psi$ are forbidden because they would connect left-handed and right-handed states that transform differently under $\SU(2)_L$. Instead, fermion masses arise through Yukawa couplings---interactions between fermions and the Higgs doublet. After symmetry breaking, these couplings generate mass terms proportional to the vacuum expectation value.

Fermion masses arise from Yukawa interactions with the Higgs doublet:
\begin{equation}
\mathcal{L}_{\text{Yukawa}} = -y_f \bar{\psi}_L \Phi \psi_R + \text{h.c.}
\end{equation}
After symmetry breaking and substituting $\langle\Phi\rangle = (0, v/\sqrt{2})^T$, these interactions generate fermion mass terms:
\begin{equation}
m_f = \frac{y_f v}{\sqrt{2}}
\end{equation}
Unlike gauge boson masses, which are predicted once $v$, $g$, and $g'$ are measured, fermion masses are parametrized by the Yukawa couplings $y_f$, which are free parameters in the SM. The large hierarchy of fermion masses (from the electron at 0.511~MeV to the top quark at 172.8~GeV, spanning six orders of magnitude) thus translates into a hierarchy of Yukawa couplings, ranging from $\sim 3 \times 10^{-6}$ for the electron to $\sim 1$ for the top quark. This ``flavor problem''---the origin of the Yukawa hierarchy---remains an open question in particle physics and motivates many extensions of the SM.

\section{Limitations of the Standard Model}
\label{sec:theory_bsm}

Despite its remarkable success, the SM fails to explain several observed phenomena and exhibits theoretical features that suggest it is not the ultimate theory of particle physics.

\subsection{Experimental Anomalies}

Several experimental observations cannot be explained within the SM framework:

\subsubsection{Dark Matter}

Cosmological and astrophysical observations provide compelling evidence for dark matter, constituting approximately 27\% of the universe's energy density~\cite{Planck:2018vyg}. The SM contains no viable dark matter candidate, as all SM particles either decay rapidly or would have been detected through their interactions.

\subsubsection{Matter-Antimatter Asymmetry}

The observed universe consists almost entirely of matter rather than antimatter. While the SM contains sources of CP violation through the CKM matrix, the magnitude is insufficient to explain the observed baryon asymmetry~\cite{Sakharov:1967dj}.

\subsubsection{Neutrino Masses}

Neutrino oscillation experiments have definitively established that neutrinos have non-zero masses~\cite{PhysRevLett.81.1158,SNO:2002tuh}. In the minimal SM, neutrinos are massless, and their mass generation requires new physics.

\subsection{Fine-Tuned Theory Parameters}

The SM contains parameters that require unexplained fine-tuning:

\subsubsection{The Hierarchy Problem}

The Higgs boson mass receives large quantum corrections from loops involving heavy particles. Without fine-tuning, these corrections would drive the Higgs mass to the Planck scale. This ``naturalness'' problem suggests the existence of new physics at the TeV scale to stabilize the electroweak scale.

\subsubsection{Strong CP Problem}

QCD allows a CP-violating term proportional to $\bar{\theta}$. Experimental bounds on the neutron electric dipole moment constrain $|\bar{\theta}| < 10^{-10}$~\cite{Abel:2020pzs}, requiring an unexplained fine-tuning.

\subsection{Extended Higgs Sector}
\label{sec:theory_extended}

The SM Higgs sector, while minimal, is not unique. Extended Higgs sectors with additional scalar fields are motivated by several theoretical considerations:

\begin{itemize}
\item Supersymmetry requires at least two Higgs doublets to give masses to both up-type and down-type fermions
\item Additional Higgs doublets can provide new sources of CP violation for baryogenesis
\item Extended scalar sectors can accommodate dark matter candidates
\item The Peccei-Quinn solution to the strong CP problem involves an additional scalar field
\end{itemize}

The simplest extension is the Two-Higgs-Doublet Model (2HDM), which introduces a second $\SU(2)_L$ doublet while maintaining the SM gauge structure.

\section{The Two-Higgs-Doublet Model}
\label{sec:theory_2hdm}

\subsection{General Review}

\subsubsection{Scalar Potential and Symmetry Breaking}

The 2HDM introduces two complex $\SU(2)_L$ doublets $\Phi_1$ and $\Phi_2$, each with hypercharge $Y = +1/2$:
\begin{equation}
\Phi_i = \begin{pmatrix} \phi_i^+ \\ \phi_i^0 \end{pmatrix}, \quad i = 1,2
\end{equation}

The most general renormalizable scalar potential respecting the gauge symmetry is:
\begin{align}
V(\Phi_1, \Phi_2) &= m_{11}^2 \Phi_1^\dagger\Phi_1 + m_{22}^2 \Phi_2^\dagger\Phi_2 - m_{12}^2(\Phi_1^\dagger\Phi_2 + \text{h.c.}) \nonumber\\
&+ \frac{\lambda_1}{2}(\Phi_1^\dagger\Phi_1)^2 + \frac{\lambda_2}{2}(\Phi_2^\dagger\Phi_2)^2 + \lambda_3(\Phi_1^\dagger\Phi_1)(\Phi_2^\dagger\Phi_2) \nonumber\\
&+ \lambda_4(\Phi_1^\dagger\Phi_2)(\Phi_2^\dagger\Phi_1) + \frac{\lambda_5}{2}\left[(\Phi_1^\dagger\Phi_2)^2 + \text{h.c.}\right]
\end{align}
where we have imposed a softly broken $\mathbb{Z}_2$ symmetry ($\Phi_1 \to \Phi_1$, $\Phi_2 \to -\Phi_2$) to avoid tree-level flavor-changing neutral currents (FCNCs).

After electroweak symmetry breaking, both doublets can acquire vevs:
\begin{equation}
\langle\Phi_i\rangle = \frac{1}{\sqrt{2}}\begin{pmatrix} 0 \\ v_i \end{pmatrix}
\end{equation}
with $v^2 = v_1^2 + v_2^2 = (246\GeV)^2$. The ratio of vevs defines an important parameter:
\begin{equation}
\tan\beta \equiv \frac{v_2}{v_1}
\end{equation}

\subsubsection{Physical Higgs Bosons}

The eight degrees of freedom in the two doublets rearrange as follows after symmetry breaking:
\begin{itemize}
\item Three Goldstone bosons ($G^\pm$, $G^0$) become the longitudinal modes of \PWpm and \PZ
\item Five physical Higgs bosons remain: \Ph, \PH (CP-even), \PA (CP-odd), and \PHpm (charged)
\end{itemize}

In the CP-conserving case, the neutral CP-even states mix through an angle $\alpha$:
\begin{equation}
\begin{pmatrix} \Ph \\ \PH \end{pmatrix} = \begin{pmatrix} \cos\alpha & \sin\alpha \\ -\sin\alpha & \cos\alpha \end{pmatrix} \begin{pmatrix} \rho_1 \\ \rho_2 \end{pmatrix}
\end{equation}
where $\rho_{1,2}$ are the neutral CP-even components of $\Phi_{1,2}$.

The masses of the physical Higgs bosons are related to the potential parameters:
\begin{align}
m_{\Ph,\PH}^2 &= \frac{1}{2}\left[(A_{11} + A_{22}) \mp \sqrt{(A_{11} - A_{22})^2 + 4A_{12}^2}\right] \\
m_\PA^2 &= m_{12}^2\left(\frac{1}{s_\beta c_\beta}\right) - \lambda_5 v^2 \\
m_{\PHpm}^2 &= m_\PA^2 + \frac{1}{2}(\lambda_5 - \lambda_4)v^2
\end{align}
where $s_\beta = \sin\beta$, $c_\beta = \cos\beta$, and $A_{ij}$ are elements of the CP-even mass matrix.

\subsubsection{Types of 2HDM}

To avoid FCNCs at tree level, discrete symmetries are imposed such that each type of fermion couples to only one Higgs doublet. This leads to four distinct types of 2HDM:

\begin{table}[htbp]
\centering
\caption{Yukawa coupling structure in the four types of 2HDM. The table shows which Higgs doublet couples to each type of fermion.}
\label{tab:2hdm_types}
\begin{tabular}{ccccc}
\hline\hline
Type & Up-type quarks & Down-type quarks & Charged leptons \\
\hline
Type-I & $\Phi_2$ & $\Phi_2$ & $\Phi_2$ \\
Type-II & $\Phi_2$ & $\Phi_1$ & $\Phi_1$ \\
Type-X (Lepton-specific) & $\Phi_2$ & $\Phi_2$ & $\Phi_1$ \\
Type-Y (Flipped) & $\Phi_2$ & $\Phi_1$ & $\Phi_2$ \\
\hline\hline
\end{tabular}
\end{table}

This analysis is primarily motivated by the Type-I 2HDM, where all fermions couple to $\Phi_2$. In this scenario, the couplings of the charged Higgs to fermions are proportional to $\cot\beta$, and for large \tanb, these couplings become suppressed.

\subsubsection{The Alignment Limit}

The discovered 125~GeV Higgs boson has properties consistent with the SM predictions within experimental uncertainties~\cite{ATLAS:2022vkf,CMS:2022dwd}. This constrains the 2HDM parameter space to be near the ``alignment limit,'' where one of the CP-even Higgs bosons has SM-like couplings.

The alignment limit is characterized by $\cos(\beta - \alpha) \to 0$ or equivalently $\sin(\beta - \alpha) \to 1$. In this limit, \Ph has SM-like couplings, while \PH couples to vector bosons proportionally to $\cos(\beta - \alpha)$.

\subsection{Charged Higgs Boson Phenomenology}
\label{sec:theory_charged}

\subsubsection{Production Mechanisms}

For $\mHc < \mt$, charged Higgs bosons are predominantly produced through top quark decays:
\begin{equation}
\PQt \to \PHc\PQb
\end{equation}
competing with the SM decay $\PQt \to \PWp\PQb$. The branching fraction depends on \mHc and \tanb:
\begin{equation}
\mathcal{B}(\PQt \to \PHc\PQb) \propto \left(\frac{m_t^2}{v^2}\cot^2\beta + \frac{m_b^2}{v^2}\tan^2\beta\right)\left(1 - \frac{m_{\PHc}^2}{m_t^2}\right)^2
\end{equation}

At low \tanb, the $m_t$ term dominates, while at high \tanb, the $m_b$ term can become important (particularly in Type-II models). In Type-I models, both terms are suppressed at high \tanb.

For $\mHc > \mt$, production occurs through:
\begin{itemize}
\item Associated production with top quark: $\Pg\Pg/\Pg\PQb \to \PHc\PQt\PQb$
\item Pair production: $\Pq\Paq \to \PHc\PHm$
\end{itemize}

\subsubsection{Decay Modes}

The charged Higgs can decay through both fermionic and bosonic channels:

\textbf{Fermionic decays:}
\begin{itemize}
\item $\PHpm \to \PQt\PAQb$ (kinematically forbidden for $\mHc < \mt$)
\item $\PHpm \to \PGtp\PGn$
\item $\PHpm \to \PQc\PAQs$, $\PHpm \to \PQc\PAQb$
\end{itemize}

\textbf{Bosonic decays:}
\begin{itemize}
\item $\PHpm \to \PWpm\Ph$
\item $\PHpm \to \PWpm\PA$
\item $\PHpm \to \PWpm\PH$
\end{itemize}

The partial widths for fermionic decays in a Type-I 2HDM are:
\begin{align}
\Gamma(\PHpm \to \PQt\PAQb) &\propto \frac{m_t^2 + m_b^2}{v^2}\cot^2\beta \\
\Gamma(\PHpm \to \PGtp\PGn) &\propto \frac{m_\tau^2}{v^2}\cot^2\beta
\end{align}

The bosonic decay $\PHpm \to \PWpm\PA$ has partial width:
\begin{equation}
\Gamma(\PHpm \to \PWpm\PA) = \frac{g^2}{64\pi}\frac{\cos^2(\beta-\alpha)}{m_\PW^2}\lambda^{3/2}(m_{\PHc}^2, m_\PW^2, m_\PA^2) \cdot m_{\PHc}
\end{equation}
where $\lambda(a,b,c) = (1 - b/a - c/a)^2 - 4bc/a^2$ is the Källén function.

In the alignment limit ($\cos(\beta-\alpha) \to 0$), this decay is suppressed. However, away from exact alignment, particularly for large \tanb in Type-I models where fermionic decays are suppressed, the bosonic decay $\PHpm \to \PWpm\PA$ can become dominant.

\subsubsection{The Target Decay Chain}

This analysis targets the decay chain:
\begin{equation}
\PHpm \to \PWpm\PA \to \PWpm\PGmp\PGmm
\end{equation}

The pseudoscalar \PA decays to fermion pairs with branching fractions proportional to the squared fermion masses. For $\mA > 2m_\Pgm$ and below the $\PQb\PAQb$ threshold, the dimuon channel provides a clean experimental signature:
\begin{equation}
\mathcal{B}(\PA \to \PGmp\PGmm) \approx \frac{m_\mu^2}{m_\tau^2 + m_c^2 + 3m_\mu^2} \approx 3\%
\end{equation}
for $2m_\mu < \mA < 2m_\tau$. For heavier \PA masses, the $\tau\tau$ and $\PQb\PAQb$ channels dominate.

\subsubsection{Off-Shell Decays}

An important feature of this analysis is the inclusion of off-shell $\PHpm \to \PWpm\PA$ decays. When $\mA > \mHc - m_\PW$, the \PW boson in the decay is virtual. The three-body decay width can be written as:
\begin{equation}
\Gamma(\PHpm \to \PA\ell\nu) = \int \frac{d\Gamma(\PHpm \to \text{W}^{\pm *}\PA)}{dm_{W^*}^2} \cdot \mathcal{B}(\text{W}^{\pm *} \to \ell\nu) \, dm_{W^*}^2
\end{equation}

While the off-shell decay rate is suppressed compared to on-shell decays, it remains significant in regions of parameter space where fermionic decays are suppressed. Calculations using 2HDMC~\cite{2HDMC} show that the product of branching fractions $\mathcal{B}(\PQt \to \PHc\PQb) \times \mathcal{B}(\PHc \to \PWpm\PA) \times \mathcal{B}(\PA \to \PGmp\PGmm)$ can reach values accessible at the LHC even in the off-shell region.

\subsection{Previous Searches and Current Constraints}
\label{sec:theory_constraints}

\subsubsection{LEP and Tevatron Results}

At LEP, the DELPHI and OPAL experiments searched for charged Higgs pair production $\Pe^+\Pe^- \to \PHc\PHm$ and set lower bounds $\mHc > 72$--80\GeV depending on the assumed decay modes~\cite{DELPHIChargedHiggs,OPALChargedHiggs}.

The CDF experiment at the Tevatron searched for $\PHpm \to \PWpm\PA \to \PWpm\PGtp\PGtm$ and set limits on the branching fraction~\cite{CDFChargedHiggs}.

\subsubsection{LHC Run 1 and Run 2 Results}

During LHC Run 1 and Run 2, both ATLAS and CMS performed extensive searches for charged Higgs bosons:
\begin{itemize}
\item $\PHpm \to \PGtp\PGn$: Strong constraints for $\mHc < \mt$~\cite{ATLAS:2018gfm,CMS:2019bfg}
\item $\PHpm \to \PQt\PAQb$: Constraints for $\mHc > \mt$~\cite{ATLAS:2021upq,CMS:2020imj}
\item $\PHpm \to \PQc\PAQs/\PQc\PAQb$: Constraints at low \tanb~\cite{ATLAS:2024oqu,CMS:2024iqz}
\item $\PHpm \to \PWpm\PA$: Searches by CMS (35.9\fbinv)~\cite{CMSRun2ChargedHiggs} and ATLAS (138\fbinv)~\cite{ATLASRun2ChargedHiggs}
\end{itemize}

The previous CMS search for $\PHpm \to \PWpm\PA \to \PWpm\PGmp\PGmm$ using 2016 data (35.9\fbinv) found no significant excess and set upper limits on the signal cross section. The ATLAS search using full Run 2 data similarly found no evidence for charged Higgs production in this channel.

\subsubsection{Theoretical Constraints}

The 2HDM parameter space is constrained by:
\begin{itemize}
\item \textbf{Perturbativity}: Requiring quartic couplings remain perturbative
\item \textbf{Unitarity}: S-matrix unitarity bounds on scalar scattering amplitudes
\item \textbf{Vacuum stability}: The potential must be bounded from below
\item \textbf{Electroweak precision observables}: The $\rho$ parameter constrains mass splittings
\item \textbf{Flavor physics}: $\PQb \to \PQs\Pgg$, $B$ meson mixing, and other processes
\end{itemize}

These constraints define the viable parameter space for the search, guiding the choice of mass ranges and benchmark scenarios considered in this analysis.
