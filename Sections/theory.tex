\chapter{Theoretical Background}
\label{ch:theory}

This chapter provides the theoretical foundation for the search presented in this thesis. We begin with an overview of the Standard Model of particle physics, followed by a discussion of its limitations and the motivation for extended Higgs sectors. The Two-Higgs-Doublet Model is then introduced, with particular emphasis on the phenomenology of charged Higgs bosons relevant to this analysis.

\section{The Standard Model of Particle Physics}
\label{sec:theory_sm}

The Standard Model (SM) is a quantum field theory that describes the fundamental particles and their interactions through three of the four known fundamental forces: the electromagnetic, weak, and strong interactions. Gravity, while well-described by general relativity at macroscopic scales, is not incorporated into the SM framework.

\subsection{Particle Content}

The matter content of the SM consists of twelve spin-$\frac{1}{2}$ fermions, organized into three generations. Each generation contains two quarks and two leptons:

\begin{table}[htbp]
\centering
\caption{The three generations of fermions in the Standard Model.}
\label{tab:sm_fermions}
\begin{tabular}{cccc}
\hline\hline
Generation & Quarks & Leptons (charged) & Leptons (neutral) \\
\hline
First & up (\PQu), down (\PQd) & electron (\Pe) & electron neutrino ($\nu_e$) \\
Second & charm (\PQc), strange (\PQs) & muon (\Pgm) & muon neutrino ($\nu_\mu$) \\
Third & top (\PQt), bottom (\PQb) & tau (\Pgt) & tau neutrino ($\nu_\tau$) \\
\hline\hline
\end{tabular}
\end{table}

Quarks carry color charge and participate in all three SM interactions. Leptons do not carry color charge and thus do not participate in strong interactions. The charged leptons interact electromagnetically and via the weak force, while neutrinos interact only through the weak force.

Each fermion exists in both left-handed and right-handed chirality states, which transform differently under the SM gauge group. This chiral structure is fundamental to the weak interaction, which violates parity maximally by coupling only to left-handed fermions (and right-handed antifermions).

\subsection{Gauge Structure and Interactions}

The SM is based on the gauge symmetry group:
\begin{equation}
G_{\text{SM}} = \SU(3)_C \times \SU(2)_L \times \U(1)_Y
\end{equation}
where $\SU(3)_C$ describes the strong interaction (quantum chromodynamics, QCD), and $\SU(2)_L \times \U(1)_Y$ describes the electroweak interaction before symmetry breaking.

The gauge bosons mediating these interactions are:
\begin{itemize}
\item Eight massless gluons (\Pg) for the strong force, carrying color charge
\item Three weak bosons (\PWpm, \PZ) and the photon (\Pgg) for the electroweak force
\end{itemize}

The electroweak sector is characterized by two coupling constants: $g$ for $\SU(2)_L$ and $g'$ for $\U(1)_Y$. The weak mixing angle $\theta_W$ relates these couplings:
\begin{equation}
\tan\theta_W = \frac{g'}{g}
\end{equation}

\subsection{Electroweak Symmetry Breaking and the Higgs Mechanism}

A central challenge in constructing the SM was explaining the masses of the \PW and \PZ bosons while maintaining the masslessness of the photon. This is achieved through the Brout-Englert-Higgs mechanism~\cite{Higgs:1964pj,Englert:1964et,Guralnik:1964eu}.

The SM introduces a complex scalar $\SU(2)_L$ doublet with hypercharge $Y = +1/2$:
\begin{equation}
\Phi = \begin{pmatrix} \phi^+ \\ \phi^0 \end{pmatrix}
\end{equation}

The scalar potential is chosen such that the neutral component acquires a vacuum expectation value (vev):
\begin{equation}
V(\Phi) = -\mu^2 \Phi^\dagger\Phi + \lambda(\Phi^\dagger\Phi)^2
\end{equation}
For $\mu^2 > 0$, the minimum occurs at $\langle\Phi\rangle = (0, v/\sqrt{2})^T$, where $v = \mu/\sqrt{\lambda} \approx 246\GeV$ is the electroweak vev.

After symmetry breaking, the $\SU(2)_L \times \U(1)_Y$ symmetry is spontaneously broken to $\U(1)_{\text{em}}$. Three of the four scalar degrees of freedom become the longitudinal polarizations of the \PWpm and \PZ bosons, giving them mass. The remaining degree of freedom manifests as the physical Higgs boson \Ph with mass $m_h = \sqrt{2\lambda}v$.

The gauge boson masses are:
\begin{equation}
m_W = \frac{gv}{2}, \qquad m_Z = \frac{v\sqrt{g^2 + g'^2}}{2} = \frac{m_W}{\cos\theta_W}
\end{equation}

Fermion masses arise from Yukawa interactions with the Higgs field:
\begin{equation}
\mathcal{L}_{\text{Yukawa}} = -y_f \bar{\psi}_L \Phi \psi_R + \text{h.c.}
\end{equation}
After symmetry breaking, these generate fermion masses $m_f = y_f v/\sqrt{2}$.

\subsection{The Top Quark}

The top quark, discovered at the Tevatron in 1995~\cite{Abe:1995hr,Abachi:1995iq}, plays a special role in this analysis as the production mechanism for light charged Higgs bosons. With a mass of $m_t \approx 172.5\GeV$~\cite{PDG2024}, it is by far the heaviest known fundamental particle.

The top quark's large mass results in an extremely short lifetime ($\tau_t \sim 5 \times 10^{-25}$~s), shorter than the hadronization timescale. Consequently, top quarks decay before forming hadrons, allowing their properties to be studied through their decay products. In the SM, the top quark decays almost exclusively via $\PQt \to \PWp\PQb$ with a branching fraction of $\approx 99.8\%$.

At hadron colliders, top quarks are predominantly produced in pairs through the strong interaction:
\begin{equation}
\Pp\Pp \to \ttbar + X
\end{equation}
At the LHC with $\sqrt{s} = 13\TeV$, the $\ttbar$ production cross section is approximately 830~pb~\cite{Czakon:2011xx}, making it one of the most abundant heavy particle production processes.

\section{Limitations of the Standard Model}
\label{sec:theory_bsm}

Despite its remarkable success, the SM fails to explain several observed phenomena and exhibits theoretical features that suggest it is not the ultimate theory of particle physics.

\subsection{Dark Matter}

Cosmological and astrophysical observations provide compelling evidence for dark matter, constituting approximately 27\% of the universe's energy density~\cite{Planck:2018vyg}. The SM contains no viable dark matter candidate, as all SM particles either decay rapidly or would have been detected through their interactions.

\subsection{Matter-Antimatter Asymmetry}

The observed universe consists almost entirely of matter rather than antimatter. While the SM contains sources of CP violation through the CKM matrix, the magnitude is insufficient to explain the observed baryon asymmetry~\cite{Sakharov:1967dj}.

\subsection{Neutrino Masses}

Neutrino oscillation experiments have definitively established that neutrinos have non-zero masses~\cite{PhysRevLett.81.1158,SNO:2002tuh}. In the minimal SM, neutrinos are massless, and their mass generation requires new physics.

\subsection{The Hierarchy Problem}

The Higgs boson mass receives large quantum corrections from loops involving heavy particles. Without fine-tuning, these corrections would drive the Higgs mass to the Planck scale. This ``naturalness'' problem suggests the existence of new physics at the TeV scale to stabilize the electroweak scale.

\subsection{Strong CP Problem}

QCD allows a CP-violating term proportional to $\bar{\theta}$. Experimental bounds on the neutron electric dipole moment constrain $|\bar{\theta}| < 10^{-10}$~\cite{Abel:2020pzs}, requiring an unexplained fine-tuning.

\section{Extended Higgs Sectors}
\label{sec:theory_extended}

The SM Higgs sector, while minimal, is not unique. Extended Higgs sectors with additional scalar fields are motivated by several theoretical considerations:

\begin{itemize}
\item Supersymmetry requires at least two Higgs doublets to give masses to both up-type and down-type fermions
\item Additional Higgs doublets can provide new sources of CP violation for baryogenesis
\item Extended scalar sectors can accommodate dark matter candidates
\item The Peccei-Quinn solution to the strong CP problem involves an additional scalar field
\end{itemize}

The simplest extension is the Two-Higgs-Doublet Model (2HDM), which introduces a second $\SU(2)_L$ doublet while maintaining the SM gauge structure.

\section{The Two-Higgs-Doublet Model}
\label{sec:theory_2hdm}

\subsection{Scalar Potential and Symmetry Breaking}

The 2HDM introduces two complex $\SU(2)_L$ doublets $\Phi_1$ and $\Phi_2$, each with hypercharge $Y = +1/2$:
\begin{equation}
\Phi_i = \begin{pmatrix} \phi_i^+ \\ \phi_i^0 \end{pmatrix}, \quad i = 1,2
\end{equation}

The most general renormalizable scalar potential respecting the gauge symmetry is:
\begin{align}
V(\Phi_1, \Phi_2) &= m_{11}^2 \Phi_1^\dagger\Phi_1 + m_{22}^2 \Phi_2^\dagger\Phi_2 - m_{12}^2(\Phi_1^\dagger\Phi_2 + \text{h.c.}) \nonumber\\
&+ \frac{\lambda_1}{2}(\Phi_1^\dagger\Phi_1)^2 + \frac{\lambda_2}{2}(\Phi_2^\dagger\Phi_2)^2 + \lambda_3(\Phi_1^\dagger\Phi_1)(\Phi_2^\dagger\Phi_2) \nonumber\\
&+ \lambda_4(\Phi_1^\dagger\Phi_2)(\Phi_2^\dagger\Phi_1) + \frac{\lambda_5}{2}\left[(\Phi_1^\dagger\Phi_2)^2 + \text{h.c.}\right]
\end{align}
where we have imposed a softly broken $\mathbb{Z}_2$ symmetry ($\Phi_1 \to \Phi_1$, $\Phi_2 \to -\Phi_2$) to avoid tree-level flavor-changing neutral currents (FCNCs).

After electroweak symmetry breaking, both doublets can acquire vevs:
\begin{equation}
\langle\Phi_i\rangle = \frac{1}{\sqrt{2}}\begin{pmatrix} 0 \\ v_i \end{pmatrix}
\end{equation}
with $v^2 = v_1^2 + v_2^2 = (246\GeV)^2$. The ratio of vevs defines an important parameter:
\begin{equation}
\tan\beta \equiv \frac{v_2}{v_1}
\end{equation}

\subsection{Physical Higgs Bosons}

The eight degrees of freedom in the two doublets rearrange as follows after symmetry breaking:
\begin{itemize}
\item Three Goldstone bosons ($G^\pm$, $G^0$) become the longitudinal modes of \PWpm and \PZ
\item Five physical Higgs bosons remain: \Ph, \PH (CP-even), \PA (CP-odd), and \PHpm (charged)
\end{itemize}

In the CP-conserving case, the neutral CP-even states mix through an angle $\alpha$:
\begin{equation}
\begin{pmatrix} \Ph \\ \PH \end{pmatrix} = \begin{pmatrix} \cos\alpha & \sin\alpha \\ -\sin\alpha & \cos\alpha \end{pmatrix} \begin{pmatrix} \rho_1 \\ \rho_2 \end{pmatrix}
\end{equation}
where $\rho_{1,2}$ are the neutral CP-even components of $\Phi_{1,2}$.

The masses of the physical Higgs bosons are related to the potential parameters:
\begin{align}
m_{\Ph,\PH}^2 &= \frac{1}{2}\left[(A_{11} + A_{22}) \mp \sqrt{(A_{11} - A_{22})^2 + 4A_{12}^2}\right] \\
m_\PA^2 &= m_{12}^2\left(\frac{1}{s_\beta c_\beta}\right) - \lambda_5 v^2 \\
m_{\PHpm}^2 &= m_\PA^2 + \frac{1}{2}(\lambda_5 - \lambda_4)v^2
\end{align}
where $s_\beta = \sin\beta$, $c_\beta = \cos\beta$, and $A_{ij}$ are elements of the CP-even mass matrix.

\subsection{Types of 2HDM}

To avoid FCNCs at tree level, discrete symmetries are imposed such that each type of fermion couples to only one Higgs doublet. This leads to four distinct types of 2HDM:

\begin{table}[htbp]
\centering
\caption{Yukawa coupling structure in the four types of 2HDM. The table shows which Higgs doublet couples to each type of fermion.}
\label{tab:2hdm_types}
\begin{tabular}{ccccc}
\hline\hline
Type & Up-type quarks & Down-type quarks & Charged leptons \\
\hline
Type-I & $\Phi_2$ & $\Phi_2$ & $\Phi_2$ \\
Type-II & $\Phi_2$ & $\Phi_1$ & $\Phi_1$ \\
Type-X (Lepton-specific) & $\Phi_2$ & $\Phi_2$ & $\Phi_1$ \\
Type-Y (Flipped) & $\Phi_2$ & $\Phi_1$ & $\Phi_2$ \\
\hline\hline
\end{tabular}
\end{table}

This analysis is primarily motivated by the Type-I 2HDM, where all fermions couple to $\Phi_2$. In this scenario, the couplings of the charged Higgs to fermions are proportional to $\cot\beta$, and for large \tanb, these couplings become suppressed.

\subsection{The Alignment Limit}

The discovered 125~GeV Higgs boson has properties consistent with the SM predictions within experimental uncertainties~\cite{ATLAS:2022vkf,CMS:2022dwd}. This constrains the 2HDM parameter space to be near the ``alignment limit,'' where one of the CP-even Higgs bosons has SM-like couplings.

The alignment limit is characterized by $\cos(\beta - \alpha) \to 0$ or equivalently $\sin(\beta - \alpha) \to 1$. In this limit, \Ph has SM-like couplings, while \PH couples to vector bosons proportionally to $\cos(\beta - \alpha)$.

\section{Charged Higgs Boson Phenomenology}
\label{sec:theory_charged}

\subsection{Production Mechanisms}

For $\mHc < \mt$, charged Higgs bosons are predominantly produced through top quark decays:
\begin{equation}
\PQt \to \PHc\PQb
\end{equation}
competing with the SM decay $\PQt \to \PWp\PQb$. The branching fraction depends on \mHc and \tanb:
\begin{equation}
\mathcal{B}(\PQt \to \PHc\PQb) \propto \left(\frac{m_t^2}{v^2}\cot^2\beta + \frac{m_b^2}{v^2}\tan^2\beta\right)\left(1 - \frac{m_{\PHc}^2}{m_t^2}\right)^2
\end{equation}

At low \tanb, the $m_t$ term dominates, while at high \tanb, the $m_b$ term can become important (particularly in Type-II models). In Type-I models, both terms are suppressed at high \tanb.

For $\mHc > \mt$, production occurs through:
\begin{itemize}
\item Associated production with top quark: $\Pg\Pg/\Pg\PQb \to \PHc\PQt\PQb$
\item Pair production: $\Pq\Paq \to \PHc\PHm$
\end{itemize}

\subsection{Decay Modes}

The charged Higgs can decay through both fermionic and bosonic channels:

\textbf{Fermionic decays:}
\begin{itemize}
\item $\PHpm \to \PQt\PAQb$ (kinematically forbidden for $\mHc < \mt$)
\item $\PHpm \to \PGtp\PGn$
\item $\PHpm \to \PQc\PAQs$, $\PHpm \to \PQc\PAQb$
\end{itemize}

\textbf{Bosonic decays:}
\begin{itemize}
\item $\PHpm \to \PWpm\Ph$
\item $\PHpm \to \PWpm\PA$
\item $\PHpm \to \PWpm\PH$
\end{itemize}

The partial widths for fermionic decays in a Type-I 2HDM are:
\begin{align}
\Gamma(\PHpm \to \PQt\PAQb) &\propto \frac{m_t^2 + m_b^2}{v^2}\cot^2\beta \\
\Gamma(\PHpm \to \PGtp\PGn) &\propto \frac{m_\tau^2}{v^2}\cot^2\beta
\end{align}

The bosonic decay $\PHpm \to \PWpm\PA$ has partial width:
\begin{equation}
\Gamma(\PHpm \to \PWpm\PA) = \frac{g^2}{64\pi}\frac{\cos^2(\beta-\alpha)}{m_\PW^2}\lambda^{3/2}(m_{\PHc}^2, m_\PW^2, m_\PA^2) \cdot m_{\PHc}
\end{equation}
where $\lambda(a,b,c) = (1 - b/a - c/a)^2 - 4bc/a^2$ is the Källén function.

In the alignment limit ($\cos(\beta-\alpha) \to 0$), this decay is suppressed. However, away from exact alignment, particularly for large \tanb in Type-I models where fermionic decays are suppressed, the bosonic decay $\PHpm \to \PWpm\PA$ can become dominant.

\subsection{The Target Decay Chain}

This analysis targets the decay chain:
\begin{equation}
\PHpm \to \PWpm\PA \to \PWpm\PGmp\PGmm
\end{equation}

The pseudoscalar \PA decays to fermion pairs with branching fractions proportional to the squared fermion masses. For $\mA > 2m_\Pgm$ and below the $\PQb\PAQb$ threshold, the dimuon channel provides a clean experimental signature:
\begin{equation}
\mathcal{B}(\PA \to \PGmp\PGmm) \approx \frac{m_\mu^2}{m_\tau^2 + m_c^2 + 3m_\mu^2} \approx 3\%
\end{equation}
for $2m_\mu < \mA < 2m_\tau$. For heavier \PA masses, the $\tau\tau$ and $\PQb\PAQb$ channels dominate.

\subsection{Off-Shell Decays}

An important feature of this analysis is the inclusion of off-shell $\PHpm \to \PWpm\PA$ decays. When $\mA > \mHc - m_\PW$, the \PW boson in the decay is virtual. The three-body decay width can be written as:
\begin{equation}
\Gamma(\PHpm \to \PA\ell\nu) = \int \frac{d\Gamma(\PHpm \to \text{W}^{\pm *}\PA)}{dm_{W^*}^2} \cdot \mathcal{B}(\text{W}^{\pm *} \to \ell\nu) \, dm_{W^*}^2
\end{equation}

While the off-shell decay rate is suppressed compared to on-shell decays, it remains significant in regions of parameter space where fermionic decays are suppressed. Calculations using 2HDMC~\cite{2HDMC} show that the product of branching fractions $\mathcal{B}(\PQt \to \PHc\PQb) \times \mathcal{B}(\PHc \to \PWpm\PA) \times \mathcal{B}(\PA \to \PGmp\PGmm)$ can reach values accessible at the LHC even in the off-shell region.

\section{Previous Searches and Current Constraints}
\label{sec:theory_constraints}

\subsection{LEP and Tevatron Results}

At LEP, the DELPHI and OPAL experiments searched for charged Higgs pair production $\Pe^+\Pe^- \to \PHc\PHm$ and set lower bounds $\mHc > 72$--80\GeV depending on the assumed decay modes~\cite{DELPHIChargedHiggs,OPALChargedHiggs}.

The CDF experiment at the Tevatron searched for $\PHpm \to \PWpm\PA \to \PWpm\PGtp\PGtm$ and set limits on the branching fraction~\cite{CDFChargedHiggs}.

\subsection{LHC Run 1 and Run 2 Results}

During LHC Run 1 and Run 2, both ATLAS and CMS performed extensive searches for charged Higgs bosons:
\begin{itemize}
\item $\PHpm \to \PGtp\PGn$: Strong constraints for $\mHc < \mt$~\cite{ATLAS:2018gfm,CMS:2019bfg}
\item $\PHpm \to \PQt\PAQb$: Constraints for $\mHc > \mt$~\cite{ATLAS:2021upq,CMS:2020imj}
\item $\PHpm \to \PQc\PAQs/\PQc\PAQb$: Constraints at low \tanb~\cite{ATLAS:2024oqu,CMS:2024iqz}
\item $\PHpm \to \PWpm\PA$: Searches by CMS (35.9\fbinv)~\cite{CMSRun2ChargedHiggs} and ATLAS (138\fbinv)~\cite{ATLASRun2ChargedHiggs}
\end{itemize}

The previous CMS search for $\PHpm \to \PWpm\PA \to \PWpm\PGmp\PGmm$ using 2016 data (35.9\fbinv) found no significant excess and set upper limits on the signal cross section. The ATLAS search using full Run 2 data similarly found no evidence for charged Higgs production in this channel.

\subsection{Theoretical Constraints}

The 2HDM parameter space is constrained by:
\begin{itemize}
\item \textbf{Perturbativity}: Requiring quartic couplings remain perturbative
\item \textbf{Unitarity}: S-matrix unitarity bounds on scalar scattering amplitudes
\item \textbf{Vacuum stability}: The potential must be bounded from below
\item \textbf{Electroweak precision observables}: The $\rho$ parameter constrains mass splittings
\item \textbf{Flavor physics}: $\PQb \to \PQs\Pgg$, $B$ meson mixing, and other processes
\end{itemize}

These constraints define the viable parameter space for the search, guiding the choice of mass ranges and benchmark scenarios considered in this analysis.
