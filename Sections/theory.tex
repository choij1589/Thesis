\chapter{Theoretical Background}
\label{ch:theory}

This chapter provides the theoretical foundation for the search presented in this thesis. We begin with an overview of the Standard Model of particle physics, followed by a discussion of its limitations and the motivation for extended Higgs sectors. The Two-Higgs-Doublet Model is then introduced, with particular emphasis on the phenomenology of charged Higgs bosons relevant to this analysis.

\section{The Standard Model of Particle Physics}
\label{sec:theory_sm}

The Standard Model (SM) is a quantum field theory that describes the fundamental particles and their interactions through three of the four known fundamental forces: the electromagnetic, weak, and strong interactions. Gravity, while well-described by general relativity at macroscopic scales, is not incorporated into the SM framework.

\subsection{Spacetime Symmetries and Particle Classification}
\label{sec:spacetime_symmetries}

The construction of the Standard Model begins not with a list of particles, but with a principle: the laws of physics must take the same form regardless of where or when an experiment is performed, or which inertial frame the observer occupies. As Weinberg emphasized, the result of an experiment on Earth must be predicted by the same laws that would apply on the Moon---not because of any dynamical mechanism, but because the laws themselves respect the symmetries of spacetime~\cite{Weinberg:1995mt}.

This requirement of invariance under spacetime transformations is encoded in the Poincar\'{e} group---the group of all isometries of Minkowski spacetime, comprising translations, rotations, and Lorentz boosts. By Noether's theorem, each continuous symmetry implies a conserved quantity: translational invariance in space and time yields conservation of momentum and energy, while rotational invariance yields conservation of angular momentum.

The most profound consequence of these symmetries for particle physics is Wigner's classification~\cite{Wigner:1939cj}: elementary particles correspond to irreducible unitary representations of the Poincar\'{e} group, labeled by two invariants---mass $m \geq 0$ and spin $s = 0, \frac{1}{2}, 1, \frac{3}{2}, \ldots$. This is not merely a convenient labeling scheme; it is the \emph{definition} of what constitutes an elementary particle.

The structure of the Lorentz group further constrains what types of fields can describe particles. The Lie algebra of the proper orthochronous Lorentz group $\mathrm{SO}^+(3,1)$ admits, upon complexification, a decomposition into two copies of the $\mathfrak{su}(2)$ algebra:
\begin{equation}
\mathfrak{so}(3,1)_{\mathbb{C}} \cong \mathfrak{su}(2) \oplus \mathfrak{su}(2)
\end{equation}
Finite-dimensional representations are therefore labeled by a pair $(j_L, j_R)$, where each index independently takes values $0, \frac{1}{2}, 1, \ldots$. The familiar fields of particle physics correspond to specific representations in this classification:
\begin{itemize}
\item $(0, 0)$: scalar field (e.g., the Higgs boson)
\item $(\frac{1}{2}, 0)$: left-handed Weyl spinor
\item $(0, \frac{1}{2})$: right-handed Weyl spinor
\item $(\frac{1}{2}, 0) \oplus (0, \frac{1}{2})$: Dirac spinor (massive fermions)
\item $(\frac{1}{2}, \frac{1}{2})$: four-vector field (gauge bosons)
\end{itemize}
The existence of two distinct spinor representations---left-handed and right-handed---is not imposed by hand but follows from the structure of spacetime itself. Their physical distinction through the weak interaction is discussed in Section~\ref{sec:gauge_structure}.

The union of quantum mechanics with special relativity carries a further consequence. Consistent quantization of fields on Minkowski spacetime---requiring Lorentz invariance, unitarity, and locality---leads to the CPT theorem: every particle must have a corresponding antiparticle with identical mass and spin but opposite internal quantum numbers. The spin-statistics theorem, arising from the same requirements, dictates that the half-integer-spin fermions described by spinor representations obey Fermi--Dirac statistics, while integer-spin bosons obey Bose--Einstein statistics. Matter and antimatter are thus not independent postulates but inevitable consequences of relativistic quantum field theory~\cite{Streater:1989vi}.

The experimentally observed matter content of the SM consists of twelve spin-$\frac{1}{2}$ fermions organized into three generations, each containing two quarks and two leptons, summarized in Table~\ref{tab:sm_fermions}.

\begin{table}[htbp]
\centering
\caption{The three generations of fermions in the Standard Model with their electric charges and approximate masses~\cite{PDG2024}.}
\label{tab:sm_fermions}
\begin{tabular}{lllcl}
\hline\hline
Generation & Particle & Type & Charge & Mass \\
\hline
First & up (\PQu) & quark & $+2/3$ & $\sim$2.2 MeV \\
      & down (\PQd) & quark & $-1/3$ & $\sim$4.7 MeV \\
      & electron (\Pe) & lepton & $-1$ & 0.511 MeV \\
      & $\nu_e$ & neutrino & $0$ & $<$1 eV \\
\hline
Second & charm (\PQc) & quark & $+2/3$ & $\sim$1.27 GeV \\
       & strange (\PQs) & quark & $-1/3$ & $\sim$93 MeV \\
       & muon (\Pgm) & lepton & $-1$ & 105.7 MeV \\
       & $\nu_\mu$ & neutrino & $0$ & $<$1 eV \\
\hline
Third & top (\PQt) & quark & $+2/3$ & $\sim$172.8 GeV \\
      & bottom (\PQb) & quark & $-1/3$ & $\sim$4.18 GeV \\
      & tau (\Pgt) & lepton & $-1$ & 1.777 GeV \\
      & $\nu_\tau$ & neutrino & $0$ & $<$1 eV \\
\hline\hline
\end{tabular}
\end{table}

Quarks carry color charge and participate in all three SM interactions, while leptons do not carry color charge and are thus absent from the strong interaction. This generational structure was not predicted from first principles but emerged from successive experimental discoveries---from the muon (1936) and strange quark physics, through the electroweak unification by Glashow, Weinberg, and Salam~\cite{Glashow:1961tr,Weinberg:1967tq,Salam:1968rm}, to the third-generation particles discovered between 1975 and 1995.

\subsection{Gauge Structure and Interactions}
\label{sec:gauge_structure}

The SM is based on the gauge symmetry group:
\begin{equation}
G_{\text{SM}} = \SU(3)_C \times \SU(2)_L \times \U(1)_Y
\end{equation}
where $\SU(3)_C$ describes the strong interaction (quantum chromodynamics, QCD), and $\SU(2)_L \times \U(1)_Y$ describes the electroweak interaction before symmetry breaking.

\subsubsection{The Gauge Principle}

The SM interactions follow from a single organizing principle: the Lagrangian must be invariant under \emph{local} gauge transformations---independent phase rotations at each spacetime point. The ordinary derivative $\partial_\mu$ is not covariant under such transformations, necessitating a connection---the covariant derivative $D_\mu = \partial_\mu - igA_\mu^a T^a$---that parallel-transports symmetry charges between neighboring points. The connection fields $A_\mu^a$ are precisely the gauge bosons, one for each generator of the symmetry group: eight gluons for $\SU(3)_C$, three weak bosons ($W^1$, $W^2$, $W^3$) for $\SU(2)_L$, and one hypercharge boson ($B$) for $\U(1)_Y$. Force-carrying particles thus arise not as ad hoc additions but as geometric consequences of local symmetry.

\subsubsection{Parity Violation and the Chiral Structure of Weak Interactions}

As discussed in Section~\ref{sec:spacetime_symmetries}, the Lorentz group admits two distinct spinor representations---left-handed $(\frac{1}{2}, 0)$ and right-handed $(0, \frac{1}{2})$. The electromagnetic and strong interactions treat these identically, preserving parity. It was therefore a profound surprise when Wu's 1957 experiment on cobalt-60 beta decay~\cite{Wu:1957my} demonstrated that the weak interaction maximally violates parity, coupling exclusively to left-handed fermions and right-handed antifermions.

This observation was codified in the $V{-}A$ (vector minus axial-vector) theory of Feynman and Gell-Mann~\cite{Feynman:1958ty} and Sudarshan and Marshak~\cite{Sudarshan:1958vf}, which describes the charged weak current as:
\begin{equation}
J^\mu_{\text{CC}} \propto \bar{\psi}\gamma^\mu(1 - \gamma^5)\psi
\end{equation}
The projection operator $(1 - \gamma^5)/2$ selects only the left-handed component of the fermion field, implementing maximal parity violation. This empirical structure finds its natural explanation in the gauge framework: left-handed fermions transform as $\SU(2)_L$ doublets, while right-handed fermions are $\SU(2)_L$ singlets. The subscript ``$L$'' in $\SU(2)_L$ directly encodes this experimental fact.

Left-handed fermions are thus organized into $\SU(2)_L$ doublets:
\begin{equation}
\ell_L = \begin{pmatrix} \nu_e \\ e^- \end{pmatrix}_L, \quad
q_L = \begin{pmatrix} u \\ d \end{pmatrix}_L
\end{equation}
while right-handed fermions are $\SU(2)_L$ singlets: $e_R$, $u_R$, $d_R$ (and similarly for the other generations). In the minimal SM, right-handed neutrinos $\nu_R$ are not included. The $\SU(2)_L$ generators act non-trivially only on left-handed doublets, so that weak charged currents couple exclusively to left-handed fermions (and right-handed antifermions).

\subsubsection{Quantum Number Assignments}

Each fermion chirality state carries conserved charges associated with the gauge symmetries. The weak isospin $I_3$ is the eigenvalue of the $\SU(2)_L$ generator $T^3$, and the weak hypercharge $Y$ is the $\U(1)_Y$ charge. After electroweak symmetry breaking, the unbroken $\U(1)_{\text{em}}$ subgroup defines the electric charge through the Gell-Mann--Nishijima relation:
\begin{equation}
Q = I_3 + Y
\label{eq:gellmann_nishijima}
\end{equation}

Table~\ref{tab:fermion_quantum_numbers} summarizes the quantum number assignments for the first generation of fermions. The pattern repeats for the second and third generations.

\begin{table}[htbp]
\centering
\caption{Quantum number assignments for the first generation of fermions under $\SU(2)_L \times \U(1)_Y$. The electric charge $Q$ is related to weak isospin $I_3$ and hypercharge $Y$ through $Q = I_3 + Y$. The pattern repeats for the second and third generations.}
\label{tab:fermion_quantum_numbers}
\begin{tabular}{lccccc}
\hline\hline
Particle & Chirality & $I_3$ & $Y$ & $Q$ & Representation \\
\hline
$\nu_e$ & Left & $+1/2$ & $-1/2$ & $0$ & \multirow{2}{*}{$\SU(2)_L$ doublet} \\
$e^-$ & Left & $-1/2$ & $-1/2$ & $-1$ & \\
$u$ & Left & $+1/2$ & $+1/6$ & $+2/3$ & \multirow{2}{*}{$\SU(2)_L$ doublet} \\
$d$ & Left & $-1/2$ & $+1/6$ & $-1/3$ & \\
\hline
$\nu_e$ & Right & $0$ & $0$ & $0$ & $\SU(2)_L$ singlet \\
$e^-$ & Right & $0$ & $-1$ & $-1$ & $\SU(2)_L$ singlet \\
$u$ & Right & $0$ & $+2/3$ & $+2/3$ & $\SU(2)_L$ singlet \\
$d$ & Right & $0$ & $-1/3$ & $-1/3$ & $\SU(2)_L$ singlet \\
\hline\hline
\end{tabular}
\end{table}

Note that left-handed fermions have $I_3 = \pm 1/2$ (doublet members), while right-handed fermions have $I_3 = 0$ (singlets). The hypercharge $Y$ differs from the electric charge $Q$; for example, the left-handed electron has $Y = -1/2$ but $Q = -1$.

\subsubsection{Gauge Bosons and Couplings}

The gauge principle predicts massless gauge bosons in the gauge eigenstate basis: three fields $A^1_\mu$, $A^2_\mu$, $A^3_\mu$ for $\SU(2)_L$ and one field $B_\mu$ for $\U(1)_Y$. However, the observed physical spectrum---the massive $\PW^\pm$ and $\PZ$ bosons alongside the massless photon $\Pgg$---requires a mechanism to generate masses while preserving gauge invariance. This mechanism, electroweak symmetry breaking, is the subject of Section~\ref{sec:ewsb}.

The physical mass eigenstates arise from mixing between the gauge eigenstates. The charged $\PWpm$ bosons are combinations of $A^1_\mu$ and $A^2_\mu$, while the neutral $\PZ$ boson and photon $\Pgg$ are rotations of $A^3_\mu$ and $B_\mu$ through the weak mixing angle $\theta_W$:
\begin{equation}
\tan\theta_W = \frac{g'}{g}
\end{equation}
where $g$ and $g'$ are the $\SU(2)_L$ and $\U(1)_Y$ coupling constants, with the experimentally measured value $\swsq \approx 0.23$~\cite{PDG2024}. The elementary electric charge is:
\begin{equation}
e = g \sin\theta_W = g' \cos\theta_W
\label{eq:electric_charge}
\end{equation}
The photon couples universally to electric charge $Q$, while the $\PZ$ boson couples to the combination:
\begin{equation}
Q_Z = I_3 - \swsq Q
\label{eq:z_coupling}
\end{equation}
which differs between fermion species depending on their quantum numbers as shown in Table~\ref{tab:fermion_quantum_numbers}.

\subsection{Electroweak Symmetry Breaking and the Higgs Mechanism}
\label{sec:ewsb}

Direct mass terms such as $\frac{1}{2}m^2 W_\mu W^\mu$ explicitly break gauge invariance, destroying the cancellations that ensure renormalizability. The Brout--Englert--Higgs mechanism~\cite{Higgs:1964pj,Englert:1964et,Guralnik:1964eu} resolves this through spontaneous symmetry breaking: the gauge symmetry is preserved in the Lagrangian but broken by the vacuum state, generating gauge boson masses dynamically while maintaining renormalizability.

\subsubsection{The Higgs Doublet and Spontaneous Symmetry Breaking}

The SM introduces a complex scalar $\SU(2)_L$ doublet with hypercharge $Y = +1/2$:
\begin{equation}
\Phi = \begin{pmatrix} \phi^+ \\ \phi^0 \end{pmatrix}
\end{equation}
with four real degrees of freedom. The most general renormalizable scalar potential is:
\begin{equation}
V(\Phi) = -\mu^2 \Phi^\dagger\Phi + \lambda(\Phi^\dagger\Phi)^2
\label{eq:higgs_potential}
\end{equation}
For $\mu^2 > 0$ and $\lambda > 0$, the minimum is not at $\Phi = 0$ but on a circle of degenerate vacua with $|\Phi|^2 = \mu^2/(2\lambda)$. The system selects one vacuum, spontaneously breaking $\SU(2)_L \times \U(1)_Y$ to $\U(1)_{\text{em}}$:
\begin{equation}
\langle\Phi\rangle = \frac{1}{\sqrt{2}}\begin{pmatrix} 0 \\ v \end{pmatrix}, \quad v = \frac{\mu}{\sqrt{\lambda}} \approx 246\GeV
\label{eq:higgs_vev}
\end{equation}

Of the four generators of $\SU(2)_L \times \U(1)_Y$, three are broken by this vacuum. By Goldstone's theorem, each broken generator produces a massless scalar---three Goldstone bosons in total. In a gauge theory, these are not physical particles: they are absorbed (``eaten'') by the $\PW^\pm$ and $\PZ$ bosons, providing the longitudinal polarization states required by massive vector bosons. In the unitary gauge, the Higgs field reduces to:
\begin{equation}
\Phi(x) = \frac{1}{\sqrt{2}}\begin{pmatrix} 0 \\ v + h(x) \end{pmatrix}
\label{eq:unitary_gauge}
\end{equation}
where $h(x)$ is the physical Higgs boson with mass $m_h = \sqrt{2\lambda}\,v \approx 125\GeV$~\cite{ATLAS:2012yda,CMS:2012qbp}. The four degrees of freedom of the original doublet thus rearrange into three longitudinal gauge boson polarizations plus one physical scalar.

\subsubsection{Gauge Boson Mass Generation}

Gauge boson masses emerge from the kinetic term $|D_\mu\Phi|^2$ evaluated at the vacuum. The covariant derivative acting on the Higgs doublet is:
\begin{equation}
D_\mu\Phi = \left(\partial_\mu - ig\frac{\sigma^a}{2}A^a_\mu - ig'Y B_\mu\right)\Phi
\label{eq:covariant_derivative}
\end{equation}
Substituting $\langle\Phi\rangle = (0, v/\sqrt{2})^T$, the terms quadratic in gauge fields yield masses. The charged combinations $W^\pm_\mu = (A^1_\mu \mp i A^2_\mu)/\sqrt{2}$ acquire mass:
\begin{equation}
m_W = \frac{gv}{2}
\label{eq:w_mass}
\end{equation}
The neutral fields $A^3_\mu$ and $B_\mu$ mix through the weak mixing angle $\theta_W$ to form the photon $A_\mu$ and $Z$ boson:
\begin{equation}
\begin{pmatrix} A_\mu \\ Z_\mu \end{pmatrix} =
\begin{pmatrix} \cos\theta_W & \sin\theta_W \\ -\sin\theta_W & \cos\theta_W \end{pmatrix}
\begin{pmatrix} B_\mu \\ A^3_\mu \end{pmatrix}
\label{eq:neutral_mixing}
\end{equation}
The photon remains massless, coupling to the unbroken $\U(1)_{\text{em}}$ generator, while the $\PZ$ boson acquires mass:
\begin{equation}
m_Z = \frac{v\sqrt{g^2 + g'^2}}{2} = \frac{m_W}{\cos\theta_W}
\label{eq:z_mass}
\end{equation}
The relation $m_Z = m_W/\cos\theta_W$ is a prediction of the symmetry breaking mechanism, confirmed experimentally to sub-percent accuracy~\cite{PDG2024}.

\subsubsection{Physical Interactions and Fermion Masses}

In the mass eigenstate basis, the electroweak covariant derivative takes the form:
\begin{equation}
D_\mu = \partial_\mu - ieA_\mu Q - i\frac{g}{\cos\theta_W}Z_\mu Q_Z - ig\left(W^+_\mu T^+ + W^-_\mu T^-\right)
\label{eq:covariant_mass_basis}
\end{equation}
where $T^\pm = (T^1 \pm iT^2)/\sqrt{2}$ are the weak isospin raising and lowering operators. The photon couples to electric charge $Q$, the $\PZ$ boson to $Q_Z = I_3 - \swsq Q$ (see Table~\ref{tab:fermion_quantum_numbers}), and the $\PWpm$ bosons mediate transitions between doublet partners ($u \leftrightarrow d$, $\nu_e \leftrightarrow e^-$).

Fermion masses cannot arise from direct mass terms $m\bar{\psi}\psi$, which would connect left-handed doublets to right-handed singlets and thus violate gauge invariance. Instead, they are generated through Yukawa couplings to the Higgs doublet:
\begin{equation}
\mathcal{L}_{\text{Yukawa}} = -y_f \bar{\psi}_L \Phi \psi_R + \text{h.c.}
\end{equation}
After symmetry breaking, these yield fermion masses $m_f = y_f v / \sqrt{2}$. Unlike gauge boson masses, which are predicted by $v$, $g$, and $g'$, the Yukawa couplings $y_f$ are free parameters spanning six orders of magnitude---from $y_e \sim 3 \times 10^{-6}$ to $y_t \sim 1$. This unexplained ``flavor hierarchy'' motivates many extensions of the SM.

\section{Motivations for an Extended Scalar Sector}
\label{sec:theory_bsm}

The Higgs sector of the Standard Model is minimal: a single $\SU(2)_L$ doublet suffices to break electroweak symmetry and generate all observed masses. However, nothing in the gauge structure forbids additional scalar fields, and extending the scalar sector is among the most conservative modifications to the SM. Remarkably, such extensions can address several of the SM's outstanding problems simultaneously.

Cosmological observations establish that approximately 27\% of the universe's energy density consists of dark matter~\cite{Planck:2018vyg}, for which the SM provides no candidate. An additional scalar doublet stabilized by a discrete $\mathbb{Z}_2$ symmetry can serve as a weakly interacting dark matter candidate. The observed matter--antimatter asymmetry of the universe requires CP violation beyond what the CKM matrix provides~\cite{Sakharov:1967dj}; additional Higgs doublets introduce new complex couplings in the scalar potential that can drive electroweak baryogenesis. Neutrino oscillation experiments have established non-zero neutrino masses~\cite{PhysRevLett.81.1158,SNO:2002tuh}, which the minimal SM cannot accommodate; extended scalar sectors offer mass generation mechanisms such as the type-II seesaw with an $\SU(2)_L$ triplet or radiative generation through an inert doublet.

Beyond these experimental anomalies, theoretical considerations also point toward extended scalar sectors. The Higgs boson mass receives quadratically divergent quantum corrections that would naturally drive it to the Planck scale---the hierarchy problem. Supersymmetry, which addresses this through partner-particle loop cancellations, requires at least two Higgs doublets to give masses to both up-type and down-type fermions. Similarly, the Peccei--Quinn solution to the strong CP problem---explaining why the QCD vacuum angle satisfies $|\bar{\theta}| < 10^{-10}$~\cite{Abel:2020pzs}---introduces an additional scalar field and predicts the axion~\cite{Peccei:1977hh}.

The simplest such extension that preserves the SM gauge structure is the Two-Higgs-Doublet Model, discussed in the following section.

\section{The Two-Higgs-Doublet Model}
\label{sec:theory_2hdm}

\subsection{General Review}

\subsubsection{Scalar Potential and Symmetry Breaking}

The 2HDM introduces two complex $\SU(2)_L$ doublets $\Phi_1$ and $\Phi_2$, each with hypercharge $Y = +1/2$:
\begin{equation}
\Phi_i = \begin{pmatrix} \phi_i^+ \\ \phi_i^0 \end{pmatrix}, \quad i = 1,2
\end{equation}

The most general renormalizable scalar potential respecting the gauge symmetry is:
\begin{align}
V(\Phi_1, \Phi_2) &= m_{11}^2 \Phi_1^\dagger\Phi_1 + m_{22}^2 \Phi_2^\dagger\Phi_2 - m_{12}^2(\Phi_1^\dagger\Phi_2 + \text{h.c.}) \nonumber\\
&+ \frac{\lambda_1}{2}(\Phi_1^\dagger\Phi_1)^2 + \frac{\lambda_2}{2}(\Phi_2^\dagger\Phi_2)^2 + \lambda_3(\Phi_1^\dagger\Phi_1)(\Phi_2^\dagger\Phi_2) \nonumber\\
&+ \lambda_4(\Phi_1^\dagger\Phi_2)(\Phi_2^\dagger\Phi_1) + \frac{\lambda_5}{2}\left[(\Phi_1^\dagger\Phi_2)^2 + \text{h.c.}\right]
\end{align}
where we have imposed a softly broken $\mathbb{Z}_2$ symmetry ($\Phi_1 \to \Phi_1$, $\Phi_2 \to -\Phi_2$) to avoid tree-level flavor-changing neutral currents (FCNCs).

After electroweak symmetry breaking, both doublets can acquire vevs:
\begin{equation}
\langle\Phi_i\rangle = \frac{1}{\sqrt{2}}\begin{pmatrix} 0 \\ v_i \end{pmatrix}
\end{equation}
with $v^2 = v_1^2 + v_2^2 = (246\GeV)^2$. The ratio of vevs defines an important parameter:
\begin{equation}
\tan\beta \equiv \frac{v_2}{v_1}
\end{equation}

\subsubsection{Physical Higgs Bosons}

The eight degrees of freedom in the two doublets rearrange as follows after symmetry breaking:
\begin{itemize}
\item Three Goldstone bosons ($G^\pm$, $G^0$) become the longitudinal modes of \PWpm and \PZ
\item Five physical Higgs bosons remain: \Ph, \PH (CP-even), \PA (CP-odd), and \PHpm (charged)
\end{itemize}

In the CP-conserving case, the neutral CP-even states mix through an angle $\alpha$:
\begin{equation}
\begin{pmatrix} \Ph \\ \PH \end{pmatrix} = \begin{pmatrix} \cos\alpha & \sin\alpha \\ -\sin\alpha & \cos\alpha \end{pmatrix} \begin{pmatrix} \rho_1 \\ \rho_2 \end{pmatrix}
\end{equation}
where $\rho_{1,2}$ are the neutral CP-even components of $\Phi_{1,2}$.

The masses of the physical Higgs bosons are related to the potential parameters:
\begin{align}
m_{\Ph,\PH}^2 &= \frac{1}{2}\left[(A_{11} + A_{22}) \mp \sqrt{(A_{11} - A_{22})^2 + 4A_{12}^2}\right] \\
m_\PA^2 &= m_{12}^2\left(\frac{1}{s_\beta c_\beta}\right) - \lambda_5 v^2 \\
m_{\PHpm}^2 &= m_\PA^2 + \frac{1}{2}(\lambda_5 - \lambda_4)v^2
\end{align}
where $s_\beta = \sin\beta$, $c_\beta = \cos\beta$, and $A_{ij}$ are elements of the CP-even mass matrix.

\subsubsection{Types of 2HDM}

To avoid FCNCs at tree level, discrete symmetries are imposed such that each type of fermion couples to only one Higgs doublet. This leads to four distinct types of 2HDM:

\begin{table}[htbp]
\centering
\caption{Yukawa coupling structure in the four types of 2HDM. The table shows which Higgs doublet couples to each type of fermion.}
\label{tab:2hdm_types}
\begin{tabular}{ccccc}
\hline\hline
Type & Up-type quarks & Down-type quarks & Charged leptons \\
\hline
Type-I & $\Phi_2$ & $\Phi_2$ & $\Phi_2$ \\
Type-II & $\Phi_2$ & $\Phi_1$ & $\Phi_1$ \\
Type-X (Lepton-specific) & $\Phi_2$ & $\Phi_2$ & $\Phi_1$ \\
Type-Y (Flipped) & $\Phi_2$ & $\Phi_1$ & $\Phi_2$ \\
\hline\hline
\end{tabular}
\end{table}

This analysis is primarily motivated by the Type-I 2HDM, where all fermions couple to $\Phi_2$. In this scenario, the couplings of the charged Higgs to fermions are proportional to $\cot\beta$, and for large \tanb, these couplings become suppressed.

\subsubsection{The Alignment Limit}

The discovered 125~GeV Higgs boson has properties consistent with the SM predictions within experimental uncertainties~\cite{ATLAS:2022vkf,CMS:2022dwd}. This constrains the 2HDM parameter space to be near the ``alignment limit,'' where one of the CP-even Higgs bosons has SM-like couplings.

The alignment limit is characterized by $\cos(\beta - \alpha) \to 0$ or equivalently $\sin(\beta - \alpha) \to 1$. In this limit, \Ph has SM-like couplings, while \PH couples to vector bosons proportionally to $\cos(\beta - \alpha)$.

\subsection{Charged Higgs Boson Phenomenology}
\label{sec:theory_charged}

\subsubsection{Production Mechanisms}

For $\mHc < \mt$, charged Higgs bosons are predominantly produced through top quark decays:
\begin{equation}
\PQt \to \PHc\PQb
\end{equation}
competing with the SM decay $\PQt \to \PWp\PQb$. The branching fraction depends on \mHc and \tanb:
\begin{equation}
\mathcal{B}(\PQt \to \PHc\PQb) \propto \left(\frac{m_t^2}{v^2}\cot^2\beta + \frac{m_b^2}{v^2}\tan^2\beta\right)\left(1 - \frac{m_{\PHc}^2}{m_t^2}\right)^2
\end{equation}

At low \tanb, the $m_t$ term dominates, while at high \tanb, the $m_b$ term can become important (particularly in Type-II models). In Type-I models, both terms are suppressed at high \tanb.

For $\mHc > \mt$, production occurs through:
\begin{itemize}
\item Associated production with top quark: $\Pg\Pg/\Pg\PQb \to \PHc\PQt\PQb$
\item Pair production: $\Pq\Paq \to \PHc\PHm$
\end{itemize}

\subsubsection{Decay Modes}

The charged Higgs can decay through both fermionic and bosonic channels:

\textbf{Fermionic decays:}
\begin{itemize}
\item $\PHpm \to \PQt\PAQb$ (kinematically forbidden for $\mHc < \mt$)
\item $\PHpm \to \PGtp\PGn$
\item $\PHpm \to \PQc\PAQs$, $\PHpm \to \PQc\PAQb$
\end{itemize}

\textbf{Bosonic decays:}
\begin{itemize}
\item $\PHpm \to \PWpm\Ph$
\item $\PHpm \to \PWpm\PA$
\item $\PHpm \to \PWpm\PH$
\end{itemize}

The partial widths for fermionic decays in a Type-I 2HDM are:
\begin{align}
\Gamma(\PHpm \to \PQt\PAQb) &\propto \frac{m_t^2 + m_b^2}{v^2}\cot^2\beta \\
\Gamma(\PHpm \to \PGtp\PGn) &\propto \frac{m_\tau^2}{v^2}\cot^2\beta
\end{align}

The bosonic decay $\PHpm \to \PWpm\PA$ has partial width:
\begin{equation}
\Gamma(\PHpm \to \PWpm\PA) = \frac{g^2}{64\pi}\frac{\cos^2(\beta-\alpha)}{m_\PW^2}\lambda^{3/2}(m_{\PHc}^2, m_\PW^2, m_\PA^2) \cdot m_{\PHc}
\end{equation}
where $\lambda(a,b,c) = (1 - b/a - c/a)^2 - 4bc/a^2$ is the Källén function.

In the alignment limit ($\cos(\beta-\alpha) \to 0$), this decay is suppressed. However, away from exact alignment, particularly for large \tanb in Type-I models where fermionic decays are suppressed, the bosonic decay $\PHpm \to \PWpm\PA$ can become dominant.

\subsubsection{The Target Decay Chain}

This analysis targets the decay chain:
\begin{equation}
\PHpm \to \PWpm\PA \to \PWpm\PGmp\PGmm
\end{equation}

The pseudoscalar \PA decays to fermion pairs with branching fractions proportional to the squared fermion masses. For $\mA > 2m_\Pgm$ and below the $\PQb\PAQb$ threshold, the dimuon channel provides a clean experimental signature:
\begin{equation}
\mathcal{B}(\PA \to \PGmp\PGmm) \approx \frac{m_\mu^2}{m_\tau^2 + m_c^2 + 3m_\mu^2} \approx 3\%
\end{equation}
for $2m_\mu < \mA < 2m_\tau$. For heavier \PA masses, the $\tau\tau$ and $\PQb\PAQb$ channels dominate.

\subsubsection{Off-Shell Decays}

An important feature of this analysis is the inclusion of off-shell $\PHpm \to \PWpm\PA$ decays. When $\mA > \mHc - m_\PW$, the \PW boson in the decay is virtual. The three-body decay width can be written as:
\begin{equation}
\Gamma(\PHpm \to \PA\ell\nu) = \int \frac{d\Gamma(\PHpm \to \text{W}^{\pm *}\PA)}{dm_{W^*}^2} \cdot \mathcal{B}(\text{W}^{\pm *} \to \ell\nu) \, dm_{W^*}^2
\end{equation}

While the off-shell decay rate is suppressed compared to on-shell decays, it remains significant in regions of parameter space where fermionic decays are suppressed. Calculations using 2HDMC~\cite{2HDMC} show that the product of branching fractions $\mathcal{B}(\PQt \to \PHc\PQb) \times \mathcal{B}(\PHc \to \PWpm\PA) \times \mathcal{B}(\PA \to \PGmp\PGmm)$ can reach values accessible at the LHC even in the off-shell region.

\subsection{Previous Searches and Current Constraints}
\label{sec:theory_constraints}

\subsubsection{LEP and Tevatron Results}

At LEP, the DELPHI and OPAL experiments searched for charged Higgs pair production $\Pe^+\Pe^- \to \PHc\PHm$ and set lower bounds $\mHc > 72$--80\GeV depending on the assumed decay modes~\cite{DELPHIChargedHiggs,OPALChargedHiggs}.

The CDF experiment at the Tevatron searched for $\PHpm \to \PWpm\PA \to \PWpm\PGtp\PGtm$ and set limits on the branching fraction~\cite{CDFChargedHiggs}.

\subsubsection{LHC Run 1 and Run 2 Results}

During LHC Run 1 and Run 2, both ATLAS and CMS performed extensive searches for charged Higgs bosons:
\begin{itemize}
\item $\PHpm \to \PGtp\PGn$: Strong constraints for $\mHc < \mt$~\cite{ATLAS:2018gfm,CMS:2019bfg}
\item $\PHpm \to \PQt\PAQb$: Constraints for $\mHc > \mt$~\cite{ATLAS:2021upq,CMS:2020imj}
\item $\PHpm \to \PQc\PAQs/\PQc\PAQb$: Constraints at low \tanb~\cite{ATLAS:2024oqu,CMS:2024iqz}
\item $\PHpm \to \PWpm\PA$: Searches by CMS (35.9\fbinv)~\cite{CMSRun2ChargedHiggs} and ATLAS (138\fbinv)~\cite{ATLASRun2ChargedHiggs}
\end{itemize}

The previous CMS search for $\PHpm \to \PWpm\PA \to \PWpm\PGmp\PGmm$ using 2016 data (35.9\fbinv) found no significant excess and set upper limits on the signal cross section. The ATLAS search using full Run 2 data similarly found no evidence for charged Higgs production in this channel.

\subsubsection{Theoretical Constraints}

The 2HDM parameter space is constrained by:
\begin{itemize}
\item \textbf{Perturbativity}: Requiring quartic couplings remain perturbative
\item \textbf{Unitarity}: S-matrix unitarity bounds on scalar scattering amplitudes
\item \textbf{Vacuum stability}: The potential must be bounded from below
\item \textbf{Electroweak precision observables}: The $\rho$ parameter constrains mass splittings
\item \textbf{Flavor physics}: $\PQb \to \PQs\Pgg$, $B$ meson mixing, and other processes
\end{itemize}

These constraints define the viable parameter space for the search, guiding the choice of mass ranges and benchmark scenarios considered in this analysis.
